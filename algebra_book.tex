%-------------------------Packages---------------------------------%

\documentclass[11pt, a4paper, titlepage, bibliography=totoc]{scrbook}

\usepackage[utf8]{inputenc}
\usepackage[T1]{fontenc}
\usepackage[ngerman]{babel}

\usepackage{enumitem}
\usepackage{extarrows}
\usepackage{amsmath, amssymb, amsthm}
\usepackage{stmaryrd}
\usepackage{calrsfs}
\usepackage{wasysym}

\usepackage{float}
\usepackage{graphicx}
\usepackage{mdframed}
\usepackage[most]{tcolorbox}
\usepackage{titlesec}

\usepackage{makeidx}
\makeindex

\usepackage{hyperref}

%----------------------------Titlesec-----------------------------------%

\usepackage{titlesec}
\titleformat{\chapter}[block]{\normalfont\Huge\bfseries}{§~\thechapter}{.5em}{}[]
\titleformat{\section}[block]{\hrule\bigskip\normalfont\Large\bfseries}{§~\thesection}{.5em}{}[\bigskip\hrule]
\titleformat{\subsection}[runin]{\normalfont\bfseries}{\thesubsection}{.5em}{}[]

%----------------------------Definitions-------------------------------%

% Mengen
\newcommand\N{\mathbb N}
\newcommand\Z{\mathbb Z}

% Operatoren
\DeclareMathOperator\Aut{Aut}
\DeclareMathOperator\GL{GL}
\DeclareMathOperator\sgn{sgn}

% Befehle zu Gruppen
\newcommand\GRPeqn{\stackrel{(N)}{=}}
\newcommand\GRPeqi{\stackrel{(I)}{=}}
\newcommand\GRPeqa{\stackrel{(A)}{=}}

% Verweise
\newcommand\LAref[1]{[$\to$ LA #1]}

% Beweise
\newcommand\bew{\textbf{Beweis:}~}

% Vereinfachungen
\newcommand\mo{{-1}}
\newcommand\ntn{{n \times n}}

% Sonstige
\setlength{\parindent}{0pt}
\newcommand{\HRule}{\rule{\linewidth}{0.5mm}}

%---------------------------Document--------------------------------------%

\begin{document}

\begin{titlepage}
	\begin{center}
		\textsc{\LARGE Universität Konstanz}\\[1.5cm]
		{\large Skriptum zur Vorlesung}\\[0.5cm]
		\HRule \\[0.4cm]
		{\huge \bfseries Einführung in die Algebra \\[0.4cm]}
		\HRule \\[.5cm]
		{\small Private Mitschrift}\\[1.5cm]
		{\large \emph{gelesen von:} \hfill Prof. Dr. Markus Schweighofer}\\
		\large\vfill
		{\large Wintersemester 2014/15}\\
		{\large Stand vom \today}	
	\end{center}
\end{titlepage}

\setcounter{tocdepth}{1}
\tableofcontents

\chapter{Gruppen}

\section{Gruppen und Untergruppen}

\subsection{Definition} Eine Gruppe\index{Gruppe} ist ein geordnetes Paar $(G, \cdot)$, wobei $G$ eine Menge ist und $\cdot : G \times G \to G$ eine meist infix (und manchmal gar nicht) notierte Abbildung mit folgenden Eigenschaften ist:
	\begin{itemize}
		\item[(A)]
			$\forall a, b, c \in G : a(bc) = (ab)c$ \quad\quad \textit{"`assoziativ"'}
		\item[(N)]
			$\exists e \in G ~\forall a \in G : ae = a = ea$ \quad\quad \textit{"`neutrales Element"'}
		\item[(I)]
			$\forall a \in G ~\exists g \in G : ab = 1 = ba$ \quad\quad \textit{"`inverse Elemente"'}
	\end{itemize}
"`$\cdot$"' heißt Gruppenmultiplikation\index{Gruppe!-nmultiplikation} oder Gruppenverknüpfung\index{Gruppe!-nverknüpfung}. Gilt zusätzlich
	\begin{itemize}
		\item[(K)] $\forall a,b \in G : ab = ba$
	\end{itemize}
so heißt $(G, \cdot)$ abelsch\index{Gruppe!abelsche} oder kommutativ\index{Gruppe!kommutative}.

\subsection*{Anmerkung} Sind $e, e' \in G$ neutral, so $e = ee' = e'$. Daher gibt es genau ein neutrales Element, für welches man oft "`$1$"' schreibt.

\subsection{Bemerkung}
	\begin{enumerate}[label=(\alph*)]
		\item
			Sei $(G, \cdot)$ eine Gruppe und $a \in G$. Seien $b, b'$ invers zu $a$. Dann
			$$b \GRPeqn b \cdot 1 \GRPeqi b(ab') \GRPeqa (ba)b' \GRPeqi 1 \cdot b \GRPeqn b'.$$
			Daher gibt es zu jedem $a \in G$ genau ein inverses Element in $G$, welches wir mit $a^\mo$ bezeichnen.
			
		\item
			(N) und (I) kann man wie folgt schreiben:
			\begin{itemize}
				\item[(N)] $\forall a \in G : a1 = a = 1a$
				\item[(I)] $\forall a \in G : aa^\mo = 1 = a^\mo a$
			\end{itemize}
			
		\item
			Oft: "`Sei $G$ eine Gruppe"', statt: "`Sei $(G, \cdot)$ eine Gruppe."'
		
		\item
			Sei $G$ eine Gruppe, $n \in \N_0$ und $a_1, ..., a_n \in G$. Dann definiert man $\prod_{i=1}^n a_i := a_1 \cdot ... \cdot a_n$ als 1 für $n = 0$ und indem man $a_1 \cdot ... \cdot a_n$ sinnvoll mit Klammern versieht, sonst. Dies hängt nicht von der Wahl der Klammerung da, wie (A) für $n = 3$ besagt. Für $n > 3$ siehe \LAref{2.1.6} oder mache es als Übung per Induktion. Falls $G$ additiv geschrieben ist, schreibt man $\sum_{i=1}^n a_i$, statt $\prod_{i=1}^n a_i$.
			
		\item
			Sei $G$ eine Gruppe, $n \in \Z$ und $a \in G$. Dann definiert man
			$$a^n := \begin{cases}\prod_{i=1}^n a, & \text{für} ~ n \geq 0, \\ \prod_{i=1}^n (a^\mo), & \text{für} ~ n \leq 0.\end{cases}$$
			Fall $G$ additiv geschrieben ist, schreibt man $na$, statt $a^n$.
	\end{enumerate}

\subsection{Definition} Ist $(G, \cdot)$ eine Gruppe, so nennt man $\#G \in \N_0 \cup \{\infty\}$ die Ordnung\index{Ordnung} von $(G,\cdot)$.

\subsection{Beispiel}
	\begin{enumerate}[label=(\alph*)]
		\item
			Für jede Menge $M$ bildet die Menge $S_M := \{f \mid f: M \to M ~ \text{bijektiv}\}$ mit der durch $fg := f \circ g$ $(f,g \in S_M)$ gegebenen Multiplikation eine Gruppe. Man nennt sie die symmetrische Gruppe\index{Gruppe!symmetrische} auf M. Das neutrale Element von $S_M$ ist die Identität auf $M$ und das zu einem $f \in S_M$ inverse Element ist die Umkehrfunktion von $f$, wodurch die Notation $f^\mo$ nicht zweideutig ist.
		
			Für $n \in \N_0$ ist $S_n := S_{\{1,...,n\}}$ eine Gruppe der Ordnung $n! := \prod_{i=1}^n i$ "`$n$ Fakultät\index{Fakultät}"'. Für $n \geq 3$ ist die nicht abelsch, dann die Transpositionen $\tau_{1,2}$ und $\tau_{2,3}$ konvertieren nicht, d.h. $\tau_{1,2}\tau_{2,3} \neq \tau_{2,3}\tau_{1,2}$. In der Tat: $(\tau_{1,2}\tau_{2,3})(1) = \tau_{1,2}(1) = 2$ und $(\tau_{2,3}\tau_{1,2})(1) = \tau_{2,3}(2) = 3$.
		
		\item
			Für jeden Vektorraum $V$ ist die Menge $\Aut(V) := \{f \mid f : V \to V ~\text{linear und bijektiv}\}$\index{Automorphismus!Vektorraum-} mit der Hintereinanderschaltung als Multiplikation eine Gruppe.
			
		\item
			Ist $R$ ein kommutativer Ring (z.\ B. $R = \Z$), so ist $\GL_n(R) := \{A \in R^\ntn \mid A ~\text{invertierbar}\} = \{ A \in R^\ntn \mid \det A \in R^\times \}$\index{General Linear Group} eine Gruppe.
	\end{enumerate}
	
\subsection{Proposition} Sei $G$ eine Gruppe und $a,b \in G$.
\begin{enumerate}[label=(\alph*)]
		\item
			$ab = 1 \iff a = b^\mo \iff b = a^\mo$
		\item
			$\left(a^\mo\right)^\mo = a$
		\item
			$(ab)^\mo = b^\mo a^\mo$
\end{enumerate}

\bew
\begin{enumerate}[label=(\alph*)]
	\item
		Gilt $ab = 1$, so $a \GRPeqn a1 \GRPeqi a(bb^\mo) \GRPeqa (ab) b^\mo = 1b \GRPeqn b^\mo$. Gilt $a = b^\mo$, so $b \GRPeqn 1b \GRPeqi (a^\mo a)b \GRPeqa a^\mo(ab) = a^\mo(b^\mo b) \GRPeqi a^\mo 1 \GRPeqn a^\mo$. Gilt $b = a^\mo$, so $ab = 1$.
		
	\item
		Aus $a a^\mo \GRPeqi 1$ folgt mit (a) $\left( a^\mo \right)^\mo = a$.
		
	\item
		Aus $(ab)(b^\mo a^\mo) \GRPeqa a(b(b^\mo a^\mo)) \GRPeqa a((b b^\mo)a^\mo) \GRPeqi a(1a^\mo) \GRPeqn aa^\mo \GRPeqi 1$ folgt mit (a) $(ab)^\mo = b^\mo a^\mo$. \qed		
\end{enumerate}

\subsection{Definition} Seien $(G, \cdot_G)$ und $(H, \cdot_H)$ Gruppen. Dann heißt $(H, \cdot_H)$ eine Untergruppe\index{Gruppe!Unter-} von $(G, \cdot_G)$, wenn $H \subseteq G$ und $\forall a, b \in H : a \cdot_H b = a \cdot_G b$.

\subsection{Proposition} Sei $(G, \cdot_G$ eine Gruppe und $H$ eine Menge. Dann ist $H$ genau dann Trägermenge\index{Träger} einer Untergruppe von $(G, \cdot_g)$, wenn $H \subseteq G$, $1_S \in H$, $\forall a,b \in H : a \cdot_S b \in H$ und $\forall a \in H : a^ \mo \in H$.

In diesem Fall gibt es genau eine Abbildung $\cdot_H : H \times H \to H$ derart, dass $(H, \cdot_H)$ eine Untergruppe von $(G, \cdot_G)$ ist. Es gilt dann $1_H = 1_G$, $\forall a,b \in H : a \cdot_H b = a \cdot_G b$ und $a^\mo = a^\mo$ (je in $G$ und $H$ gebildet).

\bew Klar oder vgl.\ LA § 2. \qed

\subsection{Bemerkung}
\begin{enumerate}[label=(\alph*)]
	\item
		Ist $(H, \cdot_H)$ Untergruppe von $(G, \cdot_G)$, so schreibt man meist $\cdot$ statt $\cdot_H$. Oft erwähnt man $\cdot_H$ gar nicht mehr und schreibt einfach "`$H$ ist Untergruppe von $G$"' oder $H \leq G$.
		
	\item
		Untergruppen abelscher Gruppen sind abelsch.
\end{enumerate}

\subsection{Beispiel}
\begin{enumerate}[label=(\alph*)]
	\item
		Für $n \in \N_0$ ist $A_n := \{\sigma \in S_n \mid \sgn \sigma = 1\}$ eine Untergruppe von $S_n$, die man alternierende Gruppe \index{Gruppe!alternierende} nennt. \LAref{§ 9.1}
\end{enumerate}

\nocite{*}
\bibliographystyle{plaindin}
\bibliography{algebra_book.bib}

\printindex

\end{document}