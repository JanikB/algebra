%-------------------------Packages---------------------------------%

\documentclass[11pt, a4paper, titlepage, bibliography=totoc]{scrbook}

\usepackage[utf8]{inputenc}
\usepackage[T1]{fontenc}
\usepackage[ngerman]{babel}

\usepackage{enumitem}
\usepackage{extarrows}
\usepackage{amsmath, amssymb, amsthm}
\usepackage{stmaryrd}
\usepackage{calrsfs}
\usepackage{wasysym}

\usepackage{float}
\usepackage{graphicx}
\usepackage{mdframed}
\usepackage{tcolorbox}
\usepackage{titlesec}

\usepackage{makeidx}
\makeindex

\usepackage[hidelinks]{hyperref}

\usepackage{tikz}
\usetikzlibrary{matrix, chains}

%----------------------------Titlesec-----------------------------------%

% In the subsection, there is some automatic label{}ing.
\usepackage{titlesec}
\titleformat{\chapter}[block]{\normalfont\Huge\bfseries}{§~\thechapter}{.5em}{}[]
\titleformat{\section}[block]{\hrule\bigskip\normalfont\Large\bfseries}{§~\thesection}{.5em}{}[\bigskip\hrule]
\titleformat{\subsection}[runin]{\normalfont\bfseries\label{fixed:\thesubsection}}{\thesubsection}{.5em}{}[]

%----------------------------Definitions-------------------------------%

% Mengen
\newcommand\N{\mathbb N}
\newcommand\Z{\mathbb Z}
\newcommand\Q{\mathbb Q}
\newcommand\R{\mathbb R}
\newcommand\C{\mathbb C}
\renewcommand\P{\mathbb P}
\newcommand\F{\mathbb F}
\newcommand\p{\mathfrak p}
\newcommand\m{\mathfrak m}
\newcommand\q{\mathfrak q}
\newcommand\Odm{
	\begin{tikzpicture}[baseline]
		\draw[fill=black] (1.6ex,0) -- (1.6ex,1.6ex) -- (0,1.6ex) -- cycle;
	\end{tikzpicture}
	~
}
\newcommand\Udm{
	\begin{tikzpicture}[baseline]
		\draw[fill=black] (0,0) -- (0,1.6ex) -- (1.6ex,0) -- cycle;
	\end{tikzpicture}
	~
}
\newcommand\odm{
	\begin{tikzpicture}[baseline]
		\draw (1.6ex,0) -- (1.6ex,1.6ex) -- (0,1.6ex) -- cycle;
	\end{tikzpicture}
	~
}
\newcommand\udm{
	\begin{tikzpicture}[baseline]
		\draw (0,0) -- (0,1.6ex) -- (1.6ex,0) -- cycle;
	\end{tikzpicture}
	~
}

% Operatoren
\DeclareMathOperator\Aut{Aut}
\DeclareMathOperator\Inn{Inn}
\DeclareMathOperator\End{End}
\DeclareMathOperator\GL{GL}
\DeclareMathOperator\SL{SL}
\DeclareMathOperator\SO{SO}
\DeclareMathOperator\sgn{sgn}
\DeclareMathOperator\im{im}
\DeclareMathOperator\spn{span}
\DeclareMathOperator\irr{irr}
\DeclareMathOperator\id{id}
\DeclareMathOperator\qf{qf}
\DeclareMathOperator\supp{supp}
\DeclareMathOperator\chara{char}
\DeclareMathOperator\ord{ord}

% Relationen
\newcommand{\eqhat}{\mathrel{\widehat{=}}}
\newcommand{\eqtilde}{\mathrel{\widetilde{=}}}

% Äquivalenzklassen
\newcommand{\eqsimleft}[1]{\mathrel{{}_#1\!\sim}}
\newcommand{\eqsimright}[1]{\mathrel{\sim_#1}}
\newcommand{\eqaccentleft}[2]{\overset{{}_#1\sim}{\phantom{{}_#1\!}#2}}
\newcommand{\eqaccentright}[2]{\overset{\sim_#2}{#1\phantom{{}_#2}}}

% Befehle zu Gruppen
\newcommand\GRPeqn{\stackrel{(N)}{=}}
\newcommand\GRPeqi{\stackrel{(I)}{=}}
\newcommand\GRPeqa{\stackrel{(A)}{=}}
\newcommand\normsub{\mathrel\triangleleft}
\newcommand\normsup{\mathrel\triangleright}

% Befehle zu Abbildungen
\newcommand\MAPmono{\lhook\joinrel\rightarrow}
\newcommand\MAPepi{\twoheadrightarrow}
\newcommand\MAPiso{\overset{\cong}{\rightarrow}}
\newcommand\MAPlongmono{\lhook\joinrel\longrightarrow}
\newcommand\MAPlongepi{\relbar\joinrel\twoheadrightarrow}
\newcommand\MAPlongiso{\overset{\cong}{\longrightarrow}}

% Verweise
\newcommand\LAref[1]{[$\to$ LA #1]}
\newcommand\ALref[1]{[$\to$ #1]}

% Beweise
\newcommand\bew{\textbf{Beweis:}~}

% Vereinfachungen
\newcommand\mo{{-1}}
\newcommand\ntn{{n \times n}}

% Verschiebe overline nach oldoverline.
% Fordere für overline mindestens Höhe eines 'A' Abstand zur Grundlinie.
\let\oldoverline\overline
\renewcommand\overline[1]{\oldoverline{\vphantom{A} #1}}

% Benutze immer overline.
\renewcommand\bar\overline
\newcommand\barsmash[2]{\smash{\overline{#1}}\vphantom{#1}^{#2}}
% Bessere Tilde für Äquivalenzklassen.
\newcommand\tildeidx[2]{\overset{\sim_{\scriptscriptstyle #2}}{#1\hphantom{\scriptscriptstyle #2}}}
% Benutze immer widehat.
\renewcommand\hat\widehat
% Benutze immer varphi statt phi und varepslion statt epsilon.
\renewcommand\phi\varphi
\renewcommand\epsilon\varepsilon

% Sonstige
\setlength{\parindent}{0pt}
\setlength{\parskip}{1em}
\newcommand{\HRule}{\rule{\linewidth}{0.5mm}}
\renewcommand\i{\mathring\imath}
\newcommand\numberfix[3]{
	\setcounter{chapter}{#1}
	\setcounter{section}{#2}
	\setcounter{subsection}{#3}
	
	\rule{\textwidth}{1pt}
	\texttt{Hier fehlt noch etwas ...}\\
	\rule{\textwidth}{1pt}
}

%---------------------------Document--------------------------------------%

\begin{document}

\KOMAoptions{twoside = false}
\begin{titlepage}
	\begin{center}
		\textsc{\LARGE Universität Konstanz}\\[1.5cm]
		{\large Skriptum zur Vorlesung}\\[0.5cm]
		\HRule \\[0.4cm]
		{\huge \bfseries Einführung in die Algebra \\[0.4cm]}
		\HRule \\[.5cm]
		{\small Private Mitschrift}\\[1.5cm]
		{\large \emph{gelesen von:} \hfill Prof. Dr. Markus Schweighofer}\\
		\large\vfill
		{\large Wintersemester 2014/15}\\
		{\large Stand vom \today}	
	\end{center}
\end{titlepage}
\KOMAoptions{twoside = true}

\setcounter{tocdepth}{1}
\tableofcontents

\chapter{Gruppen}

\section{Gruppen und Untergruppen}

\subsection{Definition} Eine Gruppe\index{Gruppe} ist ein geordnetes Paar $(G, \cdot)$, wobei $G$ eine Menge ist und $\cdot : G \times G \to G$ eine meist infix (und manchmal gar nicht) notierte Abbildung mit folgenden Eigenschaften ist:
	\begin{itemize}
		\item[(A)]
			$\forall a, b, c \in G : a(bc) = (ab)c$ \quad\quad \textit{"`assoziativ"'}
		\item[(N)]
			$\exists e \in G ~\forall a \in G : ae = a = ea$ \quad\quad \textit{"`neutrales Element"'}
		\item[(I)]
			$\forall a \in G ~\exists g \in G : ab = 1 = ba$ \quad\quad \textit{"`inverse Elemente"'}
	\end{itemize}
"`$\cdot$"' heißt Gruppenmultiplikation\index{Gruppe!-nmultiplikation} oder Gruppenverknüpfung\index{Gruppe!-nverknüpfung}. Gilt zusätzlich
	\begin{itemize}
		\item[(K)] $\forall a,b \in G : ab = ba$
	\end{itemize}
so heißt $(G, \cdot)$ abelsch\index{Gruppe!abelsche} oder kommutativ\index{Gruppe!kommutative}.

\subsection*{Anmerkung} Sind $e, e' \in G$ neutral, so $e = ee' = e'$. Daher gibt es genau ein neutrales Element, für welches man oft "`$1$"' schreibt.

\subsection{Bemerkung}
	\begin{enumerate}[label=(\alph*)]
		\item
			Sei $(G, \cdot)$ eine Gruppe und $a \in G$. Seien $b, b'$ invers zu $a$. Dann
			$$b \GRPeqn b \cdot 1 \GRPeqi b(ab') \GRPeqa (ba)b' \GRPeqi 1 \cdot b \GRPeqn b'.$$
			Daher gibt es zu jedem $a \in G$ genau ein inverses Element in $G$, welches wir mit $a^\mo$ bezeichnen.
			
		\item
			(N) und (I) kann man wie folgt schreiben:
			\begin{itemize}
				\item[(N)] $\forall a \in G : a1 = a = 1a$
				\item[(I)] $\forall a \in G : aa^\mo = 1 = a^\mo a$
			\end{itemize}
			
		\item
			Oft: "`Sei $G$ eine Gruppe"', statt: "`Sei $(G, \cdot)$ eine Gruppe."'
		
		\item
			Sei $G$ eine Gruppe, $n \in \N_0$ und $a_1, ..., a_n \in G$. Dann definiert man $\prod_{i=1}^n a_i := a_1 \cdot ... \cdot a_n$ als 1 für $n = 0$ und indem man $a_1 \cdot ... \cdot a_n$ sinnvoll mit Klammern versieht, sonst. Dies hängt nicht von der Wahl der Klammerung da, wie (A) für $n = 3$ besagt. Für $n > 3$ siehe \LAref{2.1.6} oder mache es als Übung per Induktion. Falls $G$ additiv geschrieben ist, schreibt man $\sum_{i=1}^n a_i$, statt $\prod_{i=1}^n a_i$.
			
		\item
			Sei $G$ eine Gruppe, $n \in \Z$ und $a \in G$. Dann definiert man
			$$a^n := \begin{cases}\prod_{i=1}^n a, & \text{für} ~ n \geq 0, \\ \prod_{i=1}^n (a^\mo), & \text{für} ~ n \leq 0.\end{cases}$$
			Fall $G$ additiv geschrieben ist, schreibt man $na$, statt $a^n$.
	\end{enumerate}

\subsection{Definition} Ist $(G, \cdot)$ eine Gruppe, so nennt man $\#G \in \N_0 \cup \{\infty\}$ die Ordnung\index{Ordnung} von $(G,\cdot)$.

\subsection{Beispiel}
	\begin{enumerate}[label=(\alph*)]
		\item
			Für jede Menge $M$ bildet die Menge $S_M := \{f \mid f: M \to M ~ \text{bijektiv}\}$ mit der durch $fg := f \circ g$ $(f,g \in S_M)$ gegebenen Multiplikation eine Gruppe. Man nennt sie die symmetrische Gruppe\index{Gruppe!symmetrische} auf M. Das neutrale Element von $S_M$ ist die Identität auf $M$ und das zu einem $f \in S_M$ inverse Element ist die Umkehrfunktion von $f$, wodurch die Notation $f^\mo$ nicht zweideutig ist.
		
			Für $n \in \N_0$ ist $S_n := S_{\{1,...,n\}}$ eine Gruppe der Ordnung $n! := \prod_{i=1}^n i$ "`$n$ Fakultät\index{Fakultät}"'. Für $n \geq 3$ ist die nicht abelsch, dann die Transpositionen $\tau_{1,2}$ und $\tau_{2,3}$ konvertieren nicht, d.h. $\tau_{1,2}\tau_{2,3} \neq \tau_{2,3}\tau_{1,2}$. In der Tat: $(\tau_{1,2}\tau_{2,3})(1) = \tau_{1,2}(1) = 2$ und $(\tau_{2,3}\tau_{1,2})(1) = \tau_{2,3}(2) = 3$.
		
		\item
			Für jeden Vektorraum $V$ ist die Menge $\Aut(V) := \{f \mid f : V \to V ~\text{linear und bijektiv}\}$\index{Automorphismus!Vektorraum-} mit der Hintereinanderschaltung als Multiplikation eine Gruppe.
			
		\item
			Ist $R$ ein kommutativer Ring (z.\ B. $R = \Z$), so ist $\GL_n(R) := \{A \in R^\ntn \mid A ~\text{invertierbar}\} = \{ A \in R^\ntn \mid \det A \in R^\times \}$\index{General Linear Group} eine Gruppe.
	\end{enumerate}
	
\subsection{Proposition} Sei $G$ eine Gruppe und $a,b \in G$.
\begin{enumerate}[label=(\alph*)]
		\item
			$ab = 1 \iff a = b^\mo \iff b = a^\mo$
		\item
			$\left(a^\mo\right)^\mo = a$
		\item
			$(ab)^\mo = b^\mo a^\mo$
\end{enumerate}

\bew
\begin{enumerate}[label=(\alph*)]
	\item
		Gilt $ab = 1$, so $a \GRPeqn a1 \GRPeqi a(bb^\mo) \GRPeqa (ab) b^\mo = 1b \GRPeqn b^\mo$. Gilt $a = b^\mo$, so $b \GRPeqn 1b \GRPeqi (a^\mo a)b \GRPeqa a^\mo(ab) = a^\mo(b^\mo b) \GRPeqi a^\mo 1 \GRPeqn a^\mo$. Gilt $b = a^\mo$, so $ab = 1$.
		
	\item
		Aus $a a^\mo \GRPeqi 1$ folgt mit (a) $\left( a^\mo \right)^\mo = a$.
		
	\item
		Aus $(ab)(b^\mo a^\mo) \GRPeqa a(b(b^\mo a^\mo)) \GRPeqa a((b b^\mo)a^\mo) \GRPeqi a(1a^\mo) \GRPeqn aa^\mo \GRPeqi 1$ folgt mit (a) $(ab)^\mo = b^\mo a^\mo$. \qed		
\end{enumerate}

\subsection{Definition} Seien $(G, \cdot_G)$ und $(H, \cdot_H)$ Gruppen. Dann heißt $(H, \cdot_H)$ eine Untergruppe\index{Gruppe!Unter-} von $(G, \cdot_G)$, wenn $H \subseteq G$ und $\forall a, b \in H : a \cdot_H b = a \cdot_G b$.

\subsection{Proposition} Sei $(G, \cdot_G$ eine Gruppe und $H$ eine Menge. Dann ist $H$ genau dann Trägermenge\index{Träger} einer Untergruppe von $(G, \cdot_g)$, wenn $H \subseteq G$, $1_S \in H$, $\forall a,b \in H : a \cdot_S b \in H$ und $\forall a \in H : a^ \mo \in H$.

In diesem Fall gibt es genau eine Abbildung $\cdot_H : H \times H \to H$ derart, dass $(H, \cdot_H)$ eine Untergruppe von $(G, \cdot_G)$ ist. Es gilt dann $1_H = 1_G$, $\forall a,b \in H : a \cdot_H b = a \cdot_G b$ und $a^\mo = a^\mo$ (je in $G$ und $H$ gebildet).

\bew Klar oder vgl.\ LA § 2. \qed

\subsection{Bemerkung}
\begin{enumerate}[label=(\alph*)]
	\item
		Ist $(H, \cdot_H)$ Untergruppe von $(G, \cdot_G)$, so schreibt man meist $\cdot$ statt $\cdot_H$. Oft erwähnt man $\cdot_H$ gar nicht mehr und schreibt einfach "`$H$ ist Untergruppe von $G$"' oder $H \leq G$.
		
	\item
		Untergruppen abelscher Gruppen sind abelsch.
\end{enumerate}

\subsection{Beispiel}
\begin{enumerate}[label=(\alph*)]
	\item
		Für $n \in \N_0$ ist $A_n := \{\sigma \in S_n \mid \sgn \sigma = 1\}$ eine Untergruppe von $S_n$, die man alternierende Gruppe \index{Gruppe!alternierende} nennt. \LAref{§ 9.1}
\end{enumerate}

\numberfix{1}{3}{3}
\subsection{Definition} Sei $G$ eine Gruppe. Zu jedem $H \leq G$ definieren wir Äquivalenzrelationen $\sideset{_H}{}{\mathop\sim}$ und $\sim_H$ auf $G$ durch
\begin{align*}
	a \eqsimleft H b & \iff ab^{-1} \in H & (a,b \in G) \\
	a \eqsimright H b & \iff a^{-1}b \in H & (a,b \in G)
\end{align*}

Die Äquivalenzklassen
\begin{align*}
	\eqaccentleft Ha &=\{b \in G \mid a \eqsimleft H b \} = \{b\in G \mid ab^{-1} \in H\} = \{ha \mid h\ in H\} =: Ha \\
	\eqaccentright aH &= \{b \in G \mid a \eqsimright H b \} = \{b \in G \mid a^{-1}b \in H \} = \{ah \mid h \in H\} =: aH
\end{align*}
nennt man Rechts- bzw. Linksnebenklassen\index{Nebenklasse} von $H$ nach $a$ ($a \in G$).

\subsection{Bemerkung}
	Ist $\equiv$ eine Kongruenzrelation auf $G$, so gilt nach \ref{fixed:1.3.3} für $H:=\overline{1}$ die Gleichheit $(\equiv) = (\eqsimleft H) = (\eqsimright H)$.
	
\subsection{Definition} Sei $G$ eine Gruppe. Eine Untergruppe $N$ von $G$ heißt Normalteiler\index{Normalteiler} von $G$, in Zeichen $N \triangleleft G$, wenn $\eqsimleft H {=} \eqsimright H$.

\subsection{Proposition} \label{fixed:1.3.7}
Sei $G$ eine Gruppe und $H\leq G$. Dann sind äquivalent:
\begin{enumerate}[label={\alph*)}]
	\item $H\triangleleft G$
	\item $\sideset{_H}{}{\mathop\sim}=\sim_H$
	\item $\forall a \in G:Ha=aH$
	\item $\forall a\in G: aHa^{-1} :=\{aha^{-1}\:| \: h \in H\}=H$
	\item $\forall a\in G: aHa^{-1} \subseteq H$
	\item $\sideset{_H}{}{\mathop\sim}$ ist eine Kongruenzrelation
	\item  $\sim_H$ ist eine Kongruenzrelation
	\item $H$ ist der Kern eines Gruppenhomomorphismus	
\end{enumerate}		

\proof Übung. \qed
	
\subsection{Notation und Proposition} Sei $G$ eine Gruppe und $N \triangleleft G$. Dann schreiben wir:
\begin{align*}
	(\equiv_N) &:= (\eqsimleft N) = (\eqsimright N) \\
	G/N &:= G/\equiv_N {=} \{Na\:|\: a\in G\} =\{aN\:|\: a\in G\}
\end{align*}
Weiter bezeichnen wir die Kongruenzklasse $\bar{a}=Na=aN$ von $a \in G$ auch als Nebenklasse\index{Nebenklasse} von $N$ nach $a$.

\subsection{Satz} Sei $G$ eine Gruppe. Die Zuordnungen
\begin{align*}
	\equiv & \mapsto \bar{1}\\
	\equiv_N & \mathrel{\reflectbox{$\mapsto$}} N
\end{align*}
vermitteln eine Bijektion zwischen der Menge der Kongruenzrelationen auf $G$ und der Menge der Normalteiler von $G$.

\proof Übung. \qed

\subsection{Definition und Proposition}
Sei $G$ eine Gruppe. Ein Isomorphismus $G \to G$ heißt Automorphismus\index{Automorphismus} von $G$. Bezüglich der Hintereinanderschaltung bilden die Automorphismen von $G$ eine Gruppe, die sogenannte Automorphismengruppe $\Aut(G)$\index{Automorphismus!-engruppe} von $G$. Die Konjugationen\index{Konjugation} $c_a : G \to G, b \mapsto aba^{-1}$ mit $a \in G$ sind offensichtlich Automorphismen von $G$, die sogenannten inneren Automorphismen\index{Automorphismus!innerer} von $G$. Sie bilden den Normalteiler $\Inn(G):= \{c_a \mid a \in G \}$ von $\Aut(G)$. Eine Untergruppe $N \leq G$ ist genau dann ein Normalteiler von $G$, wenn $f(N)=N$ für alle $f \in \Inn(G)$ gilt. Man nennt $N\leq G$ eine charakteristische Untergruppe\index{Untergruppe!charakteristische} von $G$, wenn $f(N)=N$ sogar für alle $f \in \Aut(G)$ gilt.

\proof Zu zeigen:
\begin{enumerate}[label={\alph*)}]
	\item $\forall a \in G: c_a \in \Aut(G)$
	\item $\Inn(G)\triangleleft \Aut(G)$	
\end{enumerate}
\textbf{Zu (a):} Übung.

\textbf{Zu (b):} Sei $a \in G$ und $f \in \Aut(G)$. Zu zeigen: $fc_af^{-1} \in \Inn(G)$. Ist $b \in G$, so $(fc_af^{-1})(b)=f\left(af^{-1}(b)a^{-1}\right) = f(a)bf(a)^{-1}$. Daher $fc_af^{-1}=c_{f(a)} \in \Inn(G)$. \qed
		
\subsection{Beispiel}
\begin{enumerate}[label={\alph*)}]
	\item
		Nicht jeder Normalteiler ist eine charakteristische Untergruppe. Ist zum Beispiel $G \neq \{1\}$ eine Gruppe, so ist $G\times \{1\} \normsub G\times G$, aber $G\times \{1\}$ ist keine charakteristische Untergruppe von G, denn $G \times G \to G \times G, (g,h) \mapsto (h,g)$ ist ein Automorphismus von $G\times G$.
		
	\item
		Sei $n \in \N$, $n \geq 3$. Dann gilt $C_n \normsub D_n$. In der Tat: Sei $A \in D_n$ und $B \in C_n$. Zu zeigen: $ABA^{-1} \in C_n$. Dies ist klar, falls $A \in C_n$, denn $C_n$ ist abelsch. Sei nun $A \in D_n \backslash C_n$. Dann ändert die Spiegelung den Drehsinn und somit $ABA^{-1}=B^{-1} \in C_n$.
		
	\item
		Als Kern des Gruppenhomomorphismus $\sgn : S_n \to \{-1,1\}$ ist $A_n$ ein  Normalteiler von $S_n$.
		
	\item Sei $R$ ein kommutativer Ring als Kern von $\det : \GL_n(R_n) \to R^{\times}$ ist $\SL_n(R) \normsub \GL_n(R)$.
	
	\item Ebenso $\SO_n \normsub \text{O}_n$.
\end{enumerate}

\subsection{Bemerkung} Wird eine Untergruppe einer Gruppe in einer Weise definiert, die offensichtlich nur auf die Gruppenstruktur Bezug nimmt, so ist nach \ref{fixed:1.2.6} klar, dass diese Untergruppe charakteristisch und insbesondere ein Normalteiler ist.
	
\subsection{Definition}	Sei $G$ eine Gruppe. Dann heißt
$$Z(G):=\{a \in G \mid \forall b \in G: ab = ba\} \normsub G$$
das Zentrum von $G$.

\subsection{Bemerkung} Mit \ref{fixed:1.2.6} ist klar, dass $Z(G)$ sogar eine charakteristische Untergruppe der Gruppe $G$ ist. Insbesondere ist $Z(G) \normsub N$. Letzteres folgt auch mit \ref{fixed:1.3.7h}, denn $Z(G)$ ist der Kern des Gruppenhomomorphismus $G\to \Aut(G),~a \mapsto c_a$. (Das Bild von $Z(G)$ ist übrigens $\Inn(G)$.)

\subsection{Homomorphiesatz für Gruppen} \LAref{§ 2.3} Seien $G$ und $H$ Gruppen, $N \normsub G$ und $f:G \to H$ ein Homomorphismus mit $N \subseteq \ker f$. Dann gibt es genau eine Abbildung $\bar{f} : G/N \to H$ mit $\bar{f}\left(\bar{a}^N\right) = f(a)$ für alle $a \in G$. Die Abbildung $\bar{f}$ ist ein Homomorphismus. Weiter gilt:
\begin{align*}
	\bar{f} \text{ injektiv } & \iff N=\ker f\\
	\bar{f} \text{ surjektiv } & \iff H=\im f
\end{align*}	 

\proof Eindeutigkeit von $\bar{f}$ ist klar.

Zur Existenz von $\bar{f}$ (Wohldefiniertheit): Seien $a,b \in G$ mit $\bar{a}^N=\bar{b}^N$, d. h. $a \equiv_N b$. Zu zeigen: $f(a)=f(b)$. Wegen $ab^{-1} \in N \subseteq \ker f$ folgt $f(ab^{-1})=1$, also $f(a)f\left(b^{-1}\right)=f(a)f(b)^{-1}=1$. Es folgt $f(a)=f(b)$.\\

$\bar{f}$ ist ein Homomorphismus: Seien $a,b \in G$. Zu zeigen: $\bar{f}\left(\bar{a}^N \bar{b}^N\right)=\bar{f}\left(\bar{a}^N\right) \bar{f}\left(\bar{b}^N\right)$.
Es gilt $\bar{f}\left(\bar{a}^N \bar{b}^N\right) \overset{\ref{fixed:1.3.2}}{=} \bar{f}\left(\overline{ab}^N\right)=f(ab)=f(a)f(b)=\bar{f}\left(\bar{a}^N\right) \bar{f}\left(\bar{b}^N\right)$.
\begin{align*}
	\bar{f} \text{ injektiv } & \iff \ker \bar{f} =\{1\} \iff \{\bar{a}^N \mid f(a)=1\} = \{1\} \iff \forall a \in \ker f: \bar{a}^N=\bar{1}^N \\
 	& \iff \ker f \subseteq N \iff \ker f = N \\
	\bar{f} \text{ surjektiv } & \iff H=\im \bar{f} \iff H=\im f
\end{align*} \qed

\numberfix{2}{1}{3}
\subsection{Beispiel}
\begin{enumerate}[label=(\alph*)]
	\item
		Für jeden Vektorraum $V$ ist die Menge $\End(V)=\{f \mid f: V \to V ~ \text{linear}\}$\index{Endomorphismus!Vektorraum-} der Endomorphismen von $V$ mit der punktweisen Addition und der Hintereinanderschaltung als Multiplikation ein Ring mit Einheitengruppe $\End(V)^{\times} = \Aut(V)$. \LAref{§ 7.1}
		
	\item
		Ist $R$ ein kommutativer Ring, so ist $R^{n \times n}$ ein  Ring mit $(R^{n \times n})^{\times} = \GL_n(R)$.
\end{enumerate}

\subsection{Definition} Seien $(A,+_A,\cdot_A)$ und $(B,+_B,\cdot_B)$ Ringe. Dann heißt $(A,+_A,\cdot_A)$ ein \emph{Unterring}\index{Unterring}\index{Ring!Unter-}  von $(B,+_B,\cdot_B)$, wenn $A \subseteq B$, $1_B \in A$, $\forall a,b \in A: a +_A b = a +_B b$, $\forall a,b \in A : a \cdot_A b = a \cdot_B b$.
 
\subsection{Proposition}
 Sei $(B,+,\cdot)$ ein Ring und $A$ eine Menge. Genau dann ist $A$ Trägermenge eines Unterrings von $(B,+,\cdot)$, wenn $\{0,1\} \subseteq A \subseteq B$, $\forall a,b \in A: a+b \in A, a \cdot b \in A$.
 
\subsection{Beispiel}
\begin{enumerate}[label=(\alph*)]
	\item
		Sei $R$ ein kommutativer Ring und  $n \in \mathbb{N}_0$. Dann sind $\Odm_R^{n \times n} = \{ A \in R^{n \times n} \mid A \text{ obere Dreiecksmatrix}\}$, $\Udm_R^{n \times n} = \{ A \in R^{n \times n} \mid A \text{ untere Dreiecksmatrix}\}$ und $\Udm_R^{n \times n} \cap \Odm_R^{n \times n} = \{ A \in R^{n \times n} \mid A \text{ Diagonalmatrix}\}$ Unterringe von $R^{n \times n}$ mit Einheitengruppen $(\Odm_R^{n \times n})^{\times} = \Odm_n(R)$, $(\Udm_R^{n \times n})^{\times} = \Udm_n(R)$ und $\left(\Udm_R^{n \times n} \cap \Odm_R^{n \times n}\right)^{\times} = \Odm_n(R) \cap \Udm_n(R)$.
		
	\item
		$\{ 0 \}$ ist kein Unterring von $\mathbb{Z}$, denn $1 \notin \{0\}$.
\end{enumerate}

\subsection{Definition} Seien $A$ und $B$ Ringe. Dann heißt $f: A \to B$ ein \emph{(Ring-)Homomorphismus}\index{Homomorphismus!Ring-} von $A$ nach $B$, wenn
\begin{align*}
	&f \text{ ein Gruppenhomomorphismus von } A \text{ nach } B \text{ ist,}\\
	&f(1)=1 \text{ und}\\
	&\forall a,b \in A: f(ab) = f(a)f(b) \text{ gilt.}
\end{align*}

Ein Ringhomomorphismus heißt
\begin{center}
	\begin{tabular}{llll}
		(Ring-) & (Einbettung oder) Mono-  & / Epi- & / Isomorphismus\\
		wenn $f$ & injektiv & / surjektiv & / bijektiv ist, \\
		in Zeichen & $f: A \MAPlongmono B$ & / $f: A \MAPlongepi B$ & / $f: A \MAPlongiso B$
	\end{tabular}
	\index{Einbettung!Ring-}
	\index{Monomorphismus!Ring-}
	\index{Epimorphismus!Ring-}
	\index{Isomorphismus!Ring-}
\end{center}

\subsection{Bemerkung} Ist $f: A \to B$ ein Ringhomomorphismus, so ist $\im f$ ein Unterring von $B$, jedoch $\ker f$ in aller Regel kein Unterring von $A$. (Denn $1 \in \ker f \iff f(1)=0$ in $B$ $\iff 1=0$ in $B$. \lightning)
 
\subsection{Bemerkung} Analog zu 1.2.7 und 1.2.8 führt man das \emph{direkte Produkt}\index{Direktes Produkt!von Ringen}\index{Produkt!direktes!von Ringen} von Ringen durch punktweise Addition und Multiplikation ein.
 
\subsection{Definition und Proposition } \ALref{§ \ref{fixed:1.3}}, \LAref{§ 3.3} Sei $R$ ein Ring. Eine \emph{Kongruenzrelation}\index{Kongruenzrelation} auf $R$ ist eine Kongruenzrelation $\equiv$  auf der additiven Gruppe von $R$ \ALref{\ref{fixed:1.3.1}}, für die zusätzlich gilt:
\begin{equation*}
	\forall a,a',b,b'\in A:((a\equiv a' ~\&~ b\equiv b')\implies ab\equiv a'b')
\end{equation*}
Ist $\equiv$ ein Kongruenzrelation auf $R$, so wird $R/\equiv$ vermöge $\overline a + \overline b = \overline{a+b}$ und $\overline a \overline b = \overline{ab}$ $(a,b\in A)$ zu einem Ring (
	\emph{"`Quotientenring"'}\index{Quotientenring}\index{Ring!Quotienten-}
	\emph{"`Faktorring"'}\index{Faktorring},\index{Ring!Faktor-}
	\emph{"`Restklassenring"'}\index{Restklassenring}\index{Ring!Restklassen-}
).
 
\subsection{Definition} Sei $R$ ein Ring. Eine Untergruppe $I$ der additiven Gruppe von $R$ heißt (beidseitiges) \emph{Ideal}\index{Ideal} von $R$, wenn:
\begin{equation*}
	\forall a\in R ~\forall b \in I : ab, ba \in I
\end{equation*}
 
\subsection{Satz} \ALref{\ref{fixed:1.3.9}} \LAref{§ 3.3} Sei $R$ ein Ring. Die Zuordnungen
\begin{align*}
	\equiv&\mapsto\overline0\\
	\equiv_I&\mapsfrom I
\end{align*}
vermitteln eine Bijektion zwischen der Menge der Kongruenzrelationen auf $R$ und der Menge der Ideale von $R$.
 
\proof Wenn wir zeigen, dass beide Abbildungen wohldefiniert sind, dann folgt mit \ref{fixed:1.3.9}, dass sie auch invers zueinander sind. Also zu zeigen:
\begin{enumerate}[label=(\alph*)]
	\item $\equiv$ ist Kongruenzrelation auf $R \implies \overline 0$ ist Ideal von $R$
	\item $I$ ist Ideal von $R \implies \equiv_I$ ist Kongruenzrelation auf $R$
\end{enumerate}

{\bf Zu (a).} Sei $\equiv$ eine Kongruenzrelation auf $R$. Aus \ref{fixed:1.3.9} wissen wir schon, dass $\overline 0$ eine Untergruppe von $R$ ist. Noch zu zeigen: $\forall a\in A:\forall b\in\overline0:ab\in\overline0$. Sei also $a\in R$ und $b\in\overline 0$. Dann $ab \overset{b \equiv 0} \equiv a0 \overset{2.1.2(e)}\equiv0$, also $ab\in\overline0$ und $ba\equiv0a\equiv0$, also $ba\in\overline0$.
 
{\bf Zu (b).} Sei $I$ eine Ideal von $R$. Aus \ref{fixed:1.3.9} wissen wir schon, dass $\equiv_I$ eine Kongruenzrelation der additiven Gruppe von $R$ ist. Noch zu zeigen: $\forall a,a',b,b'\in A:((a\equiv a' ~\&~ b\equiv b')\implies ab\equiv a'b')$. Seien also $a,a',b,b'\in R$ mit $a\equiv_Ia'$ und $b\equiv_Ib'$. Dann $ab-a'b'=a\underbrace{(b-b')}_{\in I} + b'\underbrace{(a-a')}_{\in I}\in I$, also $ab\equiv_Ia'b'$. \qed

\subsection{Notation \& Sprechweise} Sei $I$ ein Ideal des Ringes $R$. Schreibe $R/I := R/\equiv_I ~:= \{ a+I \mid a \in R \}$. Man bezeichnet die Kongruenzklasse $\overline{a}^I = a+I$ von $a\in R$ auch als \emph{Restklasse}\index{Restklasse} von $a$ modulo $I$.
 
\subsection{Bemerkung}
\begin{enumerate}[label=(\alph*)]
	\item
		Sei $I$ ein Ideal des Ringes $R$. Dann ist die Abbildung $R \to R/I, a \mapsto \overline{a}^I$ nach Definition \ref{fixed:2.1.11} ein Ringhomomorphismus, genannt \emph{kanonischer Epimorphismus}\index{Epimorphismus!Ring-!kanonischer}.
		
	\item
		Sei $f: A \to B$ ein Ringhomomorphismus. Dann ist $\ker f$ ein Ideal von $A$, aber $\im f$ im Allgemeinen kein Ideal von $B$. (Betrachte zum Beispiel $\Z \MAPmono \Q, a \mapsto a$.)
\end{enumerate}

\subsection{Homomorphiesatz für Ringe\index{Homomorphiesatz!für Ringe}}
	Seien $A, B$ Ringe, $I$ ein Ideal von $A$ und $\varphi: A \to B$ ein Homomorphismus mit $I \subseteq ker \varphi$. Dann gibt es genau eine Abbildung $\overline{\varphi}: A/I \to B$ mit $\overline{\varphi}(\overline{a}^I)=\varphi(a)$ für alle $a \in A$. Diese Abbildung $\overline{\varphi}$ ist ein Homomorphismus. Weiter gilt $\overline{\varphi}$ injektiv $\iff I = \ker \varphi$ und $\overline{\varphi}$ surjektiv $\iff B = \im \varphi$.
 
\proof Mit \ref{fixed:1.3.15} ist nur noch $\overline{\varphi}(1)=1$ und $\overline{\varphi}(\overline{a}^I \overline{b}^I) = \overline{\varphi}(\overline{a}^I) \overline{\varphi}(\overline{b}^I)$ f.a. $a,b\in A$ zz zeigen.

Dies ist klar:
\begin{align*}
  &\overline{\varphi}(1)=\overline{\varphi}(\overline{1}^I)=\varphi(1)=1 \quad\text{und} \\
  &\overline{\varphi}(\overline{a}^I \overline{b}^I)=\overline{\varphi}(\overline{ab}^I)=\varphi(ab)=\varphi(a)\varphi(b)=
  \overline{\varphi}(\overline{a}^I)\overline{\varphi}(\overline{b}^I) \quad\text{für alle}~ a,b\in A. 
\end{align*}
\qed
 
\subsection{Isomorphiesatz für Ringe\index{Isomorphiesatz! für Ringe}} Seien $A, B$ Ringe und $\varphi: A \to B$ ein Homomorphismus. Dann ist $\ker \varphi$ ein Ideal von $A$ und $\overline{\varphi}: A/\ker\varphi \to \im\varphi$ mit $\overline{\varphi}(\overline{a}^{\ker \varphi})=\varphi(a)$ für $a \in A$ ein Isomorphismus. Insbesondere $A/\ker\varphi \cong \im \varphi$.
 
\proof Direkt aus \ref{fixed:2.1.16}. \qed
 
\section{Polynomringe \small\LAref{§ 3.2}}

\subsection{Notation} Sei $R$ ein kommutativer Ring, $n \in \N_0$, $a=(a_1,...,a_n)\in R^n$ und $\alpha=(\alpha_1,...,\alpha_n) \in \N_0^n$. Schreibe dann $|\alpha|=\alpha_1+...+\alpha_n$ und $a^\alpha:=a_1^{\alpha_1}+...+a_n^{\alpha_n}$.

\subsection{Definition \& Satz} Sei $A$ ein Unterring des kommutativen Ringes $B$.
\begin{itemize}
	\item[(a)]
		Sei $n \in \N_0$ und $b=(b_1,...,b_n)\in B^n$.
		\begin{equation*}
			 A[b] := A[b_1,...,b_n]
			 :=\left\{\sum_{\alpha \in \N_0^n, \atop |\alpha|<d} a_\alpha b^\alpha \mid d \in \N_0, a_\alpha\in A\right\}
		\end{equation*}
		ist der kleinste Unterring $C$ von $B$ mit $A \cup \{b_1,...,b_n\} \subseteq C$.
   
	\item[(b)]
		Sei $E \subseteq B$. $A[E] = \bigcup\{A[b] \mid n\in \N_0, b\in B^n\}$ ist der kleinste Unterring $C$ von $B$ mit $A \cup E \subseteq C$.
\end{itemize}

\proof Dass die angegeben Mengen jeweils in jedem solchen Unterring $C$ enthalten sind, ist klar. Zu zeigen ist dann nur noch, dass sie jeweils einen Unterring bilden. Dies ist einfach und wir zeigen exemplarisch nur, dass $A[b]$ aus (a) unter Multiplikation abgeschlossen ist. Seien also $d, d' \in \N_0, a_\alpha \in A$ für alle $\alpha \in \N_0^n$ mit $|\alpha| \leq d$ und $a_\alpha' \in A$ für alle $\alpha \in \N_0^n$ mit $|\alpha| \leq d'$. Dann
\begin{equation*}
	\left(\sum_{|\alpha| \leq d} a_\alpha b^\alpha\right)\left(\sum_{|\alpha| \leq d'} a_\alpha' b^\alpha\right)
	= \sum_{|\gamma| \leq d+d'} \left(\sum_{\alpha+\beta=\gamma} a_\alpha a'^\beta\right) b^\gamma~\in A[b],
\end{equation*}
wobei man $a_\alpha:=0$ für $d<|\alpha| \leq d+d'$ und $a_\alpha':=0$ für $d'<|\alpha| \leq d+d'$ setzt. \qed

\subsection{Definition} Sei $A$ ein Unterring des kommutativen Ringes $B$.
\begin{enumerate}[label=(\alph*)]
	\item
		Sei $n \in \N_0$ und $b=(b_1,...,b_n) \in B^n$. Es heißen $b_1,...,b_n$ \emph{algebraisch unabhängig}\index{unabhängig!algebraisch}\index{algebraisch!unabhängig} über $A$ (in $B$), wenn für alle $d \in \N_0$ und alle $a_\alpha \in A (\alpha \in \N_0^n, |\alpha|\leq d)$ gilt:
		\begin{equation*}
			\sum_{\alpha \in \N_0^n, \atop |\alpha|<d} a_\alpha b^\alpha = 0 \implies \forall\alpha\in\N_0:(|\alpha|\leq d \implies a_\alpha = 0)
		\end{equation*}
		
		Es heißt $B$ \emph{Polynomring}\index{Polynomring}\index{Ring!Polynom-} über $A$ in $b_1,...,b_n$, wenn $B=A[b_1,...,b_n]$ und $b_1,...,b_n$ algebraisch unabhängig über $A$ sind.
		
	\item
		Sei $E \subseteq B$. Es heißt $E$ \emph{algebraisch unabhängig}\index{unabhängig!algebraisch}   über $A$ (in $B$), wenn für alle $n\in\N_0$ alle paarweise verschiedenen Elemente $b_1,...,b_n \in E$ algebraisch unabhängig über $A$ sind.
		
		Es heißt $B$ \emph{Polynomring}\index{Ring!Polynom-} über $A$ in $E$, wenn $B=A[E]$ und $E$ algebraisch unabhängig über $A$ ist.
\end{enumerate}

\subsection{Beispiel}
\begin{enumerate}[label=(\alph*)]
	\item
		Jeder kommutative Ring $A$ ist ein Polynomring über sich selbst in $\emptyset$.
	\item
		Der Nullring $\{0\}$ ist ein Polynomring über sich selbst in $0$.
\end{enumerate}

\subsection{Satz} Sei $A$ ein kommutativer Ring mit $0\neq1$. Sei $E$ eine Menge mit $A \cap E = \emptyset$. Dann gibt es einen Polynomring über $A$ in $E$.
 
\proof Bezeichne $\N_0^{(E)}$ die Menge aller $\alpha: E \to \N_0$ mit endlichem Träger $supp(\alpha) = \{ e\in E \mid \alpha(e) \neq 0 \}$. Mache die abelsche Gruppe $A^{\N_0^{(E)}}$ zu einem kommutativen Ring mit der "`Faltung"' $\ast$ als Multiplikation, welche gegeben ist durch 
\begin{align*}
	  (f \ast g)(\gamma):= \sum_{\alpha,\beta\in\N_0^{(E)}, \atop \alpha+\beta=\gamma} f(\alpha)g(\beta) && \left(f,g\in A^{\N_0^{(E)}}, \gamma \in \N_0^{(E)}\right)
\end{align*}
(Es handelt sich um eine endliche Summe, da $supp(\gamma)$ endlich. Man sieht sofort $f \ast g = g \ast f$, $f \ast (g+h) = f \ast g + f \ast h$ und $1 \ast f = f$ für
\begin{align*}
	1 : \N_0^{(E)} & \to A \\
	\alpha &\mapsto \begin{cases} 1, & \alpha = 0 \\ 0, & sonst \end{cases}
\end{align*}
und rechnet
\begin{align*}
	((f \ast g) \ast h)(\gamma) &= \sum_{\alpha+\beta=\gamma} (f \ast g)(\alpha)h(\beta)
 	= \sum_{\alpha+\beta=\gamma} \left( \sum_{\delta+\varepsilon=\alpha} f(\delta)g(\varepsilon) \right) h(\beta) \\
 	&= \sum_{\delta+\varepsilon+\beta=\gamma} f(\delta)g(\varepsilon)h(\beta)
 	= ... = (f \ast (g \ast h))(\gamma)
\end{align*}
für alle $f,g,h\in A^{\N_0^{(E)}}, \gamma \in\N_0^{(E)}$. \qed\footnote{Korrektur: Ist der Beweis vollständig? Hier fehlen noch 2.2.6, 2.2.7 und 2.2.8 aus dieser Vorlesung.}

\numberfix{2}{3}{5}
\subsection{Satz} \label{fixed:2.3.6} Sei $A$ ein kommutativer Ring und $S \subseteq A$ eine multiplikative Menge, die keine Nullteiler von $A$ enthält. Dann gibt es einen kommutativen Oberring $B$ von $A$ mit $S \subseteq B^\times$ und $B = S^\mo A$.

\proof Durch $(a,s) \sim (b,t) :\iff at = bs$ ($a,b \in A, s,t \in S$) wird eine Äquivalenzrelation $\sim$ auf $A \times S$ definiert. [Reflexiv und symmetrisch ist klar, transitiv: Seien $a,b,c \in A$ und $s,t,u \in S$ mit $(a,s) \sim (b,t) \sim (c,u)$. Dann $at=bs$ und $bu = ct$, also $atu = bsu = bus = cts$, das heißt $t(au-cs) = 0$ und daher $au = cs$, da $t \in S$ kein Nullteiler ist.] Der Leser zeigt als Übung, dass $+$ und $\cdot$ durch
\begin{align*}
	\widetilde{(a,s)} + \widetilde{(b,t)} &:= \widetilde{(at+bs, st)} \quad \text{und}\\
	\widetilde{(a,s)} \cdot \widetilde{(b,t)} &:= \widetilde{(ab,st)}
\end{align*}
wohldefiniert ist und $(A \times S)/{\sim}$ zu einem kommutativen Ring mit $0 = \widetilde{(0,1)}$, $1 = \widetilde{(1,1)}$ machen.

Wegen $A \cong \tilde A := \{\widetilde{(a,1)} \mid a \in A\} \subseteq (A \times S)/\sim$ reicht es zu zeigen, dass $\tilde S := \{\widetilde{(s,1)} \mid s \in S\} \subseteq \left((A \times S)/{\sim})\right)^\times$ und $(A \times S)/{\sim} = \tilde S^\mo \tilde A$. Sei hierzu $a \in A$, $s \in S$. Dann $\widetilde{(s,1)} \widetilde{(1,s)} = \widetilde{(s,s)} = \widetilde{(1,1)} = 1$, also $\widetilde{(s,1)}^\mo = \widetilde{(1,s)}$ und $\widetilde{(a,s)} = \widetilde{(s,1)}^\mo \widetilde{(a,1)} \in \tilde S^\mo \tilde A$. \qed

\subsection{Satz} \label{fixed:2.3.7} Sei $A$ ein Unterring des kommutativen Ringes $B$, $S \subseteq A \cap B^\times$ multiplikativ und $B = S^\mo A$. Sei $C$ ein weiterer Ring und $\varphi : A \to C$ ein Homomorphismus. Genau dann gibt es einen Homomorphismus $\psi : S^\mo A \to C$ mit $\varphi = \psi|_A$, wenn $\varphi(S) \subseteq C^\times$. In diesem Fall ist $\psi$ eindeutig bestimmt, denn es gilt $\psi\left(\frac{a}{s}\right) = \frac{\psi(a)}{\psi(s)}$ für $a \in A, s \in S$.

\proof Übung. \qed

\subsection{Satz} \label{fixed:2.3.8} Sei $A$ ein Unterring des kommutativen Ringes $B$, $S \subseteq A \cap B^\times$ multiplikativ und $B = S^\mo A$. Dasselbe gelte mit $C$ statt $B$. Dann gibt es genau einen Isomorphismus $\psi: B \to C$ mit $\psi|_A = \id_A$.

\proof Wende \ref{fixed:2.3.7} mit $\varphi: A \to C, a \mapsto a$ an, um zu sehen, dass $\id_A$ eine eindeutige Fotsetzung zu einem Homomorphismus $\psi : B \to C$ hat. Zu zeigen ist nur noch, dass $\psi$ ein Isomorphismus ist. Mit \ref{fixed:2.3.7} bekommt man aber auch einen Homomorphismus $\varphi: C \to B$ mit $\varphi|_A = \id_A$. Nun ist $\varphi \circ \psi: C \to C$ ein Homomorphismus mit $(\varphi \circ \psi)|_A = \id_A$ und daher $\varphi \circ \psi = \id_C$ nach \ref{fixed:2.3.7}. Ebenso $\psi \circ \varphi = \id_B$. Daher sind $\varphi$ und $\psi$ bijektiv. \qed

\subsection{Definition} \label{fixed:2.3.9} Sei $A$ ein kommutativer Ring und $S \subseteq A$ eine multiplikative Menge, die keine Nullteiler von $A$ enthält. Den (nach \ref{fixed:2.3.6} existierenden und nach \ref{fixed:2.3.8} im Wesentlichen eindeutigen) Oberring $B$ von $A$ mit $S \subseteq B^\times$ und $B = S^\mo A$ nennt man Ring der Brüche\index{Ring!der Brüche} mit Zählern\index{Zähler} aus $A$ und Nennern\index{Nenner} aus $S$ (oder Lokalisierung\index{Lokalisierung|\see{Ring!der Brüche}} von $A$ nach $S$).

Ist speziell $S$ die Menge aller Nichtnullteiler von $A$ (vgl.~\ref{fixed:2.3.2.c}), so nennt man $Q(A) = S^\mo A$ den totalen Quotientenring\index{Quotientenring!totaler} von $A$. Offenbar gilt: $Q(A)$ ist Körper $\iff$ $A$ ist Integritätsring. Ist $A$ ein Integritätsring, so nennt man den Körper $\qf(A) := Q(A) = (A\setminus\{0\})^\mo A$ daher auch den Quotientenkörper von $A$.

\subsection{Bemerkung} Es folgt nun, dass Integritätsringe genau die Unterringe von Körpern sind.

\subsection{Definition und Satz} (Körperadjunktion, vgl.~Ringadjunktion \ref{fixed:2.2.2})
\begin{enumerate}[label=(\alph*)]
	\item
		Ist $K$ ein Unterring eines Körpers $L$ und $K$ ein Körper, so nennt man
		\begin{itemize}[label=$-$]
			\item
				$K$ einen Unterkörper\index{Körper!Unterkörper} von $L$,
			\item
				$L$ einen Oberkörper\index{Körper!Oberkörper} von $K$ und
			\item
				$L|K$ ("`über"') eine Körpererweiterung\index{Körpererweiterung}.
		\end{itemize}
	
	\item
		Sei $L|K$ eine Körpererweiterung. Sind $b_1,...,b_n \in L$, so ist $K(b1,...,b_n) := (K[b_1,...,b_n]\setminus\{0\})^\mo K[b_1,...,b_n] = \qf(K[b_1,...,b_n]) \subseteq L$ der kleinste Unterkörper\index{Körper!Unterkörper!kleinster} $F$ von $L$ mit $K \cup \{b_1,...,b_n\} \subseteq F$.
	
		Ist $E \subseteq L$, so ist $K(E) := (K[E]\setminus\{0\})^\mo K[E] = \qf(K[E]) \subseteq L$ der kleinste Unterkörper $F$ von $L$ mit $K \cup E \subseteq F$.
\end{enumerate}

\proof Trivial. \qed

\subsection{Definition} (vgl.~\ref{fixed:2.3.3}) Sei $L|K$ eine Körpererweiterung.
\begin{enumerate}[label=(\alph*)]
	\item
		Sei $n \in \N_0$ und $b_1,...,b_n \in L$. Es heißt $L$ ein Körper der rationalen Funktionen\index{Körper!der rationalen Funktionen} über $K$ in $b_1,...,b_n$, wenn $L = K[b_1,...,b_n]$ und $b_1,...,b_n$ algebraisch unabhängig über $K$1 sind.
		
	\item
		Sei $E \subseteq L$. Es heißt $L$ ein Körper von rationalen Funktionen\index{Körper!von rationalen Funktionen} über $K$ in $E$, wenn $L = K[E]$ und $E$ algebraisch unabhängig über $K$ ist.\footnote{An Korrektor: War mir hier bzgl.~der Klammern und der Namen (Index!) nicht ganz sicher.}
\end{enumerate}

\subsection{Proposition} \label{fixed:2.3.13} (vgl.~\ref{fixed:2.2.6}) Sei $L|K$ eine Körpererweiterung und $E \subseteq L$ mit $L = K[E]$. Sei $R$ ein Ring und seiden $\varphi, \psi: L \to R$ Homomorphismen mit $\varphi|_{K \cup E} = \psi|_{K \cup E}$. Dann $\varphi = \psi$.

\proof $F := \{a \in L | \varphi(a) = \psi(a)\}$ ist ein Unterkörper von $L$, der $K \cup E$ enthält. Also $F = L$. \qed

\subsection{Definition und Proposition} Seien $K$ und $F$ Körper.
\begin{enumerate}[label=(\alph*)]
	\item
		$K$ besitzt nur die trivialen Ideale $K$ und $\{0\}$.
	\item
		Ist $\varphi: K \to F$ ein (Ring-)Homomorphismus, so nennt man $\varphi$ auch einen Körperhomomorphismus\index{Homomorphismus!Körper-}\index{Homomorphismus!Körper-}. In diesem Fall gilt: Da $\varphi(1) = 1 \neq 0$ in $F$, liegt $1$ nicht im Ideal $\ker \varphi$ von $K$, womit $\ker \varphi = \{0\}$ nach (a). Es ist daher $\varphi : K \MAPmono F$ eine Einbettung und $\varphi : K \MAPiso \im \varphi$ ein Isomorphismus. Insbesondere ist das Bild von $\varphi$ nicht nur ein Unterring, sondern sogar ein Unterkörper von $F$. Beachte auch, dass gelten muss $\varphi\left(-\frac{1}{a}\right) = \frac{1}{\varphi(a)}$ für alle $a \in K^\times$.\footnote{An Korrektor: Gehört da wirklich ein Minus hin?}
\end{enumerate}

\subsection{Satz} (vgl.~\ref{fixed:2.2.9}) Seien $K(E)$ und $K(F)$ Körper von rationalen Funktionen über $K$ in $E$ bzw.~$F$. Sei $f: E \to F$ eine Bijektion. Dann gibt es genau einen Isomorphismus $\psi: K(E) \to K(F)$ mit $\psi|_K = \id_K$ und $\psi|_E = f$.

\proof Zur Existenz: Nach \ref{fixed:2.2.9} gibt es einen Isomorphismus $\varphi: K[E] \to K[F]$ mit $\varphi|_K = \id_K$ und $\varphi_E = f$. Da $\varphi$ injektiv ist, gilt $\varphi(K[E]\setminus\{0\}) \subseteq K[F]\setminus\{0\} \subseteq K(F)^\times$ und \ref{fixed:2.3.7} liefert einen Homomorphismus $\psi : K(E) \to K(F)$ mit $\psi|_{K[E]} = \varphi$. Da $\psi$ ein Körperhomomorphismus ist, ist $\psi$ injektiv und $\psi$ ist ein Unterkörper von $K(F)$.\footnote{An Korrektor: Macht so keinen Sinn.} Es gilt aber $K \cup F \subseteq \im \varphi \subseteq \im \psi$, weswegen $\psi$ surjektiv ist.

Die Eindeutigkeit folgt aus \ref{fixed:2.3.13}.

\subsection{Notation und Sprechweise} (vgl.~\ref{fixed:2.2.10}) Sei $K$ ein Körper. Schreibt man $K(X_1,...X_n)$, so meint man dabei den (nach \ref{fixed:2.3.15} im Wesentlichen eindeutig bestimmen und nach \ref{fixed:2.3.5} und \ref{fixed:2.3.9} existierenden) Körper der rationalen Funktionen\index{Körper!der rationalen Funktionen!in Unbestimmten} in paarweise verschiedenen "`unbestimmten"' $X_1,...,X_n$.\footnote{An Korrektor: Index für "`Körper der rationalen Funktionen"' anpassen.}

\subsection{Definition und Proposition} Sei $A$ ein kommutativer Ring und $S \subseteq A$ eine multiplikative Menge. Wenn $S$ Nullteiler enthält (das heißt, wenn es $s \in S$ und $a \in A$ gibt mit $sa=0$), dann können wir keinen Oberring $S^\mo A$ wie in \ref{fixed:2.3.6} konstruieren (siehe \ref{fixed:2.3.5}). In diesem Fall (und allgemein) setzten wir $I_S := \{a \in A \mid \exists s \in S : sa=0\}$. Es ist $I_S$ ein Ideal von $A$, das $S$ multiplikativ ist. Es ist dann $\bar S := \{\bar s \mid s \in S\} \subseteq \bar A := A/I_S$ multiplikativ und ohne Nullteiler. Man nennt dann den Oberring $\bar S^\mo \bar A$ von $\bar A = \bar{A/I_S}$ die Lokalisierung\index{Lokalisierung} von $A$ nach $S$, in Zeichen $A_S := \bar S^\mo \bar A$. Man hat einen Homomorphismus\footnote{An Korrektor: Wie sieht der aus? Habe ich mir nicht aufgeschrieben.} $\iota_S(S) \subseteq A_S^\times$ und $\ker \iota_S = I_S$. Oft schreibt man schlampig wieder $S^\mo A$ und $\frac{a}{s}$ ($a \in A, s \in S$) statt $\bar S^\mo \bar A$ und $\frac{\bar a}{\bar s}$ ($a \in A, s \in S$).

\subsection{Satz} Sei $A$ ein kommutativer Ring und $S \subseteq A$ multiplikativ. Sei $B$ ein weiterer kommutativer Ring und $\varphi: A \to B$ ein Homomorphismus mit $\varphi(S) \subseteq B^\times$. Dann gibt es genau einen Homomorphismus $\psi: A_S \to B$ mit $\varphi = \psi \circ \iota_S$.

\proof Übung. \qed

\section{Primideale und maximale Ideale}

\subsection{Wiederholung} \label{fixed:2.4.1} Sei $R$ ein kommutativer Ring. Ist $E \subseteq R$, so ist $(E) := \{\sum_{i=1}^n a_i b_i \mid n \in \N, a_i \in R, b_i \in E\}$ das kleinste Ideal von $R$, welches $E$ enthält und man nennt es das von $E$ (in $R$) erzeugte Ideal\index{Ideal!erzeugtes} \LAref{3.3.9, 3.3.10}. Für $b_1,...,b_n \in R$ schreibt man auch $(b_1,...,b_n) := ({b_1,...,b_n}) = \{\sum_{i=1}^n a_i b_i \mid a_i \in R\}$. Ideale der Form $(b)$ mit $b \in R$ nennt man auch Hauptideale\index{Hauptideal} \LAref{3.3.11}. Es heißt $R$ ein Hauptidealring\index{Hauptidealring}, wenn $R$ ein Integritätsring ist, in dem jedes Ideal ein Hauptideal ist. $\Z$ und $K[X]$ ($K$ ein Körper, $X$ eine Unbekannte) sind Hauptidealringe \LAref{3.3.13, 10.2.2} oder \cite[§ 2.2, § 2.4]{Bosch2004}.

Ist $p \in R$, so heißt $p$ irreduzibel\index{irreduzibel} (in $R$), wenn
$$p \not\in R^\times \quad \& \quad \forall a,b \in R : (p = ab \Rightarrow (a \in R^\times \text{ oder } b \in R^\times))$$
und prim\index{prim} (in $R$), wenn
$$p \not\in R^\times \quad \& \quad \forall a,b \in R: (p|ab \Rightarrow (p|a \text{ oder } p|b)).$$

In einem Integritätsring ist jedes Primelement $\neq 0$ irreduzibel. Die Äquivalenzrelation $\eqhat$ auf $R$ ist definiert durch $a \eqhat b :\iff (a|b \text{ \& } b|a) \iff (a) = (b)$ ($a,b \in R$).

Setze $\hat a := \overset{\eqhat}{a}$ für $a \in R$. Fixiere $\P_R \subseteq R$ mit $\P_R \to \{a \in R \mid a \text{ prim}, a \neq 0\}/\eqhat, p \to \hat p$ bijektiv. (Z.~B.~$\P_\Z = \P = \{2,3,5,7,11,13,...\}$ für $R = \Z$.) Bezeichne $\N_0^{(\P_R)}$ die Menge der Funktionen $\alpha: \P_R \to \N_0$ mit endlichem Träger\index{Träger}\index{supp@\textit{supp}} $\supp(\alpha) := \{p \in \P_R \mid \alpha(p) \neq 0\}$.

Für jedes $\alpha \in \N_0^{\P_R}$ setze $\P_R^\alpha := \prod_{p \in \supp(\alpha)} p^{\alpha(p)}$. Man nennt $(c,\alpha) \in R \times \N_0^{(\P_R)}$ eine Primfaktorzerlegung\index{Primfaktorzerlegung} von $a \in R$, wenn $a = c \P_R^\alpha$. In Integritätsringen sind Primfaktorzerlegungen eindeutig. Es heißt $R$ ein faktorieller Ring\index{Ring!faktorieller}, wenn er ein Integritätsring ist, in dem jedes $a \in R\setminus\{0\}$ eine Primfaktorzerlegung besitzt. Jeder Hauptidealring ist faktoriell. In einem faktoriellen Ring ist jedes irreduzible Element prim. \cite[§ 2.4]{Bosch2004}

\subsection{Definition} Sei $R$ ein kommutativer Ring. Ein Ideal $\p$ von $R$ heißt Primideal\index{Primideal} von $R$, wenn
$$1 \not\in \p \quad \& \quad \forall a,b \in R : (ab \in \p \Rightarrow (a \in \p \text{ oder } b \in \p)).$$
Ein Ideal $I$ von $R$ heißt echt\index{Ideal!echtes}, wenn $1 \not\in I$ (oder äquivalent $I \neq R$). Ein Ideal $\m$ von $R$ heißt maximales Ideal\index{Ideal!maximales} von $R$, wenn $\m$ ein maximales Element der durch Inklusion halbgeordneten Menge aller echten Ideale von $R$ ist.

\subsection{Bemerkung} \label{fixed:2.4.3} Sei $R$ ein kommutativer Ring. Die in \ref{fixed:2.4.1} wiederholte Definition eines Primelements $p \in R$ kann man offensichtlich wie folgt lesen:
$$1 \not\in (p) \quad \& \quad \forall a,b \in R : (ab \in (p) \Rightarrow (a \in (p) \text{ oder } b \in (p))).$$
Es folgt für $p \in R$: $p$ Primelement $\iff$ $(p)$ ist Primideal

\subsection{Satz} Sei $I$ ein Ideal des kommutativen Ringes $R$. Dann gilt
\begin{enumerate}[label=(\alph*)]
	\item
		$I$ Primideal $\iff$ $R/I$ Integritätsring \quad und
	\item
		$I$ maximales Ideal $\iff$ $R/I$ Körper
\end{enumerate}

\proof Übung. \qed

\subsection{Korrolar} Jedes maximale Ideal eines kommutativen Rings ist ein Primideal.

\proof Jeder Körper ist ein Integritätsring. \qed

\subsection{Korrolar} Seien $A, B$ kommutative Ringe und $\varphi: A \to B$ ein Homomorphismus. Sei $\q$ ein Primideal von $B$. Dann ist $\p := \varphi^\mo(\q)$ ein Primideal von $A$.

\proof $\psi: A \to B/\q, a \mapsto \bar{\varphi(a)}^\q$ ist Hintereinanderschaltung der Homomorphismen $A \stackrel{\varphi}{\longrightarrow} B \stackrel{b \to \bar b^\q}{\longrightarrow} B/\q$ und daher ein Homomorphismus. Nach Isomorphiesatz \ref{fixed:2.1.17} ist $A/\ker\psi \eqtilde \im \psi$. Es ist $\psi$ ein Unterring des Integritätsrings $B/\q$ und daher auch ein Integritätsring. Somit ist auch $A/\ker\psi$ ein Integritätsring, das heißt $\ker\psi$ ein Primideal von $A$. Es gilt $\ker\psi = \{a \in A \mid \psi(a) = 0\} = \left\{a \in A \mid \bar{\psi(a)}^\q = 0\right\} = \{a \in A \mid \varphi(a) \in \q\} = \varphi^\mo(\q) = \p$. \qed

\subsection{Beispiel} Sei $K$ ein Körper. Im Polynomring $K[X,Y]$ ist $(X)$ ein Primideal, denn $K[X,Y]/(X) \eqtilde K[Y]$ ist ein Integritätsring (betrachte den Einsetzungshomomorphismus $K[X,Y] \to K[Y], p \mapsto p(0,Y)$ und wende den Isomorphiesatz \ref{fixed:2.1.17} an). Es ist $(X)$ kein maximales Ideal, denn $K[X,Y]/(X) \eqtilde K[Y]$ ist kein Körper. Dagegen ist $(X,Y)$ ein maximales Ideal von $K[X,Y]$, denn $K[X,Y]/(X,Y) \eqtilde K$ ist ein Körper (betrachte $K[X,Y] \to K, p \mapsto (0,0)$).

\subsection{Satz} In einem Hauptidealring ist jedes Primideal $\neq \{0\}$ ein maximales Ideal.

\proof Sei $R$ ein Hauptidealring und $\p \neq \{0\}$ ein Primideal in $R$. Sei $I$ ein Ideal von $R$ mir $p \subseteq I$. Zu zeigen: $I = \p$ oder $I = R$. Wähle $p,a \in R$ mit $\p = (p)$ und $I = (a)$. Die Bedingung $p \subseteq I$ bedeutet $(p) \subseteq (a)$, d.~h.~$p \in (a)$. Wähle $b \in R$ mit $p = ab$. Da $p$ gemäß \ref{fixed:2.4.3} prim ist und $R$ ein Integritätsring ist, ist $p$ irreduzibel in $R$. Also gilt $a \in R^\times$ oder $b \in R^\times$, also $I = (a) = R$ oder $I = (a) = (b^\mo p) \subseteq (p) = \p \subseteq I$. Also $I = R$ oder $I = \p$ wie gewünscht.


\numberfix{3}{2}{0}
\tikzset{
	nodes around center/.style args={#1:#2:#3:#4}{
		at={([shift={(#3)}] {{(\tikzchaincount-1)*360/(#2)+#1}}:{#4})}
	},
	nodes around center*/.style args={#1:#2:#3:#4}{
		at={([shift={(#3.{(\tikzchaincount-1)*360/(#2)+#1})}] {{(\tikzchaincount-1)*360/(#2)+#1}}:{#4})},
		anchor={(\tikzchaincount-1)*360/(#2)+#1+180}
	}
}

\section{Auflösbare Gruppen}

\subsection{Definition} Sei $G$ eine Gruppe. Für $a,b \in G$ nennt man $[a,b]:=aba^{-1}b^{-1}$ den Kommutator\index{Kommutator} von $a$ und $b$. Man nennt $G':=\langle\{[a,b]\mid a,b\in G\}\rangle\leq G$ die Kommutatorgruppe\index{Kommutatorgruppe} von $G$. Weiter definiert man für
jedes $n\in\N_0$ die $n$-te Kommutatorgruppe $G^{(n)}$ von $G$ rekursiv durch $G^{(0)}:=G$ und $G^{(n+1)}:=(G^{(n)})'$ für $n\in\N_0$.

\subsection{Bemerkung} Sei $G$ eine Gruppe.
\begin{enumerate}[label=(\alph*)]
	\item
		$\forall a,b\in G:([a,b]=1\iff ab=ba)$
		
	\item
		$G'=\{[a_1,b_1]\cdots[a_m,b_m]\mid m\in\N_0,~a_i,b_i\in G\}$

		["`$\supseteq$"' klar; "`$\subseteq$"' beachte $[a,b]^{-1}=(aba^{-1}b^{-1})^{-1}=bab^{-1}a^{-1}=[b,a]$ für $a,b\in G$]
		
	\item
		$G'$ ist der kleinste Normalteiler $N$ von $G$ mit $G/N$ abelsch.

		[$G'$ ist nach \ref{fixed:1.3.12} eine charakteristische Untergruppe und daher ein Normalteiler von $G$; ist $N \normsub G$ mit $G/N$ abelsch, so $\bar{[a,b]}^N=\bar a \bar b \bar a^{-1} \bar b^{-1}=\bar{aa^{-1}}^N \bar{bb^{-1}}^N=1$ und daher $[a,b]\in N$ für alle $a,b\in G$, woraus $G'\subseteq N$ folgt.]
\end{enumerate}

\subsection{Definition} Sei $n\in\N_0$. Eine Permutation der Form
$$	(x_1,...,x_\ell):=
	\left(
		\begin{array}{l}
			\{1,...,n\} \to \{1,...,n\} \\
			\begin{tikzpicture}
			  \node (Z) {};
			  \begin{scope}[
			    start chain=circle placed {nodes around center=270:6:Z:3em},
			    every join/.append style={<-|},
			    every node/.append style={
			      on chain=circle,
			      join,
			      minimum size=2em
			      }
			  ]
			    \foreach \cnt in {4,...,1}
			      \node {$x_\cnt$};
			          \node {$x_\ell$};
			    \node[rotate=130] {\dots};
			    \chainin (circle-begin);
			  \end{scope}
			\end{tikzpicture}\\
			x \mapsto x \text{ für } x \in \{1,...,n \} \setminus \{x_1,...,x_\ell\}
		\end{array}
	\right)
$$
mit $\ell\in\{2,...,n\}$ und paarweise verschiedenen $x_1,...,x_\ell\in\{1,\dots,n\}$ nennt man einen $\ell$-Zykel\index{Zykel@$\ell$-Zykel} in $S_n$. Man nennt $2$-Zykel auch Transpositionen \LAref{9.1.3}.

\subsection{Proposition} \ALref{\ref{fixed:1.1.12}} Sei $n\in\N_0$. Dann
$$A_n=\{\sigma_1\cdots\sigma_m\mid m\in\N_0,~\sigma_1,\dots,\sigma_m\text{ $3$-Zykel in }S_n\}.$$

\proof "`$\supseteq$"': Seien $x_1,x_2,x_3\in\{1,...,n\}$ paarweise verschieden. Zu zeigen: $(x_1\ x_2\ x_3)\in A_n$. Dies folgt aus $(x_1\ x_2\ x_3)=(x_2\ x_3)(x_1\ x_3)$.

"`$\subseteq$"': Sind $x_1,x_2,x_3,x_4\in\{1,...,n\}$ paarweise verschieden, so
$(x_1\ x_2)(x_3\ x_4)=(x_1\ x_3\ x_2)(x_1\ x_3\ x_4)$. Sind $x_1,x_2,x_3\in\{1,\dots,n\}$ paarweise verschieden, so $(x_1\ x_2)(x_2\ x_3)=(x_1\ x_2\ x_3)$. Sind $x_1,x_2\in\{1,\dots,n\}$ mit $x_1\ne x_2$, so $(x_1\ x_2)(x_1\ x_2)=1$. \qed

\subsection{Proposition} Sei $n\in\N_0$. Dann $S_n'=A_n$ und
$$
A_n'=
\begin{cases}
	\{1\}&\text{falls $n\le 3$,}\\
	V_4:=\{1,(1\ 2)(3\ 4),(1\ 3)(2\ 4),(1\ 4)(2\ 3)\}\cong V&\text{falls $n=4$,}\\
	A_n&\text{falls $n\ge5$.}
\end{cases}
$$
\begin{figure}[h]
	\centering
	\tikzstyle{every node} = [align=center]
	\begin{tikzpicture}[baseline=1.1em]
		\coordinate (a) at (0,0);
		\coordinate (b) at (2,0);
		\coordinate (c) at (2,1);
		\coordinate (d) at (0,1);
		\draw (a) -- (b) -- (c) -- (d) -- (a);
		\fill (a) circle (3pt);
		\fill (b) circle (3pt);
		\fill (c) circle (3pt);
		\fill (d) circle (3pt);
		\node[below left] at (0,0) {3};
		\node[below left] at (0,0) {3};
		\node[above left] at (0,1) {2};
		\node[above right] at (2,1) {1};
		\node[below right] at (2,0) {4};
		\draw (-1,0.5) -- (3,0.5);
		\draw (1,-0.5) -- (1,1.5);
	\end{tikzpicture} \quad {\ALref{\ref{fixed:1.1.9e}}}
\end{figure}

\proof ~

\underline{$S_n'\subseteq A_n$:} Nach \ref{fixed:3.3.2c} genügt es zu zeigen, dass $S_n/A_n$ abelsch ist. Dies ist klar, da $S_n/A_n \eqtilde C_2$ für $n\ge2$ \ref{fixed:1.3.18} und $S_n/A_n \eqtilde C_1$ für $n\in\{0,1\}$.

\underline{$A_n\subseteq S_n'$:} Nach \ref{fixed:3.3.4} genügt es zu zeigen, dass jeder $3$-Zykel in $S_n'$ liegt. Seien hierzu $x_1,x_2,x_3$ paarweise verschieden. Dann
$$(x_1\ x_2\ x_3)=(x_1\ x_3)(x_2\ x_3)(x_1\ x_3)^{-1}(x_2\ x_3)^{-1}=[(x_1\ x_3),(x_2\ x_3)]\in S_n'.$$


\underline{$A_n'=\{1\}$ für $n\le3$:} Für $n\le3$ ist $A_n \eqtilde A_n/\{1\}$ abelsch, da $\#A_n\le\#A_3=\frac{\#S_3}2=\frac{3!}2=3$.

\underline{$A_4'=V_4$:} "`$\subseteq$:"' Wegen $\#A_4=\frac{4!}2=4\cdot3=12$ gilt
$\#(A_4/V_4)=3$ und $A_4/V_4$ ist abelsch.

"`$\supseteq$:"' Ist $\{x_1,x_2,x_3,x_4\}=\{1,2,3,4\}$, so nach \ref{fixed:3.3.4}
\begin{align*}
	(x_1\ x_2)(x_3\ x_4)&=(x_1\ x_2\ x_3)(x_1\ x_2\ x_4)(x_1\ x_2\ x_3)^{-1}(x_1\ x_2 x_4)^{-1}\\
	&=[\underbrace{(x_1\ x_2\ x_3)}_{\in A_4},\underbrace{(x_1\ x_2\ x_3)}_{\in A_4}]\in A_4'.
\end{align*}

\underline{$A_n'=A_n$ falls $n\ge5$:} Sei $n\ge5$. Zu zeigen: $A_n\subseteq A_n'$. Seien $x_1,x_2,x_3\in\{1,...,n\}$ paarweise verschieden. Zu zeigen: $(x_1\ x_2\ x_3)\in A_n'$. Wähle $x_4,x_5\in\{1,...,n\}\setminus\{x_1,x_2,x_3\}$ mit $x_4\ne x_5$. Dann
$$
	(x_1\ x_2\ x_3)
	=(x_1\ x_2\ x_4)(x_1\ x_3\ x_5)(x_1\ x_2\ x_4)^{-1}(x_1\ x_3\ x_5)^{-1}
	=[(x_1\ x_2\ x_4),(x_1\ x_3\ x_5)]\in A_n'.
$$ \qed

\subsection{Definition} Sei $G$ eine Gruppe. Es heißt $(G_0,...,G_n)$ eine Normalreihe\index{Normalreihe} von $G$, wenn $G=G_0\normsup G_1\normsup...\normsup G_n=\{1\}$. In diesem Fall heißen die Gruppen $G_k/G_{k+1}$ ($k\in\{0,... n-1\}$) die Faktoren\index{Faktor!einer Normalreihe} dieser Normalreihe. Es heißt $G$ auflösbar\index{auflösbar}, wenn $G$ eine Normalreihe mit (lauter) abelschen Faktoren besitzt.

\subsection{Satz} Sei $G$ eine Gruppe. Dann
$$\text{$G$ auflösbar}\iff\exists n\in\N_0:G^{(n)}=\{1\}.$$

\proof "`$\Longleftarrow$"' Ist $n\in\N_0$ mit $G^{(n)}=\{1\}$, so ist $(G^{(0)},...,G^{(n)})$ eine Normalreihe von $G$ mit abelschen Faktoren.

"`$\implies$"' Sei $(G_0,...,G_n)$ eine Normalreihe von $G$ mit abelschen Faktoren. Wir zeigen durch Induktion nach $k\in\{0,...,n\}$, dass $G^{(k)}\subseteq G_k$:

\underline{$k=0$:} $G^{(0)}=G=G_0$

\underline{$k\to k+1$ \quad ($k\in\{0,...,n-1\}$)} \quad $G^{(k+1)}=(G^{(k)})' \overset{\text{IV}}{\underset{G^{(k)}\subseteq G_k}\subseteq} G_k'
\overset{\substack{G_k/G_{k+1} \\ \text{abelsch}}}\subseteq G_{k+1}$ \qed

\subsection{Satz} $S_n$ ist auflösbar für $n\le4$, nicht aber für $n\ge5$.

\proof Nach Proposition \ref{fixed:3.3.5} gilt $S_n^{(2)}=A_n'=\{1\}$ für $n\le3$,
$$S_4^{(3)}=A_4^{(2)}=V_4'\overset{\substack{V_4 \eqtilde V \eqtilde C_2\times C_2\\ \text{abelsch}}}=\{1\}$$
und $S_n^{(1)}=S_n^{(2)}=...=A_n\ne\{1\}$ für $n\ge5$. \qed

\subsection{Proposition} Sei $G$ eine Gruppe.
\begin{enumerate}[label=(\alph*)]
	\item
		Ist $G$ auflösbar und $H\le G$, so ist auch $H$ auflösbar.
		
	\item
		Ist $N \normsub G$, so
		$$\text{$G$ auflösbar}\iff(\text{$N$ auflösbar} \text{ \& } \text{$G/N$ auflösbar}).$$
\end{enumerate}

\proof ~

\textbf{zu (a):} Klar, da man durch Induktion $H^{(n)}\subseteq G^{(n)}$ für alle $n\in\N_0$ zeigt.

\textbf{zu (b):} Gelte $N \normsub G$. Durch Induktion zeigt man $(G/N)^{(n)}=(G^{(n)}N)/N$ für alle $n\in\N_0$ \ref{fixed:1.4.1}:

\underline{$n=0$:} $G/N=\underbrace{(GN)}_{=G}/N$

\underline{$n\to n+1$ ($n\in\N_0$):}
\begin{eqnarray*}
	(G/N)^{(n+1)}&=&((G/N)^{(n)})'\overset{\text{IV}}=((G^{(n)}N)/N)'\\
	&\overset{\text{\ref{fixed:3.3.2b}}}=&\{[\bar{a_1n_1}^N,\bar{a_1'n_1'}^N]\dotsm[\bar{a_mn_m}^N,\bar{a_m'n_m'}^N]
	\mid m\in\N_0,~a_i,a_i'\in G^{(n)},~n_i,n_i'\in\N\}\\
	&=&\{\bar{[a_1,a_1']\cdots[a_m,a_m']}^N\mid m\in\N_0,~a_i,a_i'\in G^{(n)}\}\\
	&\overset{\text{\ref{fixed:3.3.2b}}}=&\{\bar g^N\mid g \in G^{(n+1)}\} = \{\bar{gn}^N\mid g \in G^{(n+1)},~n\in N\}=(G^{(n+1)}N)/N
\end{eqnarray*}

"`$\implies$"' Ist $n\in\N$ mit $G^{(n)}=\{1\}$, so $(G/N)^{(n)}=(G^{(n)}N)/N=N/N=\{1\}$.

"`$\Longleftarrow$"' Ist $n\in\N$ mit $N^{(n)}=\{1\}$ und $(G/N)^{(n)}=\{1\}$, so $(G^{(n)}N)/N=N/N$, also $G^{(n)}\subseteq N$ und $G^{(2n)}\subseteq N^{(n)}=\{1\}$. \qed

\subsection{Satz} Sei $p\in\P$. Jede $p$-Gruppe ist auflösbar.

\proof Wir zeigen durch Induktion nach $e\in\N_0$, dass alle Gruppen $G$ mit $\#G=p^e$ auflösbar sind.

\underline{$e=0$:} \checkmark

\underline{$0,...,e-1\to e$\quad($e\in\N$):} Sei $G$ eine Gruppe mit $\#G=p^e$. Nach \ref{fixed:3.1.18} gilt $\#Z(G)>1$. Nach dem Satz von Lagrange \ref{fixed:1.3.19} gibt es also $d\in\{0,...,e-1\}$ mit $\#(G/Z(G))=p^d$ (siehe auch \ref{fixed:1.3.14}). Nach Induktionsvoraussetzung ist $G/Z(G)$ auflösbar. Da $Z(G)$ abelsch und daher auch auflösbar ist, folgt mit \ref{fixed:3.3.9b}, dass auch $G$ auflösbar ist. \qed

\subsection{Proposition} Sei $G$ eine Gruppe und $N \normsub G$. Bezeichne $\pi : G\to G/N,\ a \mapsto \bar a^N$ den kanonischen Epimorphismus. Dann wird durch die Zuordnungen
\begin{align*}
	I & \mapsto\pi(I)=I/N\qquad\text{und} \\
	\pi^{-1}(J) & \mapsfrom J
\end{align*}
eine Bijektion zwischen der Menge der Untergruppen (Normalteiler) $I$ von $G$ mit $N\subseteq I$ und der Menge der Untergruppen (Normalteiler) von $G/N$ definiert.

\proof Übung. \qed

\subsection{Satz} Sei $G$ eine endliche Gruppe und $(G_0,...,G_m)$ eine Normalreihe von $G$ mit abelschen Faktoren. Dann gibt es eine Normalreihe $(H_0,...,H_n)$ von $G$ mit $\{G_0,...,G_m\}\subseteq\{H_0,...,H_n\}$, deren Faktoren $H_k/H_{k+1}$ alle
zyklisch von Primzahlordnung sind.

\proof Ohne Einschränkung
$$G=G_0\underset\ne\normsup G_1\underset\ne\normsup\dots\underset\ne\normsup G_m=\{1\}.$$
Sei $k\in\{0,\dots,m-1\}$ mit $\#(G_k/G_{k+1})\not\in\P$. Dann gibt es sicher $J$ mit $$\{1\}\underset{\text{echt}}<J\underset{\text{echt}}<G_k/G_{k+1}$$ (z.B. wegen \ref{fixed:3.2.6a} oder indem man $J$ einfach als geeignete zyklische Untergruppe von $G_k/G_{k+1}$ wählt). Da $G_{k+1}/G_k$ abelsch ist, gilt
$$\{1\}\underset\ne\normsub J\underset\ne\normsub G_k/G_{k+1}.$$
Für $I:=\pi^{-1}(J)$ mit $\pi : G_k\to G_k/G_{k+1}$ kanonisch gilt nach \ref{fixed:3.3.11} dann
$$G_k\underset\ne\normsup I\underset\ne\normsup G_{k+1}.$$
Es ist $I$ der Kern von $G_k\MAPepi G_k/G_{k+1}\MAPepi(G_k/G_{k+1})/J$ und daher $G_k/I\eqtilde\underbrace{(\underbrace{G_k/G_{k+1}}_{\text{abelsch}})/J}_{\text{abelsch}}$ abelsch. Weiter ist $I/G_{k+1}\le\underbrace{G_k/G_{k+1}}_{\text{abelsch}}$ auch abelsch. Mache nun so weiter... \qed
\chapter{Körper \small\LAref{§ 4}}

\section{Endliche und algebraische Körpererweiterungen}

\subsection{Definition} Sei $L | K$ eine Körpererweiterung \ALref{\ref{fixed:2.3.11}}. Die Dimension $[L:K] := \dim_K L \in N \cup \{\infty\}$ des $K$-Vektorraums $L$ \LAref{§ 6.1} nennt man den (Körper-)Grad\index{Grad!einer Körpererweiterung}\index{Körpererweiterung!Grad} von $L$ über $K$ (nicht zu verwechseln mit dem Index aus \ref{fixed:1.3.19}!). Ist $[L:K] < \infty$ ($[L:K] = \infty$), so nennt man L endlich (unendlich) über $K$ und $L|K$ eine endliche\index{Körpererweiterung!endliche} (unendliche\index{Körpererweiterung!unendliche}) Körpererweiterung.

\subsection{Beispiel}
\begin{enumerate}[label=(\alph*)]
	\item
		$[K:K] = 1$ für jeden Körper $K$.
		
	\item
		$[K(X):K] = \infty$ für jeden Körper $K$.
		
	\item
		$[\C:\R] = 2$
\end{enumerate}

\subsection{Proposition} Sei $L|K$ eine Körpererweiterung von $V$ ein $L$-Vektorraum (und damit auch ein $K$-Vektorraum). Sei $A$ eine Basis des $K$-Vektorraums $L$ und $B$ eine Basis des $L$-Vektorraums $V$. Dann ist $A \times B \to AB := \{ab \mid a \in A b \in B\},~(a,b) \mapsto ab$ bijektiv und $AB$ eine Basis des $K$-Vektorraums $V$.

\proof Zu zeigen:
\begin{enumerate}[label=(\alph*)]
	\item
		$\spn_K AB = V$
		
	\item
		Für paarweise verschiedene $a_1,...,a_m \in A$ und paarweise verschiedene $b_1,...,b_n \in B$ sind $a_ 1b_1,...,a_1 b_n,...,a_m b_1,...,a_m b_n$ linear unabhängig.
\end{enumerate}

Zu (a). Für jedes $\lambda \in L$ und $b \in B$ gilt $\lambda \in \spn_K A$ und daher $\lambda b \in \spn_K Ab \subseteq \spn_K AB$. Daraus folgt $V = \spn_L B \subseteq \spn_K AB \subseteq V$.

Zu (b). Seien $\lambda_{ij} \in K$ ($1 \leq i \leq m, 1 \leq j \leq n$) mit $\sum_{i=1}^m  \sum_{j=1}^n \lambda_{ij} a_i b_j = 0$. Dann $\sum_{j=1}^n \left( \sum_{i=1}^m \lambda_{ij} a_i \right) b_j = 0$ und daher $\sum_{i=1}^m \lambda_{ij} a_i = 0$ für alle $j$, also $\lambda_{ij} = 0$ für alle $i,j$. \qed

\subsection{Sprechweise} Ein Zwischenkörper\index{Körper!Zwischenkörper} einer Körpererweiterung $L|K$ ist ein Unterkörper von $L$, der $K$ enthält.

\subsection{Korollar} Sei $F$ ein Zwischenkörper der Körpererweiterung $L|K$. Dann ist $L|K$ endlich genau dann, wenn $L|F$ und $F|K$ beide endlich sind, und in diesem Fall gilt die sogenannte "`Gradformel"'\index{Gradformel} $$[L:K] = [L:F][F:K].$$

\subsection{Definition} Sei $L|K$ eine Körpererweiterung. Dann heißt $a \in L$ algebraisch\index{algebraisch!-es Element} über $K$, wenn es $f \in K[x]\setminus\{0\}$ gilt mit $f(a)=0$ [das heißt, wenn $a$ nicht algebraisch unabhängig über $K$ ist, \ALref{\ref{fixed:2.2.3(a)}}]. Es heißt $L|K$ algebraisch\index{algebraisch!-e Körpererweiterung}, wenn jedes Element von $L$ algebraisch über $K$ ist.

\subsection{Beispiel}
\begin{enumerate}[label=(\alph*)]
	\item
		$\sqrt{2}$ ist algebraisch über $\Q$, denn $\left(\sqrt{2}\right)^2-2 = 0$.
		
	\item
		$\i$ und $\i+1$ sind algebraisch über $\Q$, denn $\i^2+1 = 0$ und $(\i+1)^2-2(\i+1)+2 = 0$.
		
	\item
		$K \in K(X)$ ist nicht algebraisch über $K$. ($K$ ein Körper.)
\end{enumerate}

\subsection{Definition} Sei $L|K$ eine Körpererweiterung und $a \in L$ algebraisch über $K$. Dann ist der Kern von $K[X] \to L,~f \mapsto f(a)$ ein Ideal von $K[X]$, welches von einem eindeutig bestimmten normierten Polynom erzeugt wird \LAref{10.2.4}, dem sogenannten Minimalpolynom\index{Minimalpolynom}\index{Polynom!Minimal-} $\irr_K(a) \in K[X]$.

\subsection{Proposition} Sei $L|K$ eine Körpererweiterung und $a \in L$ algebraisch über $K$. Dann sind für $f \in K[X]$ äquivalent:
\begin{enumerate}[label=(\alph*)]
	\item
		$f = \irr_K(a)$
		
	\item
		$f$ ist \textit{das} normierte Polynom kleinsten Grades mit $f(a)=0$.
		
	\item
		$f$ ist normiert und irreduzibel in $K[X]$ und es gilt $f(a)=0$.
		
	\item
		$f$ ist das Minimalpolynom des $K$-Vektorraumendomorphismus $\lambda_a : L \to L,~b \mapsto ab$.
\end{enumerate}

\proof ~

\underline{(a) $\Longrightarrow$ (b):} Klar

\underline{(b) $\Longrightarrow$ (c):} Gelte (b). Zu zeigen ist $f$ irreduzibel. Es gilt $f \in K[X]^\times = K^\times$, da $f(a)=0$. Seien $g,h \in K[X]$ mit $f = gh$. Zu zeigen ist $g \in K^\times$ oder $h \in K^\times$. Wegen $g(a)h(a) = (gh)(a) = f(a) = 0$ gilt $g(a) = 0$ oder $h(a) = 0$. Dann gilt aber $\deg g \geq \deg f$ oder $\deg h \geq \deg f$ und daher $h \in K^\times$ oder $g \in K^\times$.

\underline{(c) $\Longrightarrow$ (a):} Gelte (c). Wegen $f(a) = 0$ gilt dann $f \in (\irr_K(a))$\footnote{Korrektur: Hier fehlt doch was um die Klammern?}, das heißt, es gibt $g \in K[X]$ mit $f = g \irr_K(a) \in K^\times$. Letzteres ist unmöglich, also $g\in K^\times$ und sogar $g=1$, da $f$ und $\irr_K(a)$ beide normiert sind.

\underline{(a) $\iff$ (d):} Es reicht zu zeigen, dass für alle $g \in K[X]$ gilt: $g(a) = 0 \iff g(\lambda_a) = 0$ \LAref{10.2.18}. Dies folgt aus $(g(\lambda_a))(b) = (g(a))b$ für alle $b \in L$. \qed

\subsection{Proposition} Sei $L|K$ eine Körpererweiterung und $a \in L$ algebraisch über $K$. Dann ist $K[X]/(\irr_K(a))$ ein Körper und $K[X]/(\irr_K(a)) \to K[a],~\bar f \mapsto f(a)$ ein Isomorphismus. Insbesondere ist $K[a] = K(a)$ auch ein Körper und $\deg \irr_K(a) = [K(a):K]$.

\proof Nach dem Isomorphiesatz für Ringe und für $K$-Vektorräume liefert der Einsetzungshomomorphismus $K[X] \MAPepi K[a],~f \mapsto f(a)$ den Ring - und $K$-Vektorraumisomorphismus $K[X]/(\irr_K(a)) \to K[a],~\bar f \mapsto f(a)$.

Da $\irr_K(a)$ irreduzibel im Hauptidealring $K[X]$ ist, ist $K[X]/(\irr_K(a))$ nach \ref{fixed:2.4.9} (siehe auch \ref{fixed:2.4.10(b)}) ein Körper. Daher ist auch der dazu isomorphe Ring $K[a]$ ein Körper, das heißt $K[a] = K(a)$ \ALref{\ref{fixed:2.3.11(b)}}. Setzt man nun $d := \deg \irr_K(a)$, so bilden $\bar 1, \bar X, ..., \bar X^{d-1}$ offensichtlich eine Basis des $K$-Vektorraumes $K[X]/(\irr_K(a))$ und daher deren Bilder $1, a, ..., a^{d-1}$ eine Basis des $K$-Vektorraums $K[a] = K(a)$. Insbesondere ist $d = [K(a):K]$. \qed

\subsection{Beispiel} $\irr_\Q(\sqrt 2) = X^2 - 2$, $\Q(\sqrt 2) = \Q[\sqrt 2] \eqtilde \Q[X] / (X^2 - 2)$ und $1, \sqrt 2$ bilden eine $\Q$-Basis von $\Q(\sqrt 2)$.

\subsection{Satz} Sei $L|K$ eine Körpererweiterung und $a \in L$. Dann sind äquivalent:
\begin{enumerate}[label=(\alph*)]
	\item
		$a$ ist algebraisch über $K$
		
	\item
		$K(a)|K$ ist endlich
		
	\item
		$K[a] = K(a)$
\end{enumerate}

\proof ~\\
\underline{(a) $\implies$ (b):} Nach \ref{fixed:4.1.10}.

\underline{(b) $\implies$ (a):} Ist $d := [K(a) : K] < \infty$, so sind $1,a,...,a^d$ linear abhängig im $K$-Vektorraum $K(a)$

\underline{(a) $\implies$ (c):} Nach \ref{fixed:4.1.10}

\underline{(c) $\implies$ (a):} Ist $a$ nicht algebraisch über $K$, das heißt $a$ algebraisch unabhängig über $K$, so ist $K[a]$ ein Polynomring über $K$ und daher $K[a]^\times = K^\times \neq K[a] \setminus \{0\}$. Insbesondere ist dann $K[a]$ kein Körper und daher $K[a] \neq K(a)$. \qed

\subsection{Korollar} Jede endliche Körpererweiterung ist algebraisch.

\subsection{Proposition} Sei $L|K$ eine Körpererweiterung und $a_1,...,a_n \in L$ algebraisch über $K$ mit $L = K(a_1,...,a_n)$. Dann gilt $L = K[a_1,...,a_n]$ und $L|K$ ist endlich.

\proof Für jedes $i \in \{1,...,n\}$ ist $a_i$ insbesondere algebraisch über $K(a_1,...,a_{i-1})$ und daher nach \ref{fixed:4.1.12} auch $K(a_1,...,a_i)$ über $K(a_1,...,a_{i-1})$ endlich.

Es folgt mir \ref{fixed:4.1.5}, dass $L|K$ endlich ist und mit \ref{fixed:4.1.12}, dass $L = K(a_1) \cdots (a_n) = K[a_1] \cdots [a_n] = K[a_1,...,a_n]$. \qed

\subsection{Definition} Eine Körpererweiterung $L|K$ heißt endlich erzeugt\index{Körpererweiterung!endlich erzeugte}, wenn es $n \in \N_0$ und $a_1,...,a_n \in L$ gibt mit $L = K(a_1,...,a_n)$.

\subsection{Korollar} Sei $L|K$ eine Körpererweiterung. Dann ist $L|K$ endlich genau dann, wenn $L|K$ endlich erzeugt und algebraisch ist.

\subsection{Satz (Transitivität der Algebraizität)} Sei $F$ ein Zwischenkörper von $L|K$ und $F|K$ algebraisch. Ist $a \in L$ algebraisch über $F$, so ist $a$ auch algebraisch über $K$.

\proof Bezeichne die Koeffizienten von $\irr_F(a) \in F[X]$ mit $a_1,...,a_n \in F$. Dann ist $a$ sogar algebraisch über $K(a_1,...,a_n)$.

Da die Körpererweiterung $K(a_1,...,a_n)|K$ endlich erzeugt und algebraisch ist, ist sie auch endlich. Da $K(a_1,...,a_n)(a) | K(a_1,...,a_n)$ auch endlich ist, ist nach \ref{fixed:4.1.5} $K(a_1,...,a_n,a) | K$ endlich und damit algebraisch. Insbesondere ist $a$ algebraisch über $K$. \qed

\subsection{Korollar} Sei $F$ ein Zwischen Körper von $L|K$. Dann ist $L|K$ algebraisch genau dann, wenn $L|F$  beide algebraisch sind \ALref{vgl.~\ref{fixed:4.1.5}}.\footnote{Korrektur: Aussage wahrscheinlich so nicht richtig?}

\subsection{Definition und Satz} Sei $L|K$ eine Körpererweiterung. Dann ist $\barsmash KL := \{a \in L \mid a \text{ algebraisch über } K\}$ ein Zwischenkörper von $L|K$, genannt der (relative) algebraische Abschluss\index{algebraischer Abschluss!relativer} von $K$ über $L$.

\proof Zu zeigen sind:
\begin{enumerate}[label=(\alph*)]
	\item
		$L \subseteq \barsmash KL$
		
	\item
		$\forall a,b \in \barsmash KL : a+b, a \cdot b \in \barsmash KL$
		
	\item
		$\forall a \in \barsmash KL \setminus \{0\} : \frac{1}{a} \in \barsmash KL$
\end{enumerate}

\textbf{Zu (a).} Ist klar.

\textbf{Zu (b).} Sind $a,b \in \barsmash KL$, so ist $K(a,b)|K$ endlich nach \ref{fixed:4.1.14} und damit algebraisch und daher $a+b,a \cdot b \in K(a,b)$ algebraisch über $K$.

\textbf{Zu (c).} Zeigt man genauso.

\subsection{Beispiel} Den Körper $\barsmash \Q\C$ ($\barsmash \Q\R$) nennt man den Körper der algebraischen (reellen algebraischen) Zahlen\index{Körper!der (reellen) algebraischen Zahlen}.

\section{Der algebraische Abschluss}

\subsection{Satz von Kronecker} Sei $K$ ein Körper und $f \in K[X]$ irreduzibel und normiert. Dann gibt es eine endliche Körpererweiterung $L|K$ und ein $a \in L$ mit $L =  K(a)$ und $\irr_K(a) = f$.

\proof [Nach \ref{fixed:4.1.10} ist klar, dass der gesuchte Körper, falls er existiert, isomorph zu {$K[X]/(f)$} sein muss.]

$L := K[X]/(f)$ ist nach \ref{fixed:2.4.9} ein Körper. $K' := \{\bar b \mid b \in K\}$ ist ein zu $K$ isomorpher Unterkörper von $L$, da $K \MAPmono L,~b \mapsto \bar b$ und $f' := \phi(f) \in K'[X]$ mit $\phi : K[X] \MAPiso K'[X],~ b \mapsto \bar b~(b \in K),~X \mapsto X$.

Es reicht, die Behauptung für $(K',f')$ statt $(K,f)$ zu zeigen. Setzt man $a := \bar X \in L$, so ist $f' \in K'[X]$ irreduzibel mit $f'(a) = f'(\bar X) = \bar f = 0$ und daher $f' = \irr_{K'}(a)$ nach \ref{fixed:4.1.9}. \qed

\subsection{Korollar} Sei $K$ ein Körper und $f \in K[X] \setminus K$. Dann gibt es ein $L|K$ und ein $a \in L$ mit $[L:K] \leq \deg f$ und $f(a) = 0$.

\proof Wähle $g \in K[X]$ irreduzibel mit $g|f$. Wende \ref{fixed:4.2.1} auf $g$ an.

\subsection{Beispiel} \LAref{§ 4.2} Sei $K$ ein Körper, in dem es kein $a \in K$ gibt mit $a^2 = -1$. Dann ist $X^2+1$ irreduzibel in $K[X]$ und es gibt $L|K$ und $\i \in L$ mit $L = K(\i)$ und $\irr_K(\i) = X^2+1$.

\subsection{Definition} Ein Körper $K$ heißt algebraisch abgeschlossen\index{algebraisch abgeschlossen}, wenn jedes Polynom aus $K[X] \setminus K$ eine Nullstelle in $K$ hat.

\subsection{Bemerkung} Der noch zu beweisende Fundamentalsatz der Algebra besagt, dass $\C$ algebraisch abgeschlossen ist \LAref{4.2.12}.

\subsection{Proposition} Sei $K$ ein Körper. Dann sind äquivalent:
\begin{enumerate}[label=(\alph*)]
	\item
		$K$ ist algebraisch abgeschlossen.
		
	\item
		Jedes Polynom aus $K[X]\setminus\{0\}$ zerfällt \LAref{10.1.13}.
		
	\item
		Jedes irreduzible Polynom aus $K[X]$ hat den Grad $1$.
		
	\item
		$K$ ist der einzige über $K$ algebraische Oberkörper von $K$.
		
	\item
		$K$ ist der einzige über $K$ endliche Oberkörper von $K$.
\end{enumerate}

\proof ~

\underline{(a) $\implies$ (b):} Durch sukzessives Abspalten von Nullstellen \LAref{4.2.10}.

\underline{(b) $\implies$ (c):} Klar.

\underline{(c) $\implies$ (d):} Gelte (c). Sei $L|K$ algebraisch. Zu zeigen ist $L=K$. Sei $a \in L$. Zu zeigen ist $a \in K$. Nach (c) gilt $\irr_K(a) = X-c$ für ein $c \in K$. Dann aber $a - c = 0$, also $a = c \in K$.

\underline{(d) $\implies$ (e):} Klar nach \ref{fixed:4.1.13}.

\underline{(e) $\implies$ (a):} Gelte (e) und sei $f \in K[X] \setminus K$. Nach \ref{fixed:4.2.2} gibt es eine endliche Erweiterung $L$ von $K$ und ein $a \in L$ mit $f(a) = 0$. Nach (e) gilt $L=K$ und daher $a \in K$. \qed
\subsection{Lemma} Sei $K$ ein Körper. Dann gibt es eine algebraische Körpererweiterung $L|K$ derart, dass jedes Polynom aus $K[X] \setminus K$ in $L$ eine Nullstelle hat.

\proof Wir treiben die Beweisidee des Satzes von Kronecker \ref{fixed:4.2.1} bis zum Exzess. Definiere \ALref{\ref{fixed:2.2.10}}
$$I := \left( \{ f \in X_f \mid f \in K[X] \setminus K \} \right) \subseteq K[X_f \mid f \in K[X] \setminus K] =: A \footnote{Korrektur: Kann ich nicht lesen.}$$

Wir zeigen $1 \ not\in I$ und nehmen hierzu an $1 \in I$. Wähle $f_1,...,f_n \in K[X] \setminus K$ und $g_1,...,g_n \in A$ mit
\begin{equation}
	1 = \sum_{i=1}^n g_if_iX_{f_i} \footnote{Korrektur: Kann ich nicht lesen.} \tag{$\ast$}
\end{equation}
alle $f_i$ (und damit $X_{f_i}$) paarweise verschieden. Durch $n$-faches Anwenden von \ref{fixed:4.2.2} erhält man sukzessive $L|K$ und $a_1,...,a_n \in L$ mit $f_i(a_i) = 0$ für $i \in \{1,...,n\}$. Durch Einsetzen von $a_i$ für $X_{f_i}$ und zum Beispiel $0$ für die übrigen Unbestimmten in ($\ast$), folgt $1=0$.

Wegen $ 1 \not\in I$ gibt es nach \ref{fixed:2.4.14a} ein maximales Ideal $\m$ von $A$ mit $I \subseteq \m$. Dann ist $L := A/\m$ nach \ref{fixed:2.4.4b} ein Körper. Definiere $K' := \{ \bar b \mid b \in K \} \eqtilde K \subseteq L$. Es reicht zu zeigen:
\begin{enumerate}[label=(\alph*)]
	\item
		$L|K'$ ist algebraisch.
		
	\item
		Jedes Polynom aus $K'[X] \setminus K'$ hat in $L$ eine Nullstelle.
\end{enumerate}

\proof ~

\textbf{Zu (a).} $L = K'[\bar X_f \mid f \in K[x] \setminus K] \subseteq \barsmash{K'}L$, denn für alle $f \in K[X] \setminus K$ ist $\bar X_f$ algebraisch über $K'$. In der Tat: Definiert man $f' \in K'[X] \setminus K'$ wie im Beweis von \ref{fixed:4.2.1}, so gilt $f'(\bar X_f) = \bar{f(X_f)} = 0$.

\textbf{Zu (b).} Dies zeigt auch (b). \qed

\subsection{Bemerkung} Man kann zeigen, dass in der Situation von \ref{fixed:4.2.6} der Körper $L$ automatisch algebraisch abgeschlossen ist \cite[A 3.7.11]{Bosch2004} \cite[A 8.8]{Lorenz2007}. Dies ist für uns aber noch zu schwierig, weshalb wir den Trick anwenden werden, das Lemma zu iterieren, um die Existenz eines algebraischen Abschlusses im folgenden Sinn zu zeigen:

\subsection{Definition} \ALref{\ref{fixed:4.1.19}} Sei $L|K$ eine algebraische Körpererweiterung und $L$ algebraisch abgeschlossen. Dann heißt $L$ ein algebraischer Abschluss\index{algebraischer Abschluss} von $K$.

\subsection{Satz} [Ernst Steinitz, geb.~1871, gest.~1928] Jeder Körper besitzt einen algebraischen Abschluss.

\proof Sei $K$ ein Körper. Nach \ref{fixed:4.2.6} gibt es eine Folge $(K_n)_{n \in \N}$ von Körpern derart, dass $K_0 = K$ und für jedes $n \in \N_0$ $K_{n+1}|K_n$ eine algebraische Körpererweiterung ist mit der Eigenschaft, dass jedes Polynom aus $K_n[X]|K_n$ in $K_{n+1}$ eine Nullstelle hat. Definiere einen Körper $L$ durch $L := \bigcup \{K_n \mid n \in \N\}$ und $A +_L b = a +_{K_n} b$ sowie $a \cdot_L b = a \cdot_{K_n} b$ für alle $a,b \in L$ und $n \in \N$ mit $a,b \in K_n$.

Es ist $L$ offensichtlich ein algebraischer Oberkörper von $K$ (denn jedes $K_n$ ist es nach \ref{fixed:4.1.18}). Schließlich ist $L$ algebraisch abgeschlossen. Ist nämlich $f \in L[X] \setminus L$, so gibt es $n \in \N_0$ mit $f \in K_n[X] \setminus K_n$ und $f$ hat in $K_{n+1} \subseteq L$ eine Nullstelle. \qed

\subsection{Beispiel} Falls $\C$ algebraisch abgeschlossen ist (was wir später beweisen werden), so ist $\C$ ein algebraischer Abschluss von $\R$ und $\barsmash \Q\C$ \ALref{\ref{4.1.10}} ein algebraischer Abschluss von $\Q$.
\subsection{Lemma} Seien $L|K$ und $L'|K'$ eine Körpererweiterung, $\phi : K \to K'$ ein Isomorphismus, $a \in L$ und $b \in L'$. Bezeichne $\tilde \phi : K[X] \to K'[X]$ den Isomorphismus mit $\tilde \phi|_K = \phi$ und $\tilde \phi(X) = X$. Dann sind äquivalent:\footnote{Hier könnte man noch die Grafiken einfügen.}

\begin{enumerate}[label=(\alph*)]
	\item
		Es gibt einen Isomorphismus $\psi : K(a) \to K'(b)$ mit $\psi|_K = \phi$ und $\psi(a) = b$.
	
	\item
		\textit{Entweder} ist sowohl $a$ algebraisch über $K$ als auch $b$ über $K'$ mit $\tilde \phi (\irr_K(a)) = \irr_{K'}(b)$\footnote{Korrektur: Stand anders in der Vorlage.} \textit{oder} weder $a$ ist algebraisch über $K$ noch $b$ über $K'$.
\end{enumerate}

\proof ~

\underline{(a) $\implies$ (b)} Ist einfach.

\underline{(b) $\implies$ (a)} Seien zunächst weder $a$ algebraisch über $K$ noch $b$ über $K'$. Dann ist $K[a]$ (bzw. $K'[b]$) ein Polynomring über $K$ (bzw. $K'$) in der Unabhängigen $a$ (bzw. $b$). Daher findet man einen Isomorphismus $\psi_0 : K[a] \to K'[b]$ mit $\psi_0|_K = \phi$ und $\psi_0(a) = b$. Mit \ref{fixed:2.3.7} kann man $\psi_0$ zu einem Isomorphismus $\psi : K[a] \to K'(b)$ erweitern.

Seien nun sowohl $a$ algebraisch über $K$ als auch $b$ über $K'$ und es gelte $\tilde \phi(\irr_K(a)) = \irr_{K'}(b)$. Wähle nun $\psi$ so, dass das folgende Diagramm kommutiert.

\begin{figure}[h]
	\centering
	\begin{tikzpicture}
		\matrix (m) [
			matrix of math nodes,
			row sep=4em,
			column sep=6em
		] {
			K & K' \\
			K[X] & K'[X] \\
			K[X]/(\irr_K(a)) & K'[X]/(\irr_K(b)) \\
			K[a] & K'[b] \\
			K(a) & K(b) \\
		};
		% Vertikal, linke Spalte
		\path
		(m-1-1) edge[dotted] node [below, rotate=270] {$\subset$} (m-2-1);
		\path
		(m-2-1) edge[->>] node {} (m-3-1);
		\path
		(m-3-1) edge[->] node [above, rotate=90] {$\eqtilde$} (m-4-1);
		\path
		(m-4-1) edge[dotted] node [below, rotate=270] {$=$} (m-5-1);
		% Vertikal, rechte Spalte
		\path
		(m-1-2) edge[dotted] node [above, rotate=270] {$\subset$} (m-2-2);
		\path
		(m-2-2) edge[->>] node {} (m-3-2);
		\path
		(m-3-2) edge[->] node [above, rotate=270] {$\eqtilde$} (m-4-2);
		\path
		(m-4-2) edge[dotted] node [above, rotate=270] {$=$} (m-5-2);
		% Horizontal
		\path
		(m-1-1) edge[->] node [below] {$\phi$} node [above] {$\eqtilde$} (m-1-2);
		\path
		(m-2-1) edge[->] node [below] {$\tilde \phi$, $X \to X$} node [above] {$\eqtilde$} (m-2-2);
		\path
		(m-3-1) edge[->] node [below] {$\Phi$ \tiny aus \ref{fixed:2.1.17}} node [above] {$\eqtilde$} (m-3-2);
		\path
		(m-4-1) edge[dashed, ->] node [below] {$\psi$} node [above] {$\eqtilde$}(m-4-2);
		% Horizontal klein (=Diagonal)
		\path
		(m-3-1) edge[draw=none] node {\tiny $\longleftarrow$ \ref{fixed:4.1.10} $\longrightarrow$ } (m-4-2);
		\path
		(m-4-1) edge[draw=none] node {\tiny $\longleftarrow$ \ref{fixed:4.1.10} $\longrightarrow$ } (m-5-2);
	\end{tikzpicture}
\end{figure}
\qed

\subsection{Definition} Seien $L|K$ und $L'|K$ Körpererweiterungen. Ein $K$-Homomorphismus (oder Homomorphismus über $K$) \index{Homomorphismus!über Körpererweiterungen} von $L$ nach $L'$ ist ein Homomorphismus $\phi : L \to L'$ mit $\phi|_K = \id_K$. Ein $K$-Isomorphismus (oder Isomorphismus über $K$) \index{Isomorphismus!über Körpererweiterungen} ist ein surjektiver (und damit bijektiver, siehe \ref{fixed:2.3.14b} $K$-Homomorphismus. Man nennt $L$ und $L'$ $K$-isomorph (oder isomorph über $K$) \index{isomorph!über Körpererweiterung}, in Zeichen $L \equiv_K L'$, wenn es einen $K$-Isomorphismus $L \to L'$ gibt.

\subsection{Proposition} Seien $L|K$ und $L'|K$ Körpererweiterungen und $\phi : L \to L'$ ein Körperhomomorphismus. Dann ist $\phi$ ein $K$-Homomorphismus genau dann, wenn $\phi$ ein $K$-Vektorraumhomomorphismus ist.

\proof Es gilt:
\begin{align*}
	\phi|K = \id_K
	& \iff \forall a\in K : \phi(a) = a \\
	& \iff \forall a\in K : \forall b\in L : \phi(a)\phi(b) = a \phi(b) \\
	& \iff \forall a\in K : \forall b\in L : \phi(ab) = a \phi(b)
\end{align*}
\qed

\subsection{Korollar} Seien $L|K$ und $L'|K$ Körpererweiterungen, $a \in L$ und $b \in L'$. Dann sind äquivalent:
\begin{enumerate}[label=(\alph*)]
	\item
		Es gibt einen $K$-Isomorphismus $\psi: K(a) \to K(b)$ mit $\psi(a) = b$.
		
	\item
		\textit{Entweder} sind $a$ und $b$ beide algebraisch über $K$ mit demselben Minimalpolynom \textit{oder} weder $a$ noch $b$ sind algebraisch über $K$.
\end{enumerate}

\subsection{Satz} Sei $L|K$ eine Körpererweiterung, $C$ ein algebraisch abgeschlossener Körper und $\phi: K \to C$ ein Homomorphismus. Dann gibt es einen Homomorphismus $\psi: L \to C$ mit $\psi|_K = \phi$.

\begin{figure}[h]
	\centering
	\begin{tikzpicture}
	\matrix (m) [
		matrix of math nodes,
		row sep=3em,
		column sep=3em
	]{
	  & C \\
	L &   \\ 
	  & K \\
	};
	\path (m-2-1) edge [dashed,->] (m-1-2);
	\path (m-1-2) edge [<-] (m-3-2);
	\path (m-3-2) edge (m-2-1);
	\end{tikzpicture}
\end{figure}

\proof Auf $M := \{(F,\alpha) \mid \text{$F$ Zwischenkörper von $L|K$, $\alpha : F \to C$ Homomorphismus}\}$ definieren wir eine Halbordnung $\preceq$ durch $(F,\alpha) \preceq (F',\alpha') :\iff (F \subseteq F'~\&~\alpha'|_F = \alpha)$. Sei $K$ eine Kette in $M$. Ist $K = \emptyset$, so ist $(K, \phi)$ eine obere Schranke von $K$ in $(M,\preceq)$. Ist $K \neq \emptyset$, so sieht man leicht, dass  $(G,\beta)$, definiert durch $G := \bigcup\{F \mid \exists~\alpha : (F,\alpha) \in K\}$ und $\beta : G \to C,~a \mapsto \alpha(a)$ für $(F,\alpha) \in K$ mit $a \in F$, eine obere Schranke $(G,\beta)$ von $K$ in $(M,\preceq)$ definiert.

Inesgesamt beseitzt also in $(M,\preceq)$ jede Kette eine obere Schranke. Nach dem Lemma von Zorn besitzt $(M,\preceq)$ ein maximales Element $(H,\gamma)$. Es günugt, $H=L$ zu zeigen. Sei hierzu $a \in L$. Zu zeigen, dass $a \in H$. Bezeichne $\tilde \gamma : H[X] \to (\gamma(H))[X]$ den Homomorphismus mir $\tilde \gamma|_H = \gamma$ und $\tilde \gamma(X)=X$. Da $\tilde \gamma$ eine Isomorphismus ist, ist mit $p := \irr_H(a)$ auch $q := \tilde \gamma(\irr_K(a)) \in (\gamma(H))[X]$ irreduzibel und normiert. Da $L$ algebraisch abgeschlossen ist, können wir $b \in C$ mit $q(b) = 0$ wählen. Nach \ref{fixed:4.2.12} gibt es also einen Homomorphimus $\delta:H(a) \to C$ mit $\delta|_H = \gamma$ und $\delta(a) = b$. Insbesondere $(H(a),\delta) \in M$ und $(H,\gamma) \preceq (H(a), \delta)$. Aus der Maximalität von $(H,\gamma)$ folgt $(H,\gamma) = (H(a),\delta)$, inesbesondere $H = H(a)$, das heißt $a \in H$, wie gewünscht. \qed

\subsection{Korollar} Seien $L|K$ und $C|K$ Körpererweiterungen. Sei $L|K$ algebraisch und $C$ algebraisch abgeschlossen. Dann gibt es einen $K$-Homomorphismus $\phi:L \to C$, das heißt, $L$ ist $K$-isomorph zu einem Zwischenkörper von $L|K$.

\subsection{Satz} [Ernst Steinitz] Je zwei algebraische Abschlüsse eines Körper $K$ sind zueinander $K$-isomorph. \index{Steinitz, Ernst}

\proof Seien $L$ und $L'$ algebraische Abschlüsse von $K$. Dann ist $L$ $K$-isomorph zu einem Zwischenkörper $F$ von $L'|K$ nach \ref{fixed:4.2.17}. Mit $L$ ist auch $F$ algebraisch abgeschlossen. Da $L'|F$ algebraisch ist, folgt also aus \ref{fixed:4.2.6d}, dass $L'=F$. \qed

\subsection{Sprechweise und Notation} Sei $K$ ein Körper. Da nach \ref{fixed:4.2.10} der algebraische Abschluss von $K$ existiert und er nach \ref{fixed:4.2.18} bis auf $K$-Isomorphie eindeutig ist, spricht man auch von \textit{dem} algebraischen Abschluss\index{algebraischer Abschluss} $\bar K$ von $K$. Die algebraischen Overkörper von $K$ sind bis auf $K$-Isomorphie nach \ref{fixed:4.2.17} genau die Zwischenkörper von $\bar K|K$.

\section{Zerfällungskörper}

\subsection{Sprechweise} Sei $K$ ein kommutativer Ring mit $0 \neq 1$, zum Beispiel ein Körper. Man sagt dann oft "`über $K$"', statt "`in $K[X]$"'. Beispiele: "`Sei $f$ ein Polynom über $K$"', statt: "`Sei $f \in K[X]$."' -- "`$f$ zerfällt über $K$"', statt: "`f zerfällt in $K[X]$."' -- "`$f$ ist irreduzibel über $K$"', statt: "`$f$ ist irreduzibel in $K[X]$."'

\subsection{Definition} Sei $L|K$ eine Körpererweiterung und $A \subseteq K[X] \setminus \{0\}$. Dann heißt $L$ ein Zerfällungskörper\index{Zerfällungskörper} von $A$ über $K$, wenn jedes Polynom aus $A$ über $L$ zerfällt und $L = K(\{a \in L \mid \exists~f \in A : f(a)=0\})$.\footnote{Korrektur: Konnte Klammerung hier nicht richtig lesen.}

\subsection{Bemerkung} Ist $L|K$ eine Körpererweiterung und $E \subseteq \barsmash KL$, so:
\begin{align*}
	K(E)
	\overset{\ref{fixed:2.3.11}}{\underset{\ref{fixed:2.2.2}}{=}} & \bigcup\{K(a_1,...,a_n) \mid n \in \N_0,~a_i \in E\} \\
	\overset{\ref{fixed:4.1.14}}{=} & \bigcup\{K[a_1,...,a_n] \mid n \in \N_0,~a_i \in E\} \\
	\overset{\hphantom{\ref{fixed:4.1.14}}}{=} & K[E]
\end{align*}
Insbesondere kann man in \ref{fixed:4.3.2} Ring-, statt Körperadjunktion verwenden.

\subsection{Definition und Proposition} Sei $L|K$ eine Körpererweiterung und $f \in K[X] \setminus \{0\}$. Dann heißt $L$ ein Zerfällungskörper\index{Zerfällungskörper} von $f$ über $K$, falls $L$ ein Zerfällungskörper von $\{f\}$ über $K$ ist. Genau dann ist also $L$ ein Zerfällungskörper von $f$ über $K$, wenn es $c \in K^\times$, $n \in \N_0$ und $a_1,...,a_n$ gibt, mit $f = c \prod_{i=1}^n (X-a_i)$ und $L = K(a_1,...,a_n)$ (oder $L=K[a_1,...,a_n]$).

\subsection{Beispiel}

\begin{enumerate}[label=(\alph*)]
	\item
		$\C$ ist ein Zerfällungskörper von $X^2+1$ über $\R$.
		
	\item
		$\Q(\sqrt 2)$ ist ein Zerfällungskörper von $X^2-2$ über $\Q$.
		
	\item
		$\Q(e^ {\frac{2 \pi i}{6}})$ ist ein Zerfällungskörper von $X^ 6-1$ über $\Q$.
		
	\item
		$\Q\left(\sqrt[3]{2}, e^ {\frac{2 \pi \i}{3}}\right)$ ist ein Zerfällungskörper von $X^3 -2$, denn
		$$X^ 3-2 = \left(X-\sqrt[3]{2}\right)\left(X-\sqrt[3]{2} \cdot e^{\frac{2 \pi \i}{3}}\right) \left(X-\sqrt[3]{2} \cdot e^{\frac{4 \pi \i}{3}}\right)$$
		und $\Q\left(\sqrt[3]{2}, \sqrt[3]{2} \cdot e^{\frac{2 \pi \i}{3}}, \sqrt[3]{2} \cdot e^{\frac{4 \pi \i}{3}}\right) = \Q\left(\sqrt[3]{2}, e^{\frac{2 \pi \i}{3}}\right)$.
\end{enumerate}

\subsection{Bemerkung} Sei $L|K$ eine Körpererweiterung und $A \subseteq K[X] \setminus \{0\}$.
\begin{enumerate}[label=(\alph*)]
	\item
		Jeder Zerfällungskörper $L$ von $A$ über $K$ ist offensichtlich algebraisch über $K$, denn er entsteht aus $K$ durch Ajunktion von über $K$ algebraischen Elementen und ist damit nach \ref{fixed:4.1.19} ist $\barsmash KL$ enthalten und damit gleich $\barsmash KL$. Ist zusätzlich $A$ endlich, ist ist nach \ref{fixed:4.1.16} $L|K$ sogar endlich.
		
	\item
		Zerfällt jedes Polynom aus $A$ über $L$, so gibt es offensichtlich genau einen Zwischenkörper $F$ von $L|K$, der ein Zerfällungskörper von $A$ über $K$ ist, nämlich $F = K(\{a \in L \mid \exists~f \in A : f(a)=0\})$.
\end{enumerate}

\subsection{Satz} Sei $K$ ein Körper und $A \subseteq K[X] \setminus \{0\}$. Dann gibt es bis aus $K$-Isomorphie geanu einen Zerfällungskörper von $A$ über $K$.

\proof ~

\textit{Existenz:} Nehme $K(\{a \in \bar K \mid \exists~f \in A : f(a) = 0\})$ im nach \ref{fixed:4.2.10} existierenden algebraischen Abschluss $\bar K$ von $K$.

\textit{Eindeutigkeit:} Seien $L$ und $L'$ Zerfällungskörper von $A$ über $K$. Zu zeigen ist $L \eqtilde_K L'$. Da $L$ und $L'$ über $K$ algebraisch sind, sind $\bar L$ und $\bar L'$ nach \ref{fixed:4.1.17} algebraische Abschlüsse von $K$ und daher nach \ref{fixed:4.2.18} $K$-isomorph. Wähle einen $K$-Isomorphismus $\phi : \bar L \to \bar L'$. Dann sind $\phi(L)$ und $L'$ beides Zwischenkörper $\bar L'|K$, die ein Zerfällungskörper von $A$ über $K$ sind. Nach \ref{fixed:4.3.6b} gilt $\phi(L) = L'$, weshalb $\phi$ einen $K$-Isomorphismus $L \to L'$ induziert. \qed

\subsection{Definition} Sei $L|K$ eine Körpererweiterung. Ein Automorphismus\index{Automorphismus!über Körpererweiterungen} von $L|K$ (oder ein $K$-Automorphismus von $L$ über $K$) ist ein $K$-Isomorphismus von $L$ nach $L$ \ALref{\ref{fixed:4.2.13}}.

Es bezeichne $\Aut(L|K) := \{ \phi \mid \text{$\phi$ ist Automorphismus von $L|K$} \}$ die Gruppe aller Automorphismen von $L|K$.

\subsection{Definition} Sei $K$ ein Körper. Betrachte die natürliche Wirkung von $\Aut(\bar K|K)$ auf $\bar K$ und die dazugehörige Äquivalenzrelation $\sim_K$ auf $\bar K$, definiert durch $a \sim_K b :\iff \exists~\phi \in \Aut(\bar K|K) : \phi(a)=??$\footnote{Korrektur: Kann ich nicht lesen.} ($a,b \in \bar K$). Für $a,b \in \bar K$ nennt man $a$ und $b$ über $K$ zueinander konjugiert\index{konjugiert}, wenn $a \sim_K b$.

\subsection{Proposition} Sei $L|K$ eine algebraische Körpererweiterung und $\phi:L \to L$ ein $K$-Homomorphismus. Dann ist $\phi \in \Aut(L|K)$.

\proof Nach \ref{fixed:2.3.14b} ist $\phi$ injektiv. Also ist noch zu zeigen, dass $\phi$ surjektiv ist. Sei $b \in L$ und zeige also $\exists~a \in L : \phi(a) =b$. Wähle $p \in K[X] \setminus \{0\}$ mit $p(b) = 0$. Für die endliche Menge $A := \{a \in L \mid p(a) =0 \}$ gilt dann $\phi(A) \subseteq A$ und daher $\phi(A) =A$. Wegen $b \in A$ gibt es also $a \in A \subseteq L$ mit $\phi(a) = b$.

\subsection{Proposition} Sei $K$ ein Körper und $a,b \in \bar K$. Dann gilt $a \sim_K b \iff \irr_K(a) = \irr_K(b)$.

\proof "`$\Longrightarrow$"': Klar

"`$\Longleftarrow$"': Nach \ref{fixed:4.2.15} gibt es einen $K$-Homomorphismus $\phi : K(a) \to \bar K$ mit $\phi(a) =b$, den wir nach \ref{fixed:4.2.16} fortsetzen zu einem $K$-Homomorphismus $\psi:\bar K \to \bar K$. Nach \ref{fixed:4.3.10} gilt $\psi \in \Aut(L|K)$.

\begin{figure}[h]
	\centering
	\begin{tikzpicture}
	\matrix (m) [
		matrix of math nodes,
		row sep=2em,
		column sep=2em
	]{
	\bar K & \bar K \\
	K(a) & \\ 
	K & \\
	};
	\path (m-1-1) edge [->] node [above] {$\psi$} (m-1-2);
	\path (m-2-1) edge [->] node [below] {$\phi$} (m-1-2);
	\path (m-1-1) edge (m-2-1);
	\path (m-2-1) edge (m-3-1);
	
	\path (m-1-1) edge [dotted, bend right = 50] node [left] {\footnotesize alg.} (m-3-1);
	\end{tikzpicture}
\end{figure}
\qed

\subsection{Definition} Eine Körpererweiterung $L|K$ heißt normal\index{Körpererweiterung!normale}, wenn $L$ ein Zerfällungskörper einer Menge $A \subseteq K[X] \setminus \{0\}$\footnote{Korrektur: Konnte ich nicht lesen.} über $K$ ist.

\subsection{Beispiel} Jede Körpererweiterung $L|K$ vom Grad $2$ ist normal. Wählt man nämlich $a \in L \setminus K$, so ist $L = K(a)$ und $L$ der Zerfällungskörper von $\irr_K(a)$ über $K$, denn $\deg \irr_K(a)=2$.

\subsection{Satz} Sei $L|K$ eine algebraische Körpererweiterung. Dann sind äquivalent:
\begin{enumerate}[label=(\alph*)]
	\item
		$L|K$ ist normal.
		
	\item
		Jedes irreduzible Polynom aus $K[X]$ mit einer Nullstelle in $L$ zerfällt über $L$.
		
	\item
		$L$ ist Vereinigung von Äquivalenzklassen von $\sim_K$.
		
	\item
		Für jeden $K$-Homomorphismus $\phi: L \to \bar L$ gilt $\phi(L) = L$.
		
	\item
		$\forall~\phi \in \Aut(\bar L|K) : \phi(L)=L$
\end{enumerate}

\proof ~

\underline{(a) $\implies$ (d)} Sei $L$ Zerfällungskörper von $A \subseteq K[X] \setminus \{0\}$ und $\phi : L \to \bar L$. ein $K$-Homomorphismus. Mit $L$  ist auch der dazu $K$-isomorphe Körper $\phi(L)$ ein Zerfällungskörper von $A$ über $K$. Da beide Zwischenkörper von $\bar L|K$ sind, folgt aber dann $\phi(L) = L$ nach \ref{fixed:4.3.6b}.

\underline{(d) $\implies$ (e)} Klar.

\underline{(e) $\implies$ (c)} Gelte (e). Wir zeigen $L = \bigcup \left\{ \tildeidx aK \mid a \in \bar L,~\tildeidx aK \cap L \neq \emptyset \right\}$.
\begin{itemize}
	\item["`$\subseteq$"':]
		Sei $a \in L$. Dann ist $a \in \tildeidx aK \cap L$, also $\tildeidx aK \cap L \neq \emptyset$ und $a \in \tildeidx aK$.
		
	\item["`$\supseteq$"':]
		Sei $a \in \bar L$ mit $\tildeidx aK \cap L \neq \emptyset$. Zu zeigen ist $\tildeidx aK \subseteq L$. Sei ohne Einschränkung $a \in L$. Sei $b \in \tildeidx aK$. Zu zeigen ist $b \in L$. Wegen $a \sim_K b$ gibt es $\phi \in \Aut(\bar L|K)$ mit $b = \phi(a) \in \phi(L) = L$.
\end{itemize}

\underline{(c) $\implies$ (b)} Gelte (c)und sei $p \in K[X]$ irreduzibel mit einer Nullstelle in $L$. Da nach \ref{fixed:4.3.11} alle Nullstellen von $p$ in $\bar L$ zueinander konjugiert sind, liegen diese alle in $L$ wegen (c).

\underline{(b) $\implies$ (a)} Gelte (b) und setze $A := \{p \in K[X] \mid p \text{ irreduzibel},~\exists~a \in L : p(a) = 0\}$. Nach (b) zerfällt jedoch jedes Polynom aus $A$ über $L$. Da $L|K$ algebraisch ist, gilt $E := \{a \in L \mid \exists~p \in A : p(a)=0\} =L$, denn jedes Element von $L$ ist Nullstelle eines Minimalpolynoms über $K$ und daher natürlich $L = k(E)$.\footnote{Korrektur: Kann ich nicht lesen / ist mir nicht klar.}

\subsection{Beispiel}
\begin{enumerate}[label=(\alph*)]
	\item
		Nach dem Kriterium von Eisenstein \ref{fixed:2.6.3} sind $X^4-2$ und $X^2-2$ irreduzibel in $\Q[X]$ (und in $\Z[X]$). Daraus folgt $\irr_\Q(\sqrt[4]{2}) = X^4-2$ und $\irr_\Q(\sqrt 2) = X^2-2$, also $[\Q(\sqrt[4]{2}) : \Q] = 4$ und $[\Q(\sqrt 2) : \Q] = 2$. Somit sind $\Q(\sqrt[4]{2})|\Q(\sqrt 2)$ und $\Q(\sqrt2)|\Q$ beides Körpererweiterungen vom Grad $2$ und daher normal nach \ref{fixed:4.3.13}.
		
		Aber $\Q(\sqrt[4]{2})|\Q$ ist nicht normal, da das irreduzible Polynom $X^4-2 \in \Q[X]$ über $\Q(\sqrt[4]{2})$ nicht zerfällt, obwohl es eine Nullstelle hat. In der Tat: $\i\sqrt[4]{2}$ ist eine Nullstelle dieses Polynoms, welche nicht in $\Q(\sqrt[4]{2})$, ja nicht einmal in $\R$, liegt.
		
	\item
		Für jeden Körper $K$ ist $\bar K$ über $K$ normal.
\end{enumerate}

\section{Endliche Körper}

\subsection{Definition} Ist $R$ ein Ring, so heißt die eindeutig bestimmte Zahl $n \in \N_0$, welche den Kern des eindeutig bestimmten Ringhomomorphismus' $\Z \to R$ als Ideal erzeugt, die Charakterisitk\index{Charakteristik} von $R$, in Zeichen $\chara R$.\footnote{$\phi: \Z \to R,~...,~-1 \mapsto -1,~0 \mapsto 0,~1 \mapsto 1,~2 \mapsto 2,~...$}

\subsection{Bemerkung}
\begin{enumerate}[label=(\alph*)]
	\item
		Ist $R$ ein Ring, so gibt es genau einen Homomorphismus $\Z/(\chara R) \to R$. Dieser ist eine Einbettung und sein Bild ist der kleinste Unterring von $R$.
	
	\item
		ist $R$ ein Integritätsring, so gilt $\chara R \in \{0\} \cup \P$.
		
	\item
		Ist $K$ ein Körper und $p := \chara K$, so hat man im Fall $p=0$ ($p \in \P$) genau einen Homomorphismus $\Q \to K$ \ALref{\ref{fixed:2.3.7}} ($\F_p = \Z/(p) \to K$)\index{F_p@$\F_p$}. Dessen Bild ist der kleinste Unterkörper von $K$, welchen man auch Primkörper von $K$ nennt. Jeder Körper enthält also einen zu $\Q$ oder $\F_p$ ($p \in \P$) isomorphen Unterkörper.
\end{enumerate}

\subsection{Proposition} Sei $R$ ein kommutativer Ring mit $p := \chara R \in \P$. Dann ist der Frobenius-Endomorphismus [Ferdinand Georg Frobenius, geb. 1849, gest. 1917] $\Phi_R : R \to R,~a \mapsto a^p$ ein Endomorphismus.

\proof Strittig könnte nur sein, ob $(a+b)^ p = a^p + b^p$ für alle $a,b \in R$ gilt. Durch Ausmultiplizieren und Zusammenfassen der linken Seite erhält man $(a+b)^p = \sum_{k=0}^p \binom{p}{k} a^k b^{p-k}$, wobei $\binom{p}{k}$ das Bild des Binomialkoeffizienten\index{Binomialkoeffizient} $\binom{p}{k} = \frac{p!}{k!(p-k)!}$ unter $\Z \to R$ bezeichnet. Für $k \in \{1,...,p-1\}$ ist $p$ kein Teiler von $k!(p-k)!$, aber $k!(p-k)!$ ein Teiler von $p!$ und damit von $(p-1)!$. Es folgt, dass $\binom{p}{k} = p \frac{(p-1)!}{k!(p-k)!} \in (p)$ und daher $\binom{p}{k} = 0$ in $R$ für $k \in \{1,...,p-1\}$. \qed

\subsection{Definition} Sei $K$ ein Körper, $f \in K[X]$ und $a \in K$. Dann heißt $\mu(a,f) := \sup \left\{n \in \N_0 \mid (X-a)^n \text{ teilt $f$ in $K[X]$}\right\} \in \N_0 \cup \{\infty\}$ die Vielfachheit\index{Vielfachheit} von $a$ in $f$.

\subsection{Bemerkung} Sei $K$ ein Körper, $f \in K[X]$ und $a \in K$.
\begin{enumerate}[label=(\alph*)]
	\item
		$\mu(a,f) = \infty \iff f=0$
		
	\item
		$\mu(a,f) \geq 1 \iff f(a) =0$
		
	\item
		Die Definition stimmt überein mit der in \LAref{10.1.13} gegebenen Definition der Vielfachheit einer Nullstelle $a \in K$ eines Polynoms $f \in K[X]\setminus\{0\}$.
		
	\item
		$\mu(a,f) = v_{X-a}(f)$, wobei $v_{X-a}$ die in \ref{fixed:2.5.3} definierte $(X-a)$-Bewertung auf $K(X) = \qf(K[X])$ bezeichne.
\end{enumerate}

\subsection{Konvention} Ist $R$ ein Ring und $n \in \Z$, so schreibt man oft $n$ und meint damit das Bild von $n$ unter dem eindeutig bestimmten Ringhomomorphismus $Z \to R$.

\subsection{Definition} Sei $K$ ein Körper. Dan durch $1'=0$ und $(X^n)' = nX^{n-1}$ für $n \in \N$ gegebenen $K$-Vektorraumhomomorphismus $K[X] \to K[X],~f \mapsto f'$ nennt\footnote{Korrektur: Kann ich nicht lesen.} man formale Ableitung\index{Ableitung!formale} \LAref{6.3.2 \textit{f}}.

\subsection{Proposition} Sei $K$ ein Körper. Für alle $f,g \in K[X]$ gilt:
\begin{enumerate}[label=(\alph*)]
	\item
		$(fg)' = f'g + fg'$ ("`Produktregel"')\index{Produktregel}
		
	\item
		$(f(g))' = (f'(g))g'$ ("`Kettenregel"')\index{Kettenregel}
\end{enumerate}

\proof ~

\underline{zu (a):} Die Abbildung $b : K[X] \to K[X]$, $(f,g) \mapsto (fg)'-f'g-fg'$ ist bilinear. Daher reicht es zu zeigen, dass $b(X^m,X^n) = 0$ für alle $m,n \in \N$. Dies ist klar für $m=0$ oder $n=0$.\footnote{Korrektur: Kann ich nicht lesen.} Seien also $m,n \in \N$. Dann ist $b(X^m,X^n) = (m+n)X^{m+n-1} - mX^{m-1}X^n - X^mnX^{n-1} = 0$.

\underline{zu(b):} Er reicht für $n \in \N$ zu zeigen, dass für alle $g \in K[X]$ gilt: $(g^n)' = (ng^{n-1})g'$, was wir durch Induktion nach $n \in \N$ machen: $n = 1$ ist klar, also $n \to n+1~(n \in \N)$: Sei $g \in K[X]$. Dann $(g^{n+1})' = (gg^n)' = g'g^n + g(g^n)' = g'g^n + gng^{n-1}g' = (n+1)g^ng'$. \qed

\subsection{Proposition} Sei $K$ ein Körper, $p := \chara K$, $f \in K[X] \setminus \{0\}$ und $a \in K$. Dann gilt
$$p \nmid \mu(a,f) \implies \mu(a,f') = \mu(a,f') = \mu(a,f)-1$$
$$p \mid \mu(a,f) \implies \mu(a,f') = \mu(a,f') \geq \mu(a,f)$$
[Beachte, dass für $p =0$ gilt: $p \nmid \mu(a,f) \iff \mu(a,f) \geq 1 \iff f(a)=0$ und $p \mid \mu(a,f) \iff \mu(a,f)=0 \iff f(a) \neq 0$.]

\proof Setze $n := \mu(a,f)$ und schreibe $f = (X-a)^n g$ mit $g \in K[X]$. Dann gilt $g(a) \neq 0$. Ist $n=0$, so $p \mid n$ und es ist nichts zu zeigen. Sei also $n>0$. Dann:
$$f' = (X-a)^n g' + n(X-a)^{n-1} g = (X-a)^{n-1} \underbrace{((X-a) g' + ng)}_{=:h}$$
Gilt $p \mid n$, so $h = (X-a)g'$ und $f' = (X-a)^ng'$. Gilt $p \nmid n$, so $h(a) = ng(a) \neq 0$. \qed

\subsection{Definition} Sei $K$ ein Körper, $f \in K[X]$ und $a \in K$. Dann heißt $a$ eine mehrfache Nullstelle\index{Nullstelle!mehrfache} von $f$, wenn $\mu(a,f) \geq 2$.

\subsection{Proposition} Sei $K$ ein Körper, $f \in K[X]$ und $a \in K$. Dann ist $a$ eine mehrfache Nullstelle von $f$ genau dann, wenn $f(a) = f'(a) = 0$.

\proof Gilt $\mu(a,f) \geq 2$, so $\mu(a,f') \geq 1$ nach \ref{fixed:4.4.9}. Gilt umgekehrt $f(a) = f'(a)=0$, so ist natürlich $\mu(a,f) \geq 1$. Wäre $\mu(a,f)=1$, so $\chara K \nmid \mu(a,f)$ und daher $\mu(a,f') = 0$ nach \ref{fixed:4.4.9} im Widerspruch zu $f'(a)=0$. \qed

\subsection{Beispiel} Sei $p \in \P$ und $n \in \N$. Das Polynom $X^{p^n} - X \in \F_p[X]$ hat keine mehrfachen Nullstellen im algebraischen Abschluss $\bar \F_p$ von $\F_p$, denn $(X^{p^n} - X)' = p^n X^{p^n-1}-1 = -1$.

\subsection{Bemerkung} Sei $K$ ein endlicher Körper. Dann gilt $p := \chara K \in \P$ und $K$ ist ein endlich-dimensionaler Vektorraum über seinem zu $\F_p$ isomorphen Primkörper. Es folgt $\#K = p^n$ für ein $n \in \N_{\geq 1}$.

\subsection{Satz} Sei $p \in \P$, $K|\F_p$ eine Körpererweiterung und $n \in \N$. Dann sind äquivalent:
\begin{enumerate}[label=(\alph*)]
	\item
		$\#K=p^n$
		
	\item
		$K$ ist Zerfällungskörper von $X^{p^n} - X$ über $\F_p$.
\end{enumerate}

\proof ~

\underline{(a) $\implies$ (b):} Gelte $\#K = p^n$. Dann $\# K^\times = p^n - 1$ und daher $a^{p^n-1}=1$ für alle $a \in K^\times$ nach \ref{fixed:1.3.21}. Es folgt $a^{p^n} = a$ für alle $a \in K$. Es folgt $X^{p^n}-X = \prod_{a \in K} (X-a)$. Wegen $K = \F_p(K)$\footnote{Korrektur: Stimmt das so?} folgt (b).

\underline{(b) $\implies$ (a):} Gelte (b). Setzt man $F := \{a \in K \mid a^{p^n} - a=0 \}$, so besteht $F$ genau aus den Nullstellen von $X^{p^n}-X$ in $K$, woraus mit (b) und \ref{fixed:4.4.12} folgt $\#F = p^n$. Andererseits ist $F = \{a \in K \mid \Phi_K^n(a)=a\}$ ein Zwischenkörper von $K|\F_p$, denn $\Phi_K$ und damit $\Phi_K^n$ ist ein $\F_p$-Endomorphismus von $K$. Es folgt $K = \F_p(F) = F$. \qed

\subsection{Korollar}
\begin{enumerate}[label=(\alph*)]
	\item
		Ist $m \in \N$, so gibt es genau dann einen Körper $K$ mit $\#K=m$, wenn es $p \in \P$ und $n \in \N$ mit $m = p^n$ gibt.
		
	\item
		Sind $K$ und $L$ endliche Körper, so $K \eqtilde L \iff \#K = \#L$.
\end{enumerate}

\proof ~

\underline{zu (a):} Benutze \ref{fixed:4.4.13} und die Existenz von Zerfällungskörpern aus \ref{fixed:4.3.7}.

\underline{zu (b):} Seien $K$ und $L$ endliche Körper mit $\#K = \#L$. Zu zeigen $K \eqtilde L$. Nach \ref{fixed:4.4.13} gibt es $p \in \P$ und $n \in \N$ mit $\#K = \#L = p^n$. Aus dem Satz von Lagrange \ref{fixed:1.3.19} folgt dann, dass $K$ und $L$ jeweils einen zu $\F_p$ isomorphen Primkörper besitzen.

Ohne Einschränkung sei $\F_p$ sogar gleich dem Primkörper sowohl von $K$ also auch von $L$. Nach \ref{fixed:4.4.14} sind $K$ und $L$ dann beide ein Zerfällungskörper von $X^{p^n}-X$ über $\F_p$. Mit \ref{fixed:4.3.7} folgt $K \eqtilde L$.
\subsection{Notation} Sei $p \in \P$. Fixiere einen algebraischen Abschluss $\bar \F_p$ von $\F_p$ den nach \ref{fixed:4.3.6b} und \ref{fixed:4.4.14} eindeutig bestimmten Zwischenkörper von $\bar \F_p|\F_p$ mit genau $p^n$ Elementen.\footnote{Korrektur: Macht irgendwie keinen Sinn.}

\subsection{Proposition}
\begin{enumerate}[label=(\alph*)]
	\item
		$\bar \F_p = \bigcup\{\F_{p^n} \mid n \in \N\}$
		
	\item
		$\forall~m,n \in \N : (\F_{p^m} \subseteq \F_{p^n} \iff m \mid n)$
\end{enumerate}

\proof ~

\underline{zu (a):} Sei $a \in \bar \F_p$ und setze $n := [\F_p(a) : \F_p] < \infty$. Dann ist $\#F_p(a) = p^n$ und daher $a \in \F_p(a) = \F_{p^n}$.

\underline{zu (b):} Seien $m,n \in \N$. Gilt $\F_{p^m} \subseteq \F_{p^n}$, so ist $\F_{p^n}$ ein $\F_{p^m}$-Vektorraum der Dimension $k := [\F_{p^n} : \F_{p^m}]$ und daher $p^n = (p^m)^k$, das heißt $n=mk$. Gilt umgekehrt $m \mid n$, das heißt $p^n = (p^m)^k$ für ein $k \in \N$, so ist jede Nullstelle von $X^{p^m} - X$ auch eine Nullstelle von $X^{p^n} - X$.

\subsection{Lemma} Sei $G$ eine endliche Gruppe und $a,b \in G$. Gelte $ab = ba$ und $1 \in (\ord a, \ord b)$. Dann $\ord (ab) = \ord(a) \ord(b)$.

\proof Setze $m := \ord a$ und $n := \ord b$. Zu zeigen ist $\ord(ab)=mn$. Wärte $s,t \in \Z$ mit $1 = sm+tn$. Ist $k \in \Z$ mit $(ab)^k = 1$, so gilt\footnote{Korrektur: Kann ich nicht lesen.}
$$1 = ((ab)^k)^{sm} = (a^m)^{ks}(b^{sm})^k = (b^{??})^k = b^k$$
und analog $1 = a^k$, woraus $m \mid k$ und $n \mid k$ folgt, das heißt $k \in (m) \cap (n) \overset{\ref{fixed:2.8.5}}{=} (m)(n) \overset{\ref{fixed:2.8.2}}{=} (mn)$. Schließlich $(ab)^m = (a^m)^n(b^n)^m = 1$. Somit $\ord(ab)=mn$. \qed

\subsection{Satz} Endliche Untergruppen der multiplikativen Gruppe eines Körper sind zyklisch.

\proof Sei $K$ ein Körper, $G \leq K^\times$ mit $d := \#G < \infty$ und schreibe $d = p_1^{\alpha_1} \cdots p_n^{\alpha_n}$ mit $n \in \N_0$, $p_1,...,p_n \in \P$ paarweise verschieden und $\alpha_1,...,\alpha_n \in \N$. Sei $i \in \{1,...,n\}$. Da das Polynom $X^\frac{d}{p_i} - 1$ höchstens $\frac{d}{p_i} < d$ Nullstellen hat, gibt es $a_i \in G$ mit $a_i^\frac{p}{d_i} \neq 1$. Setze $b_i := a^\frac{d}{p_i^{\alpha_i}} \in G$. Wegen $b_i^{p_i^{\alpha_i}} = a_i^d = 1$, da $\ord a_i \mid \#G=d$, gilt $\ord b_i \mid p_i^{\alpha_i}$. Setzt man schließlich $b := b_1,...,b_n$, so folgt mit \ref{fixed:4.4.18}, dass $\ord(b) = p_1^{\alpha_1} \cdots p_n^{\alpha_n} = d$, also $\langle b \rangle = G$. \qed

\subsection{Korollar} Multiplikative Gruppen endlicher Körper sind zyklisch.

\subsection{Satz} Sei $p \in \P$ und $n \in \N$. Dann gibt es ein irreduzibles Polynom vom Grad $n$ in $\F_p[X]$ und für jedes solche Polynom $f$ gilt $\F_{p^n} \eqtilde \F_p[X]/(f)$.

\proof Wähle gemäß \ref{fixed:4.4.19} ein $a \in \F_{p^n}^\times$ mit $\langle a \rangle = \F_{p^n}^\times$. Dann gilt insbesondere $\F_p(a) = \F_{p^n}$. Dann ist $f := \irr_{\F_p}(a) \in \F_p[X]$ irreduzibel vom Grad $[\F_{p^n} : \F_p] = n$. Sei nun $f \in \F_p[X]/(f)$ nach \ref{fixed:2.4.9} ein Körper. Da $\bar 1, \bar X, ..., \bar X^{n-1}$ eine Basis des $\bar F_p$-Vektorraumes $\F_p[X]/(f)$ ist, gilt $\#\F_p[X]/(f) = p^n$ und daher $\F_p[X]/(f) \eqtilde \F_{p^n}$ nach \ref{fixed:4.4.15b}. \qed

\section{Separable Körpererweiterungen}

\subsection{Definition} Sei $K$ ein Körper. Ein Polynom $f \in K[X]$ heißt separabel\index{sparabel}\index{Polynom!separables}, wenn $f$ im algebraischen Abschluss $\bar K$ von $K$ keine mehrfachen Nullstellen hat.

\subsection{Warnung} Sei $K$ ein Körper. Viele Autoren nennen ein Polynom $f \in K[X]$ auch dann separabel über $K$, wenn jeder irreduzible Teiler von $f$ in $K[X]$ in unserem Sinne separabel ist.

\subsection{Proposition} Sei $K$ ein Körper und $f \in K[X]$. Dann gilt $f$ separabel $\iff \gcd_{K[X]}(f,f')=1 \iff 1 \in (t,t')_{K[X]} \iff \gcd_{\bar K[X]} (f,f')=1$.

\proof Klar mit \ref{fixed:4.4.11}.

\subsection{Korollar} Sei $K$ ein Körper und $f \in K[X]$ irreduzibel. Dann gilt $f$ separabel $\iff f' \neq 0$.

\subsection{Korollar} Sei $K$ ein Körper der Charakteristik $p \in \P \cup \{0\}$ und $f \in K[X]$ irreduzibel. Dann gilt:
\begin{enumerate}[label=(\alph*)]
	\item
		$p = 0 \implies f$ separabel
		
	\item
		$p \in \P \implies (f \text{ separabel } \iff f \not\in K[X^p])$
		
	\item
		Es gibt ein irreduzibles separables $g \in K[X]$ und ein $n \in N_0$ mit $f = g(X^{p^n})$. Hierbei sind $g$ und $n$ durch $f$ eindeutig bestimmt und für alle $a \in \bar K$ mit $f(a)=0$ gilt $\mu(a,f) = p^n$.
\end{enumerate}

\proof (a) und (b) direkt aus \ref{fixed:4.5.4}.

(c) direkt aus (a), falls $p=0$. Dann $n=0$ und $g=f$. (c) durch iteriertes Anwenden von (b), falls $p \in \P$. (Ist $g = \prod_{i=1}^d (X-a_i)$ mit $a_i \in \bar K$, so $f = \prod_{i=1}^d (X^{p^n}-a_i) = \prod_{i=1}^d (X^{p^n} - b_i^{p^n} \overset{\ref{fixed:4.4.3}}{=} \prod_{i=1}^d (X-b_i)^{p^n})$, wobei man $b_i \in \bar K$ wählt mit $b_i^{p^n} - a_i = 0$.) \qed

\nocite{*}
\bibliographystyle{plaindin}
\bibliography{algebra_book.bib}

\printindex

\end{document}