%-------------------------Packages---------------------------------%

\documentclass[11pt, a4paper, titlepage, bibliography=totoc]{scrbook}

\usepackage[utf8]{inputenc}
\usepackage[T1]{fontenc}
\usepackage[ngerman]{babel}

\usepackage{enumitem}
\usepackage{extarrows}
\usepackage{amsmath, amssymb, amsthm}
\usepackage{stmaryrd}
\usepackage{calrsfs}
\usepackage{wasysym}

\usepackage{float}
\usepackage{graphicx}
\usepackage{mdframed}
\usepackage{tcolorbox}
\usepackage{titlesec}

\usepackage{makeidx}
\makeindex

\usepackage[hidelinks]{hyperref}

\usepackage{tikz}

%----------------------------Titlesec-----------------------------------%

\usepackage{titlesec}
\titleformat{\chapter}[block]{\normalfont\Huge\bfseries}{§~\thechapter}{.5em}{}[]
\titleformat{\section}[block]{\hrule\bigskip\normalfont\Large\bfseries}{§~\thesection}{.5em}{}[\bigskip\hrule]
\titleformat{\subsection}[runin]{\normalfont\bfseries}{\thesubsection}{.5em}{}[]

%----------------------------Definitions-------------------------------%

% Mengen
\newcommand\N{\mathbb N}
\newcommand\Z{\mathbb Z}
\newcommand\Q{\mathbb Q}
\newcommand\R{\mathbb R}
\newcommand\C{\mathbb C}
\renewcommand\P{\mathbb P}
\newcommand\p{\mathfrak p}
\newcommand\m{\mathfrak m}
\newcommand\q{\mathfrak q}
\newcommand\Odm{
	\begin{tikzpicture}[baseline]
		\draw[fill=black] (1.6ex,0) -- (1.6ex,1.6ex) -- (0,1.6ex) -- cycle;
	\end{tikzpicture}
	~
}
\newcommand\Udm{
	\begin{tikzpicture}[baseline]
		\draw[fill=black] (0,0) -- (0,1.6ex) -- (1.6ex,0) -- cycle;
	\end{tikzpicture}
	~
}
\newcommand\odm{
	\begin{tikzpicture}[baseline]
		\draw (1.6ex,0) -- (1.6ex,1.6ex) -- (0,1.6ex) -- cycle;
	\end{tikzpicture}
	~
}
\newcommand\udm{
	\begin{tikzpicture}[baseline]
		\draw (0,0) -- (0,1.6ex) -- (1.6ex,0) -- cycle;
	\end{tikzpicture}
	~
}

% Operatoren
\DeclareMathOperator\Aut{Aut}
\DeclareMathOperator\End{End}
\DeclareMathOperator\GL{GL}
\DeclareMathOperator\sgn{sgn}
\DeclareMathOperator\im{im}
\DeclareMathOperator\spn{span}
\DeclareMathOperator\irr{irr}
\DeclareMathOperator\id{id}
\DeclareMathOperator\qf{qf}
\DeclareMathOperator\supp{supp}

% Relationen
\newcommand{\eqhat}{\mathrel{\widehat{=}}}
\newcommand{\eqtilde}{\mathrel{\widetilde{=}}}

% Befehle zu Gruppen
\newcommand\GRPeqn{\stackrel{(N)}{=}}
\newcommand\GRPeqi{\stackrel{(I)}{=}}
\newcommand\GRPeqa{\stackrel{(A)}{=}}

% Befehle zu Abbildungen
\newcommand\MAPmono{\lhook\joinrel\rightarrow}
\newcommand\MAPepi{\twoheadrightarrow}
\newcommand\MAPiso{\overset{\cong}{\rightarrow}}
\newcommand\MAPlongmono{\lhook\joinrel\longrightarrow}
\newcommand\MAPlongepi{\relbar\joinrel\twoheadrightarrow}
\newcommand\MAPlongiso{\overset{\cong}{\longrightarrow}}

% Verweise
\newcommand\LAref[1]{[$\to$ LA #1]}
\newcommand\ALref[1]{[$\to$ #1]}

% Beweise
\newcommand\bew{\textbf{Beweis:}~}

% Vereinfachungen
\newcommand\mo{{-1}}
\newcommand\ntn{{n \times n}}

% Verschiebe overline nach oldoverline.
% Fordere für overline mindestens Höhe eines 'A' Abstand zur Grundlinie.
\let\oldoverline\overline
\renewcommand\overline[1]{\oldoverline{\vphantom{A} #1}}

% Benutze immer overline.
\renewcommand\bar\overline
% Benutze immer widehat.
\renewcommand\hat\widehat
% Benutze immer varphi statt phi und varepslion statt epsilon.
\renewcommand\phi\varphi
\renewcommand\epsilon\varepsilon

% Sonstige
\setlength{\parindent}{0pt}
\newcommand{\HRule}{\rule{\linewidth}{0.5mm}}
\renewcommand\i{\mathring\imath}
\newcommand\numberfix[3]{
	\setcounter{chapter}{#1}
	\setcounter{section}{#2}
	\setcounter{subsection}{#3}
	
	\rule{\textwidth}{1pt}
	\texttt{Hier fehlt noch etwas ...}\\
	\rule{\textwidth}{1pt}
}

%---------------------------Document--------------------------------------%

\begin{document}

\KOMAoptions{twoside = false}
\begin{titlepage}
	\begin{center}
		\textsc{\LARGE Universität Konstanz}\\[1.5cm]
		{\large Skriptum zur Vorlesung}\\[0.5cm]
		\HRule \\[0.4cm]
		{\huge \bfseries Einführung in die Algebra \\[0.4cm]}
		\HRule \\[.5cm]
		{\small Private Mitschrift}\\[1.5cm]
		{\large \emph{gelesen von:} \hfill Prof. Dr. Markus Schweighofer}\\
		\large\vfill
		{\large Wintersemester 2014/15}\\
		{\large Stand vom \today}	
	\end{center}
\end{titlepage}
\KOMAoptions{twoside = true}

\setcounter{tocdepth}{1}
\tableofcontents

\chapter{Gruppen}

\section{Gruppen und Untergruppen}

\subsection{Definition} Eine Gruppe\index{Gruppe} ist ein geordnetes Paar $(G, \cdot)$, wobei $G$ eine Menge ist und $\cdot : G \times G \to G$ eine meist infix (und manchmal gar nicht) notierte Abbildung mit folgenden Eigenschaften ist:
	\begin{itemize}
		\item[(A)]
			$\forall a, b, c \in G : a(bc) = (ab)c$ \quad\quad \textit{"`assoziativ"'}
		\item[(N)]
			$\exists e \in G ~\forall a \in G : ae = a = ea$ \quad\quad \textit{"`neutrales Element"'}
		\item[(I)]
			$\forall a \in G ~\exists g \in G : ab = 1 = ba$ \quad\quad \textit{"`inverse Elemente"'}
	\end{itemize}
"`$\cdot$"' heißt Gruppenmultiplikation\index{Gruppe!-nmultiplikation} oder Gruppenverknüpfung\index{Gruppe!-nverknüpfung}. Gilt zusätzlich
	\begin{itemize}
		\item[(K)] $\forall a,b \in G : ab = ba$
	\end{itemize}
so heißt $(G, \cdot)$ abelsch\index{Gruppe!abelsche} oder kommutativ\index{Gruppe!kommutative}.

\subsection*{Anmerkung} Sind $e, e' \in G$ neutral, so $e = ee' = e'$. Daher gibt es genau ein neutrales Element, für welches man oft "`$1$"' schreibt.

\subsection{Bemerkung}
	\begin{enumerate}[label=(\alph*)]
		\item
			Sei $(G, \cdot)$ eine Gruppe und $a \in G$. Seien $b, b'$ invers zu $a$. Dann
			$$b \GRPeqn b \cdot 1 \GRPeqi b(ab') \GRPeqa (ba)b' \GRPeqi 1 \cdot b \GRPeqn b'.$$
			Daher gibt es zu jedem $a \in G$ genau ein inverses Element in $G$, welches wir mit $a^\mo$ bezeichnen.
			
		\item
			(N) und (I) kann man wie folgt schreiben:
			\begin{itemize}
				\item[(N)] $\forall a \in G : a1 = a = 1a$
				\item[(I)] $\forall a \in G : aa^\mo = 1 = a^\mo a$
			\end{itemize}
			
		\item
			Oft: "`Sei $G$ eine Gruppe"', statt: "`Sei $(G, \cdot)$ eine Gruppe."'
		
		\item
			Sei $G$ eine Gruppe, $n \in \N_0$ und $a_1, ..., a_n \in G$. Dann definiert man $\prod_{i=1}^n a_i := a_1 \cdot ... \cdot a_n$ als 1 für $n = 0$ und indem man $a_1 \cdot ... \cdot a_n$ sinnvoll mit Klammern versieht, sonst. Dies hängt nicht von der Wahl der Klammerung da, wie (A) für $n = 3$ besagt. Für $n > 3$ siehe \LAref{2.1.6} oder mache es als Übung per Induktion. Falls $G$ additiv geschrieben ist, schreibt man $\sum_{i=1}^n a_i$, statt $\prod_{i=1}^n a_i$.
			
		\item
			Sei $G$ eine Gruppe, $n \in \Z$ und $a \in G$. Dann definiert man
			$$a^n := \begin{cases}\prod_{i=1}^n a, & \text{für} ~ n \geq 0, \\ \prod_{i=1}^n (a^\mo), & \text{für} ~ n \leq 0.\end{cases}$$
			Fall $G$ additiv geschrieben ist, schreibt man $na$, statt $a^n$.
	\end{enumerate}

\subsection{Definition} Ist $(G, \cdot)$ eine Gruppe, so nennt man $\#G \in \N_0 \cup \{\infty\}$ die Ordnung\index{Ordnung} von $(G,\cdot)$.

\subsection{Beispiel}
	\begin{enumerate}[label=(\alph*)]
		\item
			Für jede Menge $M$ bildet die Menge $S_M := \{f \mid f: M \to M ~ \text{bijektiv}\}$ mit der durch $fg := f \circ g$ $(f,g \in S_M)$ gegebenen Multiplikation eine Gruppe. Man nennt sie die symmetrische Gruppe\index{Gruppe!symmetrische} auf M. Das neutrale Element von $S_M$ ist die Identität auf $M$ und das zu einem $f \in S_M$ inverse Element ist die Umkehrfunktion von $f$, wodurch die Notation $f^\mo$ nicht zweideutig ist.
		
			Für $n \in \N_0$ ist $S_n := S_{\{1,...,n\}}$ eine Gruppe der Ordnung $n! := \prod_{i=1}^n i$ "`$n$ Fakultät\index{Fakultät}"'. Für $n \geq 3$ ist die nicht abelsch, dann die Transpositionen $\tau_{1,2}$ und $\tau_{2,3}$ konvertieren nicht, d.h. $\tau_{1,2}\tau_{2,3} \neq \tau_{2,3}\tau_{1,2}$. In der Tat: $(\tau_{1,2}\tau_{2,3})(1) = \tau_{1,2}(1) = 2$ und $(\tau_{2,3}\tau_{1,2})(1) = \tau_{2,3}(2) = 3$.
		
		\item
			Für jeden Vektorraum $V$ ist die Menge $\Aut(V) := \{f \mid f : V \to V ~\text{linear und bijektiv}\}$\index{Automorphismus!Vektorraum-} mit der Hintereinanderschaltung als Multiplikation eine Gruppe.
			
		\item
			Ist $R$ ein kommutativer Ring (z.\ B. $R = \Z$), so ist $\GL_n(R) := \{A \in R^\ntn \mid A ~\text{invertierbar}\} = \{ A \in R^\ntn \mid \det A \in R^\times \}$\index{General Linear Group} eine Gruppe.
	\end{enumerate}
	
\subsection{Proposition} Sei $G$ eine Gruppe und $a,b \in G$.
\begin{enumerate}[label=(\alph*)]
		\item
			$ab = 1 \iff a = b^\mo \iff b = a^\mo$
		\item
			$\left(a^\mo\right)^\mo = a$
		\item
			$(ab)^\mo = b^\mo a^\mo$
\end{enumerate}

\bew
\begin{enumerate}[label=(\alph*)]
	\item
		Gilt $ab = 1$, so $a \GRPeqn a1 \GRPeqi a(bb^\mo) \GRPeqa (ab) b^\mo = 1b \GRPeqn b^\mo$. Gilt $a = b^\mo$, so $b \GRPeqn 1b \GRPeqi (a^\mo a)b \GRPeqa a^\mo(ab) = a^\mo(b^\mo b) \GRPeqi a^\mo 1 \GRPeqn a^\mo$. Gilt $b = a^\mo$, so $ab = 1$.
		
	\item
		Aus $a a^\mo \GRPeqi 1$ folgt mit (a) $\left( a^\mo \right)^\mo = a$.
		
	\item
		Aus $(ab)(b^\mo a^\mo) \GRPeqa a(b(b^\mo a^\mo)) \GRPeqa a((b b^\mo)a^\mo) \GRPeqi a(1a^\mo) \GRPeqn aa^\mo \GRPeqi 1$ folgt mit (a) $(ab)^\mo = b^\mo a^\mo$. \qed		
\end{enumerate}

\subsection{Definition} Seien $(G, \cdot_G)$ und $(H, \cdot_H)$ Gruppen. Dann heißt $(H, \cdot_H)$ eine Untergruppe\index{Gruppe!Unter-} von $(G, \cdot_G)$, wenn $H \subseteq G$ und $\forall a, b \in H : a \cdot_H b = a \cdot_G b$.

\subsection{Proposition} Sei $(G, \cdot_G$ eine Gruppe und $H$ eine Menge. Dann ist $H$ genau dann Trägermenge\index{Träger} einer Untergruppe von $(G, \cdot_g)$, wenn $H \subseteq G$, $1_S \in H$, $\forall a,b \in H : a \cdot_S b \in H$ und $\forall a \in H : a^ \mo \in H$.

In diesem Fall gibt es genau eine Abbildung $\cdot_H : H \times H \to H$ derart, dass $(H, \cdot_H)$ eine Untergruppe von $(G, \cdot_G)$ ist. Es gilt dann $1_H = 1_G$, $\forall a,b \in H : a \cdot_H b = a \cdot_G b$ und $a^\mo = a^\mo$ (je in $G$ und $H$ gebildet).

\bew Klar oder vgl.\ LA § 2. \qed

\subsection{Bemerkung}
\begin{enumerate}[label=(\alph*)]
	\item
		Ist $(H, \cdot_H)$ Untergruppe von $(G, \cdot_G)$, so schreibt man meist $\cdot$ statt $\cdot_H$. Oft erwähnt man $\cdot_H$ gar nicht mehr und schreibt einfach "`$H$ ist Untergruppe von $G$"' oder $H \leq G$.
		
	\item
		Untergruppen abelscher Gruppen sind abelsch.
\end{enumerate}

\subsection{Beispiel}
\begin{enumerate}[label=(\alph*)]
	\item
		Für $n \in \N_0$ ist $A_n := \{\sigma \in S_n \mid \sgn \sigma = 1\}$ eine Untergruppe von $S_n$, die man alternierende Gruppe \index{Gruppe!alternierende} nennt. \LAref{§ 9.1}
\end{enumerate}

\numberfix{2}{1}{3}
\subsection{Beispiel}
\begin{enumerate}[label=(\alph*)]
	\item
		Für jeden Vektorraum $V$ ist die Menge $\End(V)=\{f \mid f: V \to V ~ \text{linear}\}$\index{Endomorphismus!Vektorraum-} der Endomorphismen von $V$ mit der punktweisen Addition und der Hintereinanderschaltung als Multiplikation ein Ring mit Einheitengruppe $\End(V)^{\times} = \Aut(V)$. \LAref{§ 7.1}
		
	\item
		Ist $R$ ein kommutativer Ring, so ist $R^{n \times n}$ ein  Ring mit $(R^{n \times n})^{\times} = \GL_n(R)$.
\end{enumerate}

\subsection{Definition} Seien $(A,+_A,\cdot_A)$ und $(B,+_B,\cdot_B)$ Ringe. Dann heißt $(A,+_A,\cdot_A)$ ein \emph{Unterring}\index{Unterring}\index{Ring!Unter-}  von $(B,+_B,\cdot_B)$, wenn $A \subseteq B$, $1_B \in A$, $\forall a,b \in A: a +_A b = a +_B b$, $\forall a,b \in A : a \cdot_A b = a \cdot_B b$.
 
\subsection{Proposition}
 Sei $(B,+,\cdot)$ ein Ring und $A$ eine Menge. Genau dann ist $A$ Trägermenge eines Unterrings von $(B,+,\cdot)$, wenn $\{0,1\} \subseteq A \subseteq B$, $\forall a,b \in A: a+b \in A, a \cdot b \in A$.
 
\subsection{Beispiel}
\begin{enumerate}[label=(\alph*)]
	\item
		Sei $R$ ein kommutativer Ring und  $n \in \mathbb{N}_0$. Dann sind $\Odm_R^{n \times n} = \{ A \in R^{n \times n} \mid A \text{ obere Dreiecksmatrix}\}$, $\Udm_R^{n \times n} = \{ A \in R^{n \times n} \mid A \text{ untere Dreiecksmatrix}\}$ und $\Udm_R^{n \times n} \cap \Odm_R^{n \times n} = \{ A \in R^{n \times n} \mid A \text{ Diagonalmatrix}\}$ Unterringe von $R^{n \times n}$ mit Einheitengruppen $(\Odm_R^{n \times n})^{\times} = \Odm_n(R)$, $(\Udm_R^{n \times n})^{\times} = \Udm_n(R)$ und $\left(\Udm_R^{n \times n} \cap \Odm_R^{n \times n}\right)^{\times} = \Odm_n(R) \cap \Udm_n(R)$.
		
	\item
		$\{ 0 \}$ ist kein Unterring von $\mathbb{Z}$, denn $1 \notin \{0\}$.
\end{enumerate}

\subsection{Definition} Seien $A$ und $B$ Ringe. Dann heißt $f: A \to B$ ein \emph{(Ring-)Homomorphismus}\index{Homomorphismus!Ring-} von $A$ nach $B$, wenn
\begin{align*}
	&f \text{ ein Gruppenhomomorphismus von } A \text{ nach } B \text{ ist,}\\
	&f(1)=1 \text{ und}\\
	&\forall a,b \in A: f(ab) = f(a)f(b) \text{ gilt.}
\end{align*}

Ein Ringhomomorphismus heißt
\begin{center}
	\begin{tabular}{llll}
		(Ring-) & (Einbettung oder) Mono-  & / Epi- & / Isomorphismus\\
		wenn $f$ & injektiv & / surjektiv & / bijektiv ist, \\
		in Zeichen & $f: A \MAPlongmono B$ & / $f: A \MAPlongepi B$ & / $f: A \MAPlongiso B$
	\end{tabular}
	\index{Einbettung!Ring-}
	\index{Monomorphismus!Ring-}
	\index{Epimorphismus!Ring-}
	\index{Isomorphismus!Ring-}
\end{center}

\subsection{Bemerkung} Ist $f: A \to B$ ein Ringhomomorphismus, so ist $\im f$ ein Unterring von $B$, jedoch $\ker f$ in aller Regel kein Unterring von $A$. (Denn $1 \in \ker f \iff f(1)=0$ in $B$ $\iff 1=0$ in $B$. \lightning)
 
\subsection{Bemerkung} Analog zu 1.2.7 und 1.2.8 führt man das \emph{direkte Produkt}\index{Direktes Produkt!von Ringen}\index{Produkt!direktes!von Ringen} von Ringen durch punktweise Addition und Multiplikation ein.
 
\subsection{Definition und Proposition } \ALref{§ \ref{fixed:1.3}}, \LAref{§ 3.3} Sei $R$ ein Ring. Eine \emph{Kongruenzrelation}\index{Kongruenzrelation} auf $R$ ist eine Kongruenzrelation $\equiv$  auf der additiven Gruppe von $R$ \ALref{\ref{fixed:1.3.1}}, für die zusätzlich gilt:
\begin{equation*}
	\forall a,a',b,b'\in A:((a\equiv a' ~\&~ b\equiv b')\implies ab\equiv a'b')
\end{equation*}
Ist $\equiv$ ein Kongruenzrelation auf $R$, so wird $R/\equiv$ vermöge $\overline a + \overline b = \overline{a+b}$ und $\overline a \overline b = \overline{ab}$ $(a,b\in A)$ zu einem Ring (
	\emph{"`Quotientenring"'}\index{Quotientenring}\index{Ring!Quotienten-}
	\emph{"`Faktorring"'}\index{Faktorring},\index{Ring!Faktor-}
	\emph{"`Restklassenring"'}\index{Restklassenring}\index{Ring!Restklassen-}
).
 
\subsection{Definition} Sei $R$ ein Ring. Eine Untergruppe $I$ der additiven Gruppe von $R$ heißt (beidseitiges) \emph{Ideal}\index{Ideal} von $R$, wenn:
\begin{equation*}
	\forall a\in R ~\forall b \in I : ab, ba \in I
\end{equation*}
 
\subsection{Satz} \ALref{\ref{fixed:1.3.9}} \LAref{§ 3.3} Sei $R$ ein Ring. Die Zuordnungen
\begin{align*}
	\equiv&\mapsto\overline0\\
	\equiv_I&\mapsfrom I
\end{align*}
vermitteln eine Bijektion zwischen der Menge der Kongruenzrelationen auf $R$ und der Menge der Ideale von $R$.
 
\proof Wenn wir zeigen, dass beide Abbildungen wohldefiniert sind, dann folgt mit \ref{fixed:1.3.9}, dass sie auch invers zueinander sind. Also zu zeigen:
\begin{enumerate}[label=(\alph*)]
	\item $\equiv$ ist Kongruenzrelation auf $R \implies \overline 0$ ist Ideal von $R$
	\item $I$ ist Ideal von $R \implies \equiv_I$ ist Kongruenzrelation auf $R$
\end{enumerate}

{\bf Zu (a).} Sei $\equiv$ eine Kongruenzrelation auf $R$. Aus \ref{fixed:1.3.9} wissen wir schon, dass $\overline 0$ eine Untergruppe von $R$ ist. Noch zu zeigen: $\forall a\in A:\forall b\in\overline0:ab\in\overline0$. Sei also $a\in R$ und $b\in\overline 0$. Dann $ab \overset{b \equiv 0} \equiv a0 \overset{2.1.2(e)}\equiv0$, also $ab\in\overline0$ und $ba\equiv0a\equiv0$, also $ba\in\overline0$.
 
{\bf Zu (b).} Sei $I$ eine Ideal von $R$. Aus \ref{fixed:1.3.9} wissen wir schon, dass $\equiv_I$ eine Kongruenzrelation der additiven Gruppe von $R$ ist. Noch zu zeigen: $\forall a,a',b,b'\in A:((a\equiv a' ~\&~ b\equiv b')\implies ab\equiv a'b')$. Seien also $a,a',b,b'\in R$ mit $a\equiv_Ia'$ und $b\equiv_Ib'$. Dann $ab-a'b'=a\underbrace{(b-b')}_{\in I} + b'\underbrace{(a-a')}_{\in I}\in I$, also $ab\equiv_Ia'b'$. \qed

\subsection{Notation \& Sprechweise} Sei $I$ ein Ideal des Ringes $R$. Schreibe $R/I := R/\equiv_I ~:= \{ a+I \mid a \in R \}$. Man bezeichnet die Kongruenzklasse $\overline{a}^I = a+I$ von $a\in R$ auch als \emph{Restklasse}\index{Restklasse} von $a$ modulo $I$.
 
\subsection{Bemerkung}
\begin{enumerate}[label=(\alph*)]
	\item
		Sei $I$ ein Ideal des Ringes $R$. Dann ist die Abbildung $R \to R/I, a \mapsto \overline{a}^I$ nach Definition \ref{fixed:2.1.11} ein Ringhomomorphismus, genannt \emph{kanonischer Epimorphismus}\index{Epimorphismus!Ring-!kanonischer}.
		
	\item
		Sei $f: A \to B$ ein Ringhomomorphismus. Dann ist $\ker f$ ein Ideal von $A$, aber $\im f$ im Allgemeinen kein Ideal von $B$. (Betrachte zum Beispiel $\Z \MAPmono \Q, a \mapsto a$.)
\end{enumerate}

\subsection{Homomorphiesatz für Ringe\index{Homomorphiesatz!für Ringe}}
	Seien $A, B$ Ringe, $I$ ein Ideal von $A$ und $\varphi: A \to B$ ein Homomorphismus mit $I \subseteq ker \varphi$. Dann gibt es genau eine Abbildung $\overline{\varphi}: A/I \to B$ mit $\overline{\varphi}(\overline{a}^I)=\varphi(a)$ für alle $a \in A$. Diese Abbildung $\overline{\varphi}$ ist ein Homomorphismus. Weiter gilt $\overline{\varphi}$ injektiv $\iff I = \ker \varphi$ und $\overline{\varphi}$ surjektiv $\iff B = \im \varphi$.
 
\proof Mit \ref{fixed:1.3.15} ist nur noch $\overline{\varphi}(1)=1$ und $\overline{\varphi}(\overline{a}^I \overline{b}^I) = \overline{\varphi}(\overline{a}^I) \overline{\varphi}(\overline{b}^I)$ f.a. $a,b\in A$ zz zeigen.

Dies ist klar:
\begin{align*}
  &\overline{\varphi}(1)=\overline{\varphi}(\overline{1}^I)=\varphi(1)=1 \quad\text{und} \\
  &\overline{\varphi}(\overline{a}^I \overline{b}^I)=\overline{\varphi}(\overline{ab}^I)=\varphi(ab)=\varphi(a)\varphi(b)=
  \overline{\varphi}(\overline{a}^I)\overline{\varphi}(\overline{b}^I) \quad\text{für alle}~ a,b\in A. 
\end{align*}
\qed
 
\subsection{Isomorphiesatz für Ringe\index{Isomorphiesatz! für Ringe}} Seien $A, B$ Ringe und $\varphi: A \to B$ ein Homomorphismus. Dann ist $\ker \varphi$ ein Ideal von $A$ und $\overline{\varphi}: A/\ker\varphi \to \im\varphi$ mit $\overline{\varphi}(\overline{a}^{\ker \varphi})=\varphi(a)$ für $a \in A$ ein Isomorphismus. Insbesondere $A/\ker\varphi \cong \im \varphi$.
 
\proof Direkt aus \ref{fixed:2.1.16}. \qed
 
\section{Polynomringe \small\LAref{§ 3.2}}

\subsection{Notation} Sei $R$ ein kommutativer Ring, $n \in \N_0$, $a=(a_1,...,a_n)\in R^n$ und $\alpha=(\alpha_1,...,\alpha_n) \in \N_0^n$. Schreibe dann $|\alpha|=\alpha_1+...+\alpha_n$ und $a^\alpha:=a_1^{\alpha_1}+...+a_n^{\alpha_n}$.

\subsection{Definition \& Satz} Sei $A$ ein Unterring des kommutativen Ringes $B$.
\begin{itemize}
	\item[(a)]
		Sei $n \in \N_0$ und $b=(b_1,...,b_n)\in B^n$.
		\begin{equation*}
			 A[b] := A[b_1,...,b_n]
			 :=\left\{\sum_{\alpha \in \N_0^n, \atop |\alpha|<d} a_\alpha b^\alpha \mid d \in \N_0, a_\alpha\in A\right\}
		\end{equation*}
		ist der kleinste Unterring $C$ von $B$ mit $A \cup \{b_1,...,b_n\} \subseteq C$.
   
	\item[(b)]
		Sei $E \subseteq B$. $A[E] = \bigcup\{A[b] \mid n\in \N_0, b\in B^n\}$ ist der kleinste Unterring $C$ von $B$ mit $A \cup E \subseteq C$.
\end{itemize}

\proof Dass die angegeben Mengen jeweils in jedem solchen Unterring $C$ enthalten sind, ist klar. Zu zeigen ist dann nur noch, dass sie jeweils einen Unterring bilden. Dies ist einfach und wir zeigen exemplarisch nur, dass $A[b]$ aus (a) unter Multiplikation abgeschlossen ist. Seien also $d, d' \in \N_0, a_\alpha \in A$ für alle $\alpha \in \N_0^n$ mit $|\alpha| \leq d$ und $a_\alpha' \in A$ für alle $\alpha \in \N_0^n$ mit $|\alpha| \leq d'$. Dann
\begin{equation*}
	\left(\sum_{|\alpha| \leq d} a_\alpha b^\alpha\right)\left(\sum_{|\alpha| \leq d'} a_\alpha' b^\alpha\right)
	= \sum_{|\gamma| \leq d+d'} \left(\sum_{\alpha+\beta=\gamma} a_\alpha a'^\beta\right) b^\gamma~\in A[b],
\end{equation*}
wobei man $a_\alpha:=0$ für $d<|\alpha| \leq d+d'$ und $a_\alpha':=0$ für $d'<|\alpha| \leq d+d'$ setzt. \qed

\subsection{Definition} Sei $A$ ein Unterring des kommutativen Ringes $B$.
\begin{enumerate}[label=(\alph*)]
	\item
		Sei $n \in \N_0$ und $b=(b_1,...,b_n) \in B^n$. Es heißen $b_1,...,b_n$ \emph{algebraisch unabhängig}\index{unabhängig!algebraisch}\index{algebraisch!unabhängig} über $A$ (in $B$), wenn für alle $d \in \N_0$ und alle $a_\alpha \in A (\alpha \in \N_0^n, |\alpha|\leq d)$ gilt:
		\begin{equation*}
			\sum_{\alpha \in \N_0^n, \atop |\alpha|<d} a_\alpha b^\alpha = 0 \implies \forall\alpha\in\N_0:(|\alpha|\leq d \implies a_\alpha = 0)
		\end{equation*}
		
		Es heißt $B$ \emph{Polynomring}\index{Polynomring}\index{Ring!Polynom-} über $A$ in $b_1,...,b_n$, wenn $B=A[b_1,...,b_n]$ und $b_1,...,b_n$ algebraisch unabhängig über $A$ sind.
		
	\item
		Sei $E \subseteq B$. Es heißt $E$ \emph{algebraisch unabhängig}\index{unabhängig!algebraisch}   über $A$ (in $B$), wenn für alle $n\in\N_0$ alle paarweise verschiedenen Elemente $b_1,...,b_n \in E$ algebraisch unabhängig über $A$ sind.
		
		Es heißt $B$ \emph{Polynomring}\index{Ring!Polynom-} über $A$ in $E$, wenn $B=A[E]$ und $E$ algebraisch unabhängig über $A$ ist.
\end{enumerate}

\subsection{Beispiel}
\begin{enumerate}[label=(\alph*)]
	\item
		Jeder kommutative Ring $A$ ist ein Polynomring über sich selbst in $\emptyset$.
	\item
		Der Nullring $\{0\}$ ist ein Polynomring über sich selbst in $0$.
\end{enumerate}

\subsection{Satz} Sei $A$ ein kommutativer Ring mit $0\neq1$. Sei $E$ eine Menge mit $A \cap E = \emptyset$. Dann gibt es einen Polynomring über $A$ in $E$.
 
\proof Bezeichne $\N_0^{(E)}$ die Menge aller $\alpha: E \to \N_0$ mit endlichem Träger $supp(\alpha) = \{ e\in E \mid \alpha(e) \neq 0 \}$. Mache die abelsche Gruppe $A^{\N_0^{(E)}}$ zu einem kommutativen Ring mit der "`Faltung"' $\ast$ als Multiplikation, welche gegeben ist durch 
\begin{align*}
	  (f \ast g)(\gamma):= \sum_{\alpha,\beta\in\N_0^{(E)}, \atop \alpha+\beta=\gamma} f(\alpha)g(\beta) && \left(f,g\in A^{\N_0^{(E)}}, \gamma \in \N_0^{(E)}\right)
\end{align*}
(Es handelt sich um eine endliche Summe, da $supp(\gamma)$ endlich. Man sieht sofort $f \ast g = g \ast f$, $f \ast (g+h) = f \ast g + f \ast h$ und $1 \ast f = f$ für
\begin{align*}
	1 : \N_0^{(E)} & \to A \\
	\alpha &\mapsto \begin{cases} 1, & \alpha = 0 \\ 0, & sonst \end{cases}
\end{align*}
und rechnet
\begin{align*}
	((f \ast g) \ast h)(\gamma) &= \sum_{\alpha+\beta=\gamma} (f \ast g)(\alpha)h(\beta)
 	= \sum_{\alpha+\beta=\gamma} \left( \sum_{\delta+\varepsilon=\alpha} f(\delta)g(\varepsilon) \right) h(\beta) \\
 	&= \sum_{\delta+\varepsilon+\beta=\gamma} f(\delta)g(\varepsilon)h(\beta)
 	= ... = (f \ast (g \ast h))(\gamma)
\end{align*}
für alle $f,g,h\in A^{\N_0^{(E)}}, \gamma \in\N_0^{(E)}$. \qed\footnote{Korrektur: Ist der Beweis vollständig? Hier fehlen noch 2.2.6, 2.2.7 und 2.2.8 aus dieser Vorlesung.}

\numberfix{2}{3}{5}
\subsection{Satz} \label{fixed:2.3.6} Sei $A$ ein kommutativer Ring und $S \subseteq A$ eine multiplikative Menge, die keine Nullteiler von $A$ enthält. Dann gibt es einen kommutativen Oberring $B$ von $A$ mit $S \subseteq B^\times$ und $B = S^\mo A$.

\proof Durch $(a,s) \sim (b,t) :\iff at = bs$ ($a,b \in A, s,t \in S$) wird eine Äquivalenzrelation $\sim$ auf $A \times S$ definiert. [Reflexiv und symmetrisch ist klar, transitiv: Seien $a,b,c \in A$ und $s,t,u \in S$ mit $(a,s) \sim (b,t) \sim (c,u)$. Dann $at=bs$ und $bu = ct$, also $atu = bsu = bus = cts$, das heißt $t(au-cs) = 0$ und daher $au = cs$, da $t \in S$ kein Nullteiler ist.] Der Leser zeigt als Übung, dass $+$ und $\cdot$ durch
\begin{align*}
	\widetilde{(a,s)} + \widetilde{(b,t)} &:= \widetilde{(at+bs, st)} \quad \text{und}\\
	\widetilde{(a,s)} \cdot \widetilde{(b,t)} &:= \widetilde{(ab,st)}
\end{align*}
wohldefiniert ist und $(A \times S)/{\sim}$ zu einem kommutativen Ring mit $0 = \widetilde{(0,1)}$, $1 = \widetilde{(1,1)}$ machen.

Wegen $A \cong \tilde A := \{\widetilde{(a,1)} \mid a \in A\} \subseteq (A \times S)/\sim$ reicht es zu zeigen, dass $\tilde S := \{\widetilde{(s,1)} \mid s \in S\} \subseteq \left((A \times S)/{\sim})\right)^\times$ und $(A \times S)/{\sim} = \tilde S^\mo \tilde A$. Sei hierzu $a \in A$, $s \in S$. Dann $\widetilde{(s,1)} \widetilde{(1,s)} = \widetilde{(s,s)} = \widetilde{(1,1)} = 1$, also $\widetilde{(s,1)}^\mo = \widetilde{(1,s)}$ und $\widetilde{(a,s)} = \widetilde{(s,1)}^\mo \widetilde{(a,1)} \in \tilde S^\mo \tilde A$. \qed

\subsection{Satz} \label{fixed:2.3.7} Sei $A$ ein Unterring des kommutativen Ringes $B$, $S \subseteq A \cap B^\times$ multiplikativ und $B = S^\mo A$. Sei $C$ ein weiterer Ring und $\varphi : A \to C$ ein Homomorphismus. Genau dann gibt es einen Homomorphismus $\psi : S^\mo A \to C$ mit $\varphi = \psi|_A$, wenn $\varphi(S) \subseteq C^\times$. In diesem Fall ist $\psi$ eindeutig bestimmt, denn es gilt $\psi\left(\frac{a}{s}\right) = \frac{\psi(a)}{\psi(s)}$ für $a \in A, s \in S$.

\proof Übung. \qed

\subsection{Satz} \label{fixed:2.3.8} Sei $A$ ein Unterring des kommutativen Ringes $B$, $S \subseteq A \cap B^\times$ multiplikativ und $B = S^\mo A$. Dasselbe gelte mit $C$ statt $B$. Dann gibt es genau einen Isomorphismus $\psi: B \to C$ mit $\psi|_A = \id_A$.

\proof Wende \ref{fixed:2.3.7} mit $\varphi: A \to C, a \mapsto a$ an, um zu sehen, dass $\id_A$ eine eindeutige Fotsetzung zu einem Homomorphismus $\psi : B \to C$ hat. Zu zeigen ist nur noch, dass $\psi$ ein Isomorphismus ist. Mit \ref{fixed:2.3.7} bekommt man aber auch einen Homomorphismus $\varphi: C \to B$ mit $\varphi|_A = \id_A$. Nun ist $\varphi \circ \psi: C \to C$ ein Homomorphismus mit $(\varphi \circ \psi)|_A = \id_A$ und daher $\varphi \circ \psi = \id_C$ nach \ref{fixed:2.3.7}. Ebenso $\psi \circ \varphi = \id_B$. Daher sind $\varphi$ und $\psi$ bijektiv. \qed

\subsection{Definition} \label{fixed:2.3.9} Sei $A$ ein kommutativer Ring und $S \subseteq A$ eine multiplikative Menge, die keine Nullteiler von $A$ enthält. Den (nach \ref{fixed:2.3.6} existierenden und nach \ref{fixed:2.3.8} im Wesentlichen eindeutigen) Oberring $B$ von $A$ mit $S \subseteq B^\times$ und $B = S^\mo A$ nennt man Ring der Brüche\index{Ring!der Brüche} mit Zählern\index{Zähler} aus $A$ und Nennern\index{Nenner} aus $S$ (oder Lokalisierung\index{Lokalisierung|\see{Ring!der Brüche}} von $A$ nach $S$).

Ist speziell $S$ die Menge aller Nichtnullteiler von $A$ (vgl.~\ref{fixed:2.3.2.c}), so nennt man $Q(A) = S^\mo A$ den totalen Quotientenring\index{Quotientenring!totaler} von $A$. Offenbar gilt: $Q(A)$ ist Körper $\iff$ $A$ ist Integritätsring. Ist $A$ ein Integritätsring, so nennt man den Körper $\qf(A) := Q(A) = (A\setminus\{0\})^\mo A$ daher auch den Quotientenkörper von $A$.

\subsection{Bemerkung} Es folgt nun, dass Integritätsringe genau die Unterringe von Körpern sind.

\subsection{Definition und Satz} (Körperadjunktion, vgl.~Ringadjunktion \ref{fixed:2.2.2})
\begin{enumerate}[label=(\alph*)]
	\item
		Ist $K$ ein Unterring eines Körpers $L$ und $K$ ein Körper, so nennt man
		\begin{itemize}[label=$-$]
			\item
				$K$ einen Unterkörper\index{Körper!Unterkörper} von $L$,
			\item
				$L$ einen Oberkörper\index{Körper!Oberkörper} von $K$ und
			\item
				$L|K$ ("`über"') eine Körpererweiterung\index{Körpererweiterung}.
		\end{itemize}
	
	\item
		Sei $L|K$ eine Körpererweiterung. Sind $b_1,...,b_n \in L$, so ist $K(b1,...,b_n) := (K[b_1,...,b_n]\setminus\{0\})^\mo K[b_1,...,b_n] = \qf(K[b_1,...,b_n]) \subseteq L$ der kleinste Unterkörper\index{Körper!Unterkörper!kleinster} $F$ von $L$ mit $K \cup \{b_1,...,b_n\} \subseteq F$.
	
		Ist $E \subseteq L$, so ist $K(E) := (K[E]\setminus\{0\})^\mo K[E] = \qf(K[E]) \subseteq L$ der kleinste Unterkörper $F$ von $L$ mit $K \cup E \subseteq F$.
\end{enumerate}

\proof Trivial. \qed

\subsection{Definition} (vgl.~\ref{fixed:2.3.3}) Sei $L|K$ eine Körpererweiterung.
\begin{enumerate}[label=(\alph*)]
	\item
		Sei $n \in \N_0$ und $b_1,...,b_n \in L$. Es heißt $L$ ein Körper der rationalen Funktionen\index{Körper!der rationalen Funktionen} über $K$ in $b_1,...,b_n$, wenn $L = K[b_1,...,b_n]$ und $b_1,...,b_n$ algebraisch unabhängig über $K$1 sind.
		
	\item
		Sei $E \subseteq L$. Es heißt $L$ ein Körper von rationalen Funktionen\index{Körper!von rationalen Funktionen} über $K$ in $E$, wenn $L = K[E]$ und $E$ algebraisch unabhängig über $K$ ist.\footnote{An Korrektor: War mir hier bzgl.~der Klammern und der Namen (Index!) nicht ganz sicher.}
\end{enumerate}

\subsection{Proposition} \label{fixed:2.3.13} (vgl.~\ref{fixed:2.2.6}) Sei $L|K$ eine Körpererweiterung und $E \subseteq L$ mit $L = K[E]$. Sei $R$ ein Ring und seiden $\varphi, \psi: L \to R$ Homomorphismen mit $\varphi|_{K \cup E} = \psi|_{K \cup E}$. Dann $\varphi = \psi$.

\proof $F := \{a \in L | \varphi(a) = \psi(a)\}$ ist ein Unterkörper von $L$, der $K \cup E$ enthält. Also $F = L$. \qed

\subsection{Definition und Proposition} Seien $K$ und $F$ Körper.
\begin{enumerate}[label=(\alph*)]
	\item
		$K$ besitzt nur die trivialen Ideale $K$ und $\{0\}$.
	\item
		Ist $\varphi: K \to F$ ein (Ring-)Homomorphismus, so nennt man $\varphi$ auch einen Körperhomomorphismus\index{Homomorphismus!Körper-}\index{Homomorphismus!Körper-}. In diesem Fall gilt: Da $\varphi(1) = 1 \neq 0$ in $F$, liegt $1$ nicht im Ideal $\ker \varphi$ von $K$, womit $\ker \varphi = \{0\}$ nach (a). Es ist daher $\varphi : K \MAPmono F$ eine Einbettung und $\varphi : K \MAPiso \im \varphi$ ein Isomorphismus. Insbesondere ist das Bild von $\varphi$ nicht nur ein Unterring, sondern sogar ein Unterkörper von $F$. Beachte auch, dass gelten muss $\varphi\left(-\frac{1}{a}\right) = \frac{1}{\varphi(a)}$ für alle $a \in K^\times$.\footnote{An Korrektor: Gehört da wirklich ein Minus hin?}
\end{enumerate}

\subsection{Satz} (vgl.~\ref{fixed:2.2.9}) Seien $K(E)$ und $K(F)$ Körper von rationalen Funktionen über $K$ in $E$ bzw.~$F$. Sei $f: E \to F$ eine Bijektion. Dann gibt es genau einen Isomorphismus $\psi: K(E) \to K(F)$ mit $\psi|_K = \id_K$ und $\psi|_E = f$.

\proof Zur Existenz: Nach \ref{fixed:2.2.9} gibt es einen Isomorphismus $\varphi: K[E] \to K[F]$ mit $\varphi|_K = \id_K$ und $\varphi_E = f$. Da $\varphi$ injektiv ist, gilt $\varphi(K[E]\setminus\{0\}) \subseteq K[F]\setminus\{0\} \subseteq K(F)^\times$ und \ref{fixed:2.3.7} liefert einen Homomorphismus $\psi : K(E) \to K(F)$ mit $\psi|_{K[E]} = \varphi$. Da $\psi$ ein Körperhomomorphismus ist, ist $\psi$ injektiv und $\psi$ ist ein Unterkörper von $K(F)$.\footnote{An Korrektor: Macht so keinen Sinn.} Es gilt aber $K \cup F \subseteq \im \varphi \subseteq \im \psi$, weswegen $\psi$ surjektiv ist.

Die Eindeutigkeit folgt aus \ref{fixed:2.3.13}.

\subsection{Notation und Sprechweise} (vgl.~\ref{fixed:2.2.10}) Sei $K$ ein Körper. Schreibt man $K(X_1,...X_n)$, so meint man dabei den (nach \ref{fixed:2.3.15} im Wesentlichen eindeutig bestimmen und nach \ref{fixed:2.3.5} und \ref{fixed:2.3.9} existierenden) Körper der rationalen Funktionen\index{Körper!der rationalen Funktionen!in Unbestimmten} in paarweise verschiedenen "`unbestimmten"' $X_1,...,X_n$.\footnote{An Korrektor: Index für "`Körper der rationalen Funktionen"' anpassen.}

\subsection{Definition und Proposition} Sei $A$ ein kommutativer Ring und $S \subseteq A$ eine multiplikative Menge. Wenn $S$ Nullteiler enthält (das heißt, wenn es $s \in S$ und $a \in A$ gibt mit $sa=0$), dann können wir keinen Oberring $S^\mo A$ wie in \ref{fixed:2.3.6} konstruieren (siehe \ref{fixed:2.3.5}). In diesem Fall (und allgemein) setzten wir $I_S := \{a \in A \mid \exists s \in S : sa=0\}$. Es ist $I_S$ ein Ideal von $A$, das $S$ multiplikativ ist. Es ist dann $\bar S := \{\bar s \mid s \in S\} \subseteq \bar A := A/I_S$ multiplikativ und ohne Nullteiler. Man nennt dann den Oberring $\bar S^\mo \bar A$ von $\bar A = \bar{A/I_S}$ die Lokalisierung\index{Lokalisierung} von $A$ nach $S$, in Zeichen $A_S := \bar S^\mo \bar A$. Man hat einen Homomorphismus\footnote{An Korrektor: Wie sieht der aus? Habe ich mir nicht aufgeschrieben.} $\iota_S(S) \subseteq A_S^\times$ und $\ker \iota_S = I_S$. Oft schreibt man schlampig wieder $S^\mo A$ und $\frac{a}{s}$ ($a \in A, s \in S$) statt $\bar S^\mo \bar A$ und $\frac{\bar a}{\bar s}$ ($a \in A, s \in S$).

\subsection{Satz} Sei $A$ ein kommutativer Ring und $S \subseteq A$ multiplikativ. Sei $B$ ein weiterer kommutativer Ring und $\varphi: A \to B$ ein Homomorphismus mit $\varphi(S) \subseteq B^\times$. Dann gibt es genau einen Homomorphismus $\psi: A_S \to B$ mit $\varphi = \psi \circ \iota_S$.

\proof Übung. \qed

\section{Primideale und maximale Ideale}

\subsection{Wiederholung} \label{fixed:2.4.1} Sei $R$ ein kommutativer Ring. Ist $E \subseteq R$, so ist $(E) := \{\sum_{i=1}^n a_i b_i \mid n \in \N, a_i \in R, b_i \in E\}$ das kleinste Ideal von $R$, welches $E$ enthält und man nennt es das von $E$ (in $R$) erzeugte Ideal\index{Ideal!erzeugtes} \LAref{3.3.9, 3.3.10}. Für $b_1,...,b_n \in R$ schreibt man auch $(b_1,...,b_n) := ({b_1,...,b_n}) = \{\sum_{i=1}^n a_i b_i \mid a_i \in R\}$. Ideale der Form $(b)$ mit $b \in R$ nennt man auch Hauptideale\index{Hauptideal} \LAref{3.3.11}. Es heißt $R$ ein Hauptidealring\index{Hauptidealring}, wenn $R$ ein Integritätsring ist, in dem jedes Ideal ein Hauptideal ist. $\Z$ und $K[X]$ ($K$ ein Körper, $X$ eine Unbekannte) sind Hauptidealringe \LAref{3.3.13, 10.2.2} oder \cite[§ 2.2, § 2.4]{Bosch2004}.

Ist $p \in R$, so heißt $p$ irreduzibel\index{irreduzibel} (in $R$), wenn
$$p \not\in R^\times \quad \& \quad \forall a,b \in R : (p = ab \Rightarrow (a \in R^\times \text{ oder } b \in R^\times))$$
und prim\index{prim} (in $R$), wenn
$$p \not\in R^\times \quad \& \quad \forall a,b \in R: (p|ab \Rightarrow (p|a \text{ oder } p|b)).$$

In einem Integritätsring ist jedes Primelement $\neq 0$ irreduzibel. Die Äquivalenzrelation $\eqhat$ auf $R$ ist definiert durch $a \eqhat b :\iff (a|b \text{ \& } b|a) \iff (a) = (b)$ ($a,b \in R$).

Setze $\hat a := \overset{\eqhat}{a}$ für $a \in R$. Fixiere $\P_R \subseteq R$ mit $\P_R \to \{a \in R \mid a \text{ prim}, a \neq 0\}/\eqhat, p \to \hat p$ bijektiv. (Z.~B.~$\P_\Z = \P = \{2,3,5,7,11,13,...\}$ für $R = \Z$.) Bezeichne $\N_0^{(\P_R)}$ die Menge der Funktionen $\alpha: \P_R \to \N_0$ mit endlichem Träger\index{Träger}\index{supp@\textit{supp}} $\supp(\alpha) := \{p \in \P_R \mid \alpha(p) \neq 0\}$.

Für jedes $\alpha \in \N_0^{\P_R}$ setze $\P_R^\alpha := \prod_{p \in \supp(\alpha)} p^{\alpha(p)}$. Man nennt $(c,\alpha) \in R \times \N_0^{(\P_R)}$ eine Primfaktorzerlegung\index{Primfaktorzerlegung} von $a \in R$, wenn $a = c \P_R^\alpha$. In Integritätsringen sind Primfaktorzerlegungen eindeutig. Es heißt $R$ ein faktorieller Ring\index{Ring!faktorieller}, wenn er ein Integritätsring ist, in dem jedes $a \in R\setminus\{0\}$ eine Primfaktorzerlegung besitzt. Jeder Hauptidealring ist faktoriell. In einem faktoriellen Ring ist jedes irreduzible Element prim. \cite[§ 2.4]{Bosch2004}

\subsection{Definition} Sei $R$ ein kommutativer Ring. Ein Ideal $\p$ von $R$ heißt Primideal\index{Primideal} von $R$, wenn
$$1 \not\in \p \quad \& \quad \forall a,b \in R : (ab \in \p \Rightarrow (a \in \p \text{ oder } b \in \p)).$$
Ein Ideal $I$ von $R$ heißt echt\index{Ideal!echtes}, wenn $1 \not\in I$ (oder äquivalent $I \neq R$). Ein Ideal $\m$ von $R$ heißt maximales Ideal\index{Ideal!maximales} von $R$, wenn $\m$ ein maximales Element der durch Inklusion halbgeordneten Menge aller echten Ideale von $R$ ist.

\subsection{Bemerkung} \label{fixed:2.4.3} Sei $R$ ein kommutativer Ring. Die in \ref{fixed:2.4.1} wiederholte Definition eines Primelements $p \in R$ kann man offensichtlich wie folgt lesen:
$$1 \not\in (p) \quad \& \quad \forall a,b \in R : (ab \in (p) \Rightarrow (a \in (p) \text{ oder } b \in (p))).$$
Es folgt für $p \in R$: $p$ Primelement $\iff$ $(p)$ ist Primideal

\subsection{Satz} Sei $I$ ein Ideal des kommutativen Ringes $R$. Dann gilt
\begin{enumerate}[label=(\alph*)]
	\item
		$I$ Primideal $\iff$ $R/I$ Integritätsring \quad und
	\item
		$I$ maximales Ideal $\iff$ $R/I$ Körper
\end{enumerate}

\proof Übung. \qed

\subsection{Korrolar} Jedes maximale Ideal eines kommutativen Rings ist ein Primideal.

\proof Jeder Körper ist ein Integritätsring. \qed

\subsection{Korrolar} Seien $A, B$ kommutative Ringe und $\varphi: A \to B$ ein Homomorphismus. Sei $\q$ ein Primideal von $B$. Dann ist $\p := \varphi^\mo(\q)$ ein Primideal von $A$.

\proof $\psi: A \to B/\q, a \mapsto \bar{\varphi(a)}^\q$ ist Hintereinanderschaltung der Homomorphismen $A \stackrel{\varphi}{\longrightarrow} B \stackrel{b \to \bar b^\q}{\longrightarrow} B/\q$ und daher ein Homomorphismus. Nach Isomorphiesatz \ref{fixed:2.1.17} ist $A/\ker\psi \eqtilde \im \psi$. Es ist $\psi$ ein Unterring des Integritätsrings $B/\q$ und daher auch ein Integritätsring. Somit ist auch $A/\ker\psi$ ein Integritätsring, das heißt $\ker\psi$ ein Primideal von $A$. Es gilt $\ker\psi = \{a \in A \mid \psi(a) = 0\} = \left\{a \in A \mid \bar{\psi(a)}^\q = 0\right\} = \{a \in A \mid \varphi(a) \in \q\} = \varphi^\mo(\q) = \p$. \qed

\subsection{Beispiel} Sei $K$ ein Körper. Im Polynomring $K[X,Y]$ ist $(X)$ ein Primideal, denn $K[X,Y]/(X) \eqtilde K[Y]$ ist ein Integritätsring (betrachte den Einsetzungshomomorphismus $K[X,Y] \to K[Y], p \mapsto p(0,Y)$ und wende den Isomorphiesatz \ref{fixed:2.1.17} an). Es ist $(X)$ kein maximales Ideal, denn $K[X,Y]/(X) \eqtilde K[Y]$ ist kein Körper. Dagegen ist $(X,Y)$ ein maximales Ideal von $K[X,Y]$, denn $K[X,Y]/(X,Y) \eqtilde K$ ist ein Körper (betrachte $K[X,Y] \to K, p \mapsto (0,0)$).

\subsection{Satz} In einem Hauptidealring ist jedes Primideal $\neq \{0\}$ ein maximales Ideal.

\proof Sei $R$ ein Hauptidealring und $\p \neq \{0\}$ ein Primideal in $R$. Sei $I$ ein Ideal von $R$ mir $p \subseteq I$. Zu zeigen: $I = \p$ oder $I = R$. Wähle $p,a \in R$ mit $\p = (p)$ und $I = (a)$. Die Bedingung $p \subseteq I$ bedeutet $(p) \subseteq (a)$, d.~h.~$p \in (a)$. Wähle $b \in R$ mit $p = ab$. Da $p$ gemäß \ref{fixed:2.4.3} prim ist und $R$ ein Integritätsring ist, ist $p$ irreduzibel in $R$. Also gilt $a \in R^\times$ oder $b \in R^\times$, also $I = (a) = R$ oder $I = (a) = (b^\mo p) \subseteq (p) = \p \subseteq I$. Also $I = R$ oder $I = \p$ wie gewünscht.


\numberfix{3}{0}{0}
\chapter{Körper \small\LAref{§ 4}}

\section{Endliche und algebraische Körpererweiterungen}

\subsection{Definition} Sei $L | K$ eine Körpererweiterung \ALref{\ref{fixed:2.3.11}}. Die Dimension $[L:K] := \dim_K L \in N \cup \{\infty\}$ des $K$-Vektorraums $L$ \LAref{§ 6.1} nennt man den (Körper-)Grad\index{Grad!einer Körpererweiterung}\index{Körpererweiterung!Grad} von $L$ über $K$ (nicht zu verwechseln mit dem Index aus \ref{fixed:1.3.19}!). Ist $[L:K] < \infty$ ($[L:K] = \infty$), so nennt man L endlich (unendlich) über $K$ und $L|K$ eine endliche\index{Körpererweiterung!endliche} (unendliche\index{Körpererweiterung!unendliche}) Körpererweiterung.

\subsection{Beispiel}
\begin{enumerate}[label=(\alph*)]
	\item
		$[K:K] = 1$ für jeden Körper $K$.
		
	\item
		$[K(X):K] = \infty$ für jeden Körper $K$.
		
	\item
		$[\C:\R] = 2$
\end{enumerate}

\subsection{Proposition} Sei $L|K$ eine Körpererweiterung von $V$ ein $L$-Vektorraum (und damit auch ein $K$-Vektorraum). Sei $A$ eine Basis des $K$-Vektorraums $L$ und $B$ eine Basis des $L$-Vektorraums $V$. Dann ist $A \times B \to AB := \{ab \mid a \in A b \in B\},~(a,b) \mapsto ab$ bijektiv und $AB$ eine Basis des $K$-Vektorraums $V$.

\proof Zu zeigen:
\begin{enumerate}[label=(\alph*)]
	\item
		$\spn_K AB = V$
		
	\item
		Für paarweise verschiedene $a_1,...,a_m \in A$ und paarweise verschiedene $b_1,...,b_n \in B$ sind $a_ 1b_1,...,a_1 b_n,...,a_m b_1,...,a_m b_n$ linear unabhängig.
\end{enumerate}

Zu (a). Für jedes $\lambda \in L$ und $b \in B$ gilt $\lambda \in \spn_K A$ und daher $\lambda b \in \spn_K Ab \subseteq \spn_K AB$. Daraus folgt $V = \spn_L B \subseteq \spn_K AB \subseteq V$.

Zu (b). Seien $\lambda_{ij} \in K$ ($1 \leq i \leq m, 1 \leq j \leq n$) mit $\sum_{i=1}^m  \sum_{j=1}^n \lambda_{ij} a_i b_j = 0$. Dann $\sum_{j=1}^n \left( \sum_{i=1}^m \lambda_{ij} a_i \right) b_j = 0$ und daher $\sum_{i=1}^m \lambda_{ij} a_i = 0$ für alle $j$, also $\lambda_{ij} = 0$ für alle $i,j$. \qed

\subsection{Sprechweise} Ein Zwischenkörper\index{Körper!Zwischenkörper} einer Körpererweiterung $L|K$ ist ein Unterkörper von $L$, der $K$ enthält.

\subsection{Korollar} Sei $F$ ein Zwischenkörper der Körpererweiterung $L|K$. Dann ist $L|K$ endlich genau dann, wenn $L|F$ und $F|K$ beide endlich sind, und in diesem Fall gilt die sogenannte "`Gradformel"'\index{Gradformel} $$[L:K] = [L:F][F:K].$$

\subsection{Definition} Sei $L|K$ eine Körpererweiterung. Dann heißt $a \in L$ algebraisch\index{algebraisch!-es Element} über $K$, wenn es $f \in K[x]\setminus\{0\}$ gilt mit $f(a)=0$ [das heißt, wenn $a$ nicht algebraisch unabhängig über $K$ ist, \ALref{\ref{fixed:2.2.3(a)}}]. Es heißt $L|K$ algebraisch\index{algebraisch!-e Körpererweiterung}, wenn jedes Element von $L$ algebraisch über $K$ ist.

\subsection{Beispiel}
\begin{enumerate}[label=(\alph*)]
	\item
		$\sqrt{2}$ ist algebraisch über $\Q$, denn $\left(\sqrt{2}\right)^2-2 = 0$.
		
	\item
		$\i$ und $\i+1$ sind algebraisch über $\Q$, denn $\i^2+1 = 0$ und $(\i+1)^2-2(\i+1)+2 = 0$.
		
	\item
		$K \in K(X)$ ist nicht algebraisch über $K$. ($K$ ein Körper.)
\end{enumerate}

\subsection{Definition} Sei $L|K$ eine Körpererweiterung und $a \in L$ algebraisch über $K$. Dann ist der Kern von $K[X] \to L,~f \mapsto f(a)$ ein Ideal von $K[X]$, welches von einem eindeutig bestimmten normierten Polynom erzeugt wird \LAref{10.2.4}, dem sogenannten Minimalpolynom\index{Minimalpolynom}\index{Polynom!Minimal-} $\irr_K(a) \in K[X]$.

\subsection{Proposition} Sei $L|K$ eine Körpererweiterung und $a \in L$ algebraisch über $K$. Dann sind für $f \in K[X]$ äquivalent:
\begin{enumerate}[label=(\alph*)]
	\item
		$f = \irr_K(a)$
		
	\item
		$f$ ist \textit{das} normierte Polynom kleinsten Grades mit $f(a)=0$.
		
	\item
		$f$ ist normiert und irreduzibel in $K[X]$ und es gilt $f(a)=0$.
		
	\item
		$f$ ist das Minimalpolynom des $K$-Vektorraumendomorphismus $\lambda_a : L \to L,~b \mapsto ab$.
\end{enumerate}

\proof ~

\underline{(a) $\Longrightarrow$ (b):} Klar

\underline{(b) $\Longrightarrow$ (c):} Gelte (b). Zu zeigen ist $f$ irreduzibel. Es gilt $f \in K[X]^\times = K^\times$, da $f(a)=0$. Seien $g,h \in K[X]$ mit $f = gh$. Zu zeigen ist $g \in K^\times$ oder $h \in K^\times$. Wegen $g(a)h(a) = (gh)(a) = f(a) = 0$ gilt $g(a) = 0$ oder $h(a) = 0$. Dann gilt aber $\deg g \geq \deg f$ oder $\deg h \geq \deg f$ und daher $h \in K^\times$ oder $g \in K^\times$.

\underline{(c) $\Longrightarrow$ (a):} Gelte (c). Wegen $f(a) = 0$ gilt dann $f \in (\irr_K(a))$\footnote{Korrektur: Hier fehlt doch was um die Klammern?}, das heißt, es gibt $g \in K[X]$ mit $f = g \irr_K(a) \in K^\times$. Letzteres ist unmöglich, also $g\in K^\times$ und sogar $g=1$, da $f$ und $\irr_K(a)$ beide normiert sind.

\underline{(a) $\iff$ (d):} Es reicht zu zeigen, dass für alle $g \in K[X]$ gilt: $g(a) = 0 \iff g(\lambda_a) = 0$ \LAref{10.2.18}. Dies folgt aus $(g(\lambda_a))(b) = (g(a))b$ für alle $b \in L$. \qed

\subsection{Proposition} Sei $L|K$ eine Körpererweiterung und $a \in L$ algebraisch über $K$. Dann ist $K[X]/(\irr_K(a))$ ein Körper und $K[X]/(\irr_K(a)) \to K[a],~\bar f \mapsto f(a)$ ein Isomorphismus. Insbesondere ist $K[a] = K(a)$ auch ein Körper und $\deg \irr_K(a) = [K(a):K]$.

\proof Nach dem Isomorphiesatz für Ringe und für $K$-Vektorräume liefert der Einsetzungshomomorphismus $K[X] \MAPepi K[a],~f \mapsto f(a)$ den Ring - und $K$-Vektorraumisomorphismus $K[X]/(\irr_K(a)) \to K[a],~\bar f \mapsto f(a)$.

Da $\irr_K(a)$ irreduzibel im Hauptidealring $K[X]$ ist, ist $K[X]/(\irr_K(a))$ nach \ref{fixed:2.4.9} (siehe auch \ref{fixed:2.4.10(b)}) ein Körper. Daher ist auch der dazu isomorphe Ring $K[a]$ ein Körper, das heißt $K[a] = K(a)$ \ALref{\ref{fixed:2.3.11(b)}}. Setzt man nun $d := \deg \irr_K(a)$, so bilden $\bar 1, \bar X, ..., \bar X^{d-1}$ offensichtlich eine Basis des $K$-Vektorraumes $K[X]/(\irr_K(a))$ und daher deren Bilder $1, a, ..., a^{d-1}$ eine Basis des $K$-Vektorraums $K[a] = K(a)$. Insbesondere ist $d = [K(a):K]$. \qed


\nocite{*}
\bibliographystyle{plaindin}
\bibliography{algebra_book.bib}

\printindex

\end{document}